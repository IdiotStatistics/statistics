%\chapter{統計量を使ったモデルとデータの比較}
\chapter{モデルにおける統計量の性質}
統計モデルからサンプリングを行った標本から、統計量を計算すると、その統計量より偏った値が出現する確率が得られる。この性質を利用して、モデルからサンプリングを行った標本について、統計量がより偏った値が一定値を下回るならば、このモデルからサンプリングされていないと判定を下し、ある特定の値より大きいならば、このモデルから得られた標本であると判断を下す。

%この確率が低いとき、標本がモデルからサンプリングされたものではないと一定の割合で判断を下すことにする。
%このことを利用して、標本をモデルからサンプリングしたと判断するものとそうでないものに仕分けを行う。

%言い換えるなら、ある母数を持つ統計モデルから標本がサンプリングされたのかい?そうじゃないのか?どっちなんだい?に答える方法の一つである。
%これを科学においては、ある母数をもつ統計モデルによって推測してもいいのかい?そうじゃないのかい?どっちなんだい?統計モデルが「ピー」と答える。
%\footnote{本章で説明する仮説検定とは、前者がモデルとの乖離の程度を調べる方法であり、後者は仮説の真偽を調べる数学的方法という点で完全に異なる。}。

\section{自己標本の批判}
統計モデルからサンプリングした標本の統計量が従う確率密度関数が理論的に求められる。
正規モデルの場合、以下の通りである。
\begin{equation*}
    Z = \frac{\sqrt{n}(\bar{x}-\mu)}{\sigma} \sim N(0,1)
\end{equation*}
この性質を利用することで、$Z$値よりも偏った値がどの程度の確率で出現するかが計算できる。
例えば、$Z=0$であれば、モデル内でこれ以上に偏った値が出現する確率は約$0.5$程度であり、よくある値であることがわかる。
一方、$Z=1.96$であれば、モデル内でこれ以上に偏った値が得られる確率は約$0.025$程度となり、なかなかレアな値であると思うことができる。
%ここで注意しなければならないのは、$p$値はモデル内における$Z$値よりも偏った値の出現する確率であるということである。
%言い換えれば、現実において$Z$値よりも偏った値の出現する確率についてはなんら言及できていない。
%統計量以上に偏った値がモデルにおいて出現する確率を$p$値という。



\begin{defi}
    統計モデルにおいて、ある統計量よりも偏った(大きいまたは小さな)統計量が得られる確率を$p$値と呼ぶ。
    %ある標本から求められた統計量以上に大きな値が得られる確率を$p$値と呼ぶ。
    %絶対にダメと判断されないときは、統計モデルを採択(棄却の対義語)すると宣言しない。
    %統計モデルが棄却されるのは、統計モデルの仮定によって変化する。本書の範囲内であれば、統計モデルの母数、分布関数、独立同一の分布関数からサンプリングされたことによる。
    %最尤統計モデルにおいて、棄却されない統計モデルの母数の範囲を信頼区間といい、棄却されるモデルの母数の範囲を棄却域という。
\end{defi}

あるモデルでの標本の得られにくさを、$p$値に変換して検討することが可能である。
$p$値が小さいなら、統計量$Z$値以上に偏った値の得られる確率が低いということであるので、
$Z$値の元の標本もそもモデルからは得られにくいということを示す。
%つまり、モデルから標本の得られにくさの指標の一つが$p$値であるとも言え、$p$値が小さいほど、その標本はそのモデルから得られにくい。

\if 0
\begin{center}
    標本$\rightarrow$ 統計量 $\rightarrow$ $p$値($p$値の小ささが標本のモデル上での得られにくさの指標になる)
\end{center}
\fi




\subsection{$p$値の計算練習}
%$Z(\bar{x},\mu)$以上の値が得られる確率}
$Z(\bar{x},\mu)\sim N(0,1)$により、$Z$以上の値が得られる確率の計算を練習してみよう。つまり、以下をコンピュータに計算させる。
\begin{equation*}
    p = \varPhi(Z(\bar{x},\mu)>x)
\end{equation*}
正規モデル$M(\mu=168;\sigma^2=6.8)$から得た標本の統計量を、$\bar{x}=172.4$、サンプルサイズを$n=10$とする。
$Z$値は$Z(\bar{x},\mu)=2.04$となる。これを元に、以下のスクリプトを実行すると、$p$値が約0.04程度であることがわかる。

\begin{lstlisting}
xbar = 172.4
mu = 168
sigma2 = 6.8**2
n=10
Z = np.sqrt(n)*(xbar-mu)/np.sqrt(sigma2)
print(Z)
p=1-norm.cdf(Z,0,1)
print(p*2)
\end{lstlisting}



\subsection{自己標本の否定確率}
%ここから、いくつか数学的な構造について紹介する。
%この内容は科学において使うのは非常に難しい。
%例えば、有意水準$\alpha$や検定量$\beta$などを決定することはできない。
%$Z_i$がモデル由来であるかを自己標本の批判を元に判断する。

正規モデル$M$において、$M$の標本を得てその統計検定量Zが極端な値を取るとき、$M$の標本ではないと判定する。
ここで、極端な値とは、$Z$が標準正規分布に従うことから、標準偏差の2よりも偏った値である場合に、$M$の標本ではないとする。
このとき、標準正規分布表を参照すれば$P(|Z|>2)=0.02275\times 2=0.046$程度となる。

標準偏差の代わりに確率を元に判定を行う。
$P(|Z|>x)=0.046$はキリが悪いので代わりに、$P(|Z|>x)=0.05$を設定する。
標準正規分布表を参照すれば$P(|Z|>x)=0.05$となるのは$x=1.96$程度である。
この値よりも大きな$|Z|$となる標本は、$M$の標本ではないと判定する。

あるモデルの統計検定量がより偏った値を取る確率$0.05$を他の統計モデルでも適用すれば、さまざまなモデルで統一的に判定が行える。また、あるサンプルサイズの標本を100個モデルからサンプリングした場合、$95$回はモデルから生成されたものと判断し、残りの5回についてはモデルから生成されていないと判断する。


\begin{defi}
 モデルからサンプリングされた標本のうち、モデルから生成されたものではないと判定する割合を$\alpha$とし、有意水準と呼ぶ。
 言い換えれば、$\alpha$値は、統計モデルからサンプリングされた値について、これが元の統計モデルからサンプリングであることを判定する閾値\footnote{閾値(読み:いきち)=限界値}である。
\end{defi}

%本書では、上記の歴史的な事情にしたがって、$\alpha=0.05$を利用して、理論計算を行う。

%これは、モデル$M$の標本であるはずなのに、外れた値であれば、そのモデル$M$
%モデルが生成したはずの標本であるが、閾値を決めてモデルから生成されたものではないとするのである。
%モデルが自身から得られた標本を批判するのであるから、自己の標本を批判するのである。
%これは、モデルを元に、標本がモデルにより予測できるかどうかを考えている。


%具体的には、正規モデルを利用すれば、その統計量$Z$が$N(0,1)$に従うことがわかっている。
%$Z$の値が偏った値になっていれば、その出現頻度は低くなるので、$P(|z|<Z)=95/100$となる$Z$を計算する。
%この$Z$は具体的に計算でき、$Z_{0.95}=1.96$である。
%ここから、$|z_i|<Z_{0.95}=1.96$となる$z_i$の個数を数えればおよそ$95$になる。
%また、$|P(|z_i|>0.95)| > 1.96$ならば、$|z_i|>Z_{0.95}=1.96$である。


%統計モデルの分布関数が変化すれば、その統計モデルにおける信頼区間・棄却域の値も変わる\footnote{中心極限定理を利用し、統計量の出現範囲を近似することが多い。}。
%実際のデータが$\alpha\%$の割合で棄却されるということではない。モデルから生成された標本であるのに、この統計モデルから生成されていないと疑いをかける。

\subsection{母数平均の変化に応じた信頼区間}
%$95\%$信頼区間と$p$値の関係}
正規モデル$M$においてその統計検定量$Z$について、式変形を行う。
\begin{eqnarray}
    & &|Z| < z_{0.025}( = 1.96 ) \notag \\
    &\rightarrow& \frac{\sqrt{n}|\bar{x}-\mu|}{\sigma} < z_{0.025} \notag\\
    &\rightarrow& \mu-\frac{\sigma}{\sqrt{n}}z_{0.025} < \bar{x} < \mu+\frac{\sigma}{\sqrt{n}}z_{0.025} \label{confidence_interval_eqn}
\end{eqnarray}

\begin{defi}
式\eqref{confidence_interval_eqn}の$\bar{x}$の区間を信頼区間といい、次の式で定義される。
\begin{equation*}
    A=\{x;\mu-\frac{\sigma}{\sqrt{n}}z_{0.025} < x < \mu+\frac{\sigma}{\sqrt{n}}z_{0.025} \}
\end{equation*}
これ以外の区間を棄却域と言う。ここでは、$R=\mathbb{R}\backslash A$が棄却域である。
% Comment
%信頼区間の定義は、統計検定量の範囲ではなく、平均値の範囲
\end{defi}
信頼区間の範囲は、サンプルサイズ$n$、有意水準$\alpha$およびモデルの母数$\mu,\sigma^2$により決まる。

まず、$\mu$の変化に応じて、信頼区間が変化する様子を確かめる。
図\ref{fig:confidence_interval_model}には、モデル毎の平均値と信頼区間を描いた。$\mu$の大きさにによらず信頼区間の幅は同じである。
各$\mu$に対して、信頼区間の内側で$\bar{x}$が$95\%$の確率で見つかることを統計モデル$M(\mu)$が推測する。
この外側にある$\bar{x}$になる標本については統計モデルの標本ではないと判定を下す。

\begin{figure}
    \begin{center}
        \includegraphics[width=10cm]{./image/04_/confidence_interval_model.pdf}
        \caption{横軸にモデルの母数$\mu$、縦軸に、モデルが予測する平均値$\bar{x}$、エラーバーに$95\%$信頼区間を描いた。$N=10,\sigma^2=6.8^2$}
        \label{fig:confidence_interval_model}

    \end{center}
\end{figure}



\section{統計量をもとにしたモデル間類似度(検出力)}
母数の異なる二つの統計モデル$M_a,M_b$について考察する。
$M_a$の信頼区間内の統計量が$M_b$において出現する確率を検出力という。
%言い換えれば一方で出現する統計量が他方のモデルにおいて出現する確率である。
これは、$M_a$から$M_b$への統計量を元にしたモデル間類似度と言える。

\subsection{検出力の定義}
$M_a$におけるある統計量についてその信頼区間を$A_a$とするとき、$M_b$において$A_a$内の統計量が出現する確率を$\beta$とする。
具体的には以下の式で表される。
\begin{eqnarray*}
    & &P_a(x \in R_a) = \alpha\\
    & & P_b(x \in A_a) = P_b(x\notin R_a )=\beta
\end{eqnarray*}
ここで、$R_a,A_a$はそれぞれ統計モデル$M_a$の棄却域、信頼区間を表し、$P_a,P_b$は、それぞれ統計モデル$M_a,M_b$におけるある統計量がしたがう分布の密度関数。
$1-\beta$を検出力という\footnote{検出力を検定力または統計力と呼ぶこともある。\\ \url{https://id.fnshr.info/2014/12/17/stats-done-wrong-03/}}。


検出力$1-\beta$は、二つの異なるモデルを統計検定量を基準に比較するための指標である。
%$$1-\beta$が$1$に近ければ、
二つの統計モデルの母数がよく一致するならば、$\beta$は$1-\alpha$に近い値を取り、また、$M_a$に対する$M_a$の検出力は、$1-\alpha$である。
一方、モデルの母数が一致していないならば、$\beta$は0に近い値を取る。
%$M_a$を棄却する閾値を低く設定すると、$\beta$は大きな値になる。

$\beta$は$\alpha$を変数にする関数になっており、
$\alpha$を0に近づけていくと、信頼区間は徐々に大きくなり、$1-\beta$は小さくなる。
$\alpha$を大きくすると、信頼区間は徐々に狭くなり、$1-\beta$は大きくなる。



\subsection{正規分布モデルの検出力}
正規モデルをつかって、$P_a(x \in R_a),P_b(x \in A_a)$を計算する。
$\sigma^2$がすでに与えられた正規モデルを$M(\mu;\sigma^2)$とし、$M_a=M(\mu_a),M_b=M(\mu_b)$とする。
$M_a$または、$M_b$からサンプリングされた確率変数の平均値は、それぞれ$\bar{x}_a\sim N(\mu_a,\sigma/n)$、$\bar{x}_b\sim N(\mu_b,\sigma/n)$である。
また、$M_a$の信頼区間$A_a$は、$|\bar{x}_a|<\mu_a+\sigma / \sqrt{n}z_{2.5\%}$である。
このとき、$P_a$を$N(\mu_a,\sigma/n)$の確率密度関数とすると、
\begin{equation*}
    P_a(x \in A_a) = 1-\alpha
\end{equation*}
であるのは定義から明らか。
また、$P_b$を$N(\mu_b,\sigma/n)$の確率密度関数とすると、
\begin{equation*}
    P_b(x \in A_a ) = \beta
\end{equation*}
である。

\begin{figure}
\begin{center}
 \includegraphics[width=15cm]{./image/04_/power_of_a_test_2.pdf}
 \caption{統計モデル$M_a,M_b$から計算された統計量$\bar{x}$の確率分布$P_a,P_b$。(a)灰色の範囲は$M_a$の信頼区間。(b)灰色の領域は、$1-\beta$の領域を示している。$\beta$の面積が非常に小さいので、グラフ上に描画できていない。(c)$\mu_b$が$\mu_a$に近いときの$\beta$と$1-\beta$の領域。(d)灰色の範囲の面積が$\alpha$を示している。}
 \label{fig:power_of_test_alpha_beta}
\end{center}
\end{figure}


図\ref{fig:power_of_test_alpha_beta}に検出力と$\alpha$の領域を図示した。
%$M_a$の$95\%$信頼区間は、$|\mu|<\mu_a+z_{0.025}\frac{\sigma}{\sqrt{N}}$とした。
信頼区間は、図\ref{fig:power_of_test_alpha_beta}(a)において灰色で塗った$x$軸の範囲である。$\alpha$は図\ref{fig:power_of_test_alpha_beta}(d)の灰色で塗りつぶした領域の面積である。
検出力$1-\beta$は、$M_b$における$M_a$の信頼区間の外側の領域の面積なので、図\ref{fig:power_of_test_alpha_beta}(b)の濃い灰色の範囲である。
図\label{fig:power_of_test_alpha_beta}(c)は、$M_b$の母数$\mu_b$を$M_a$の母数$\mu_a$に近付けたときの$\beta,1-\beta$。



\begin{figure}
    \begin{center}
        \includegraphics[width=15cm]{./image/04_/power_of_a_test_3.pdf}
        \caption{統計モデル$M_a,M_b$から計算された統計量$\bar{x}$の確率分布$P_a,P_b$。(a)$\mu_a,\mu_b$のサンプルサイズ$1$の平均値がしたがう確率密度関数$N(\mu_a,\sigma^2/1),N(\mu_a,\sigma^2/1)$。(b)(a)と同じ$\mu_a,\mu_b$に対して、サンプルサイズを$30$にした場合の確率密度関数。(c)$\mu_a,\mu_b$が(a)よりも近いときの$\bar{x}$の確率密度関数。(d)(c)と同じ$\mu_a,\mu_b$に対してサンプルサイズを$30$にした場合の$\bar{x}$の確率密度関数。}
        \label{fig:power_of_test_alpha_beta_sample_size}
    \end{center}
    \end{figure}

\paragraph{サンプルサイズによる$\beta$の変化}

$\alpha$、$M_a$の母数平均$\mu_a$、$M_b$の母数平均$\mu_b$を固定したまま、サンプルサイズを変化させ,
$\beta$の変化を図\ref{fig:power_of_test_alpha_beta_sample_size}に示す。$\bar{x}$が従う分布($N(\mu,\sigma^2/n)$)の分散がサンプルサイズによって変化することは明らかである。このことから、サンプルサイズが大きくなると、信頼区間は徐々に狭くなり、$1-\beta$は大きくなる。サンプルサイズが小をさくすると$1-\beta$も小さくなる。

\paragraph{モデルの母数による$\beta$の変化}
$\mu_a$を固定し、$\mu_b$を変化させたときの検出力$1-\beta$を図\ref{fig:power_of_test_N_mu0_variable}に示した。
%モデルの母数平均$\mu_a,\mu_b$と$\beta$の関係について考察する。
$\mu_a$と、$\mu_b$が一致していれば、$P_b(x \in A_a )$は$1-\alpha$になる。
$\mu_b$が$\mu_a$から離れていくと、$P_b(x \in A_a)=0$に近づいていく。

\paragraph{$\alpha$による$\beta$の変化}
$\alpha$が小くなれば、信頼区間の幅が大きくなり、$1-\beta$も小くなる。
$\alpha$が大きくなれば、信頼区間の幅が小くなり、$1-\beta$も大きくなる。

%ここまでをまとめると、検出力は、モデルの母数、サンプルサイズ、有意水準によって変化する。


\begin{figure}
    \begin{center}
        \includegraphics[width=15cm]{./image/04_/power_of_test.pdf}
        \label{fig:power_of_test_N_mu0_variable}
        \caption{$\mu_a$を変数にしたときの検出力(検出力関数)。}
    \end{center}
\end{figure}

\subsection{$\beta$の代数計算}
正規モデル$M_a,M_b$を使って、$\beta$を計算する。
$M_a$の信頼区間は、
\begin{equation*}
    -z_{0.025}\leq \frac{\sqrt{n}(\bar{x}-\mu_a)}{\sigma}\leq z_{0.025}
\end{equation*}
より、
\begin{equation*}
    A_a = \{ x ; \mu_a -\frac{\sigma}{\sqrt{n}}z_{0.025} \leq x \leq \mu_a +\frac{\sigma}{\sqrt{n}}z_{0.025} \}
\end{equation*}
である。ここで、$a=\mu_a -\frac{\sigma}{\sqrt{n}}z_{0.025},b = \mu_a +\frac{\sigma}{\sqrt{n}}z_{0.025} $とおく。棄却域は$A_a$以外の確率変数である。$M_b$の標本平均$\bar{x}_b$は、$N(\mu_b,\frac{\sigma^2}{n})$に従うので、その確率密度関数において、$A_a$が出現する確率が$\beta$である。
%ここで、$\frac{\sqrt{n}(\bar{x}_b-\mu_b)}{\sigma}\sim N(0,1)$である。
このことを利用すると、
$a,b$は、$N(\mu_b,\frac{\sigma^2}{n})$の確率変数だとすると、$a,b$を標準正規分布へ規格化したときの確率変数をそれぞれ$A,B$とする。
すると、$A$は以下の計算式により求められる。
\begin{eqnarray*}
    A &=& \frac{\sqrt{n}(a-\mu_b)}{\sigma} \\
    %&=& \frac{\sqrt{n}(\mu_a-\frac{\sigma}{\sqrt{n} z_{\alpha/2}})}{\sigma}\\
    &=& -z_{\alpha/2}+\frac{\sqrt{n}}{\sigma}(\mu_a-\mu_b)
\end{eqnarray*}
同様に、$B$は以下の通り。
\begin{eqnarray*}
    B &=& \frac{\sqrt{n}(b-\mu_b)}{\sigma} \\
    %&=& \frac{\sqrt{n}(\mu_a-\frac{\sigma}{\sqrt{n} z_{\alpha/2}})}{\sigma}\\
    &=& z_{\alpha/2}+\frac{\sqrt{n}}{\sigma}(\mu_a-\mu_b)
\end{eqnarray*}
以上より、確率密度関数$N(0,1)$において、$-z_{\alpha/2}+\frac{\sqrt{n}}{\sigma}(\mu_a-\mu_b) \leq x\leq  z_{\alpha/2}+\frac{\sqrt{n}}{\sigma}(\mu_a-\mu_b)$の間での積分値が$1-\beta$である。


\paragraph{計算}

$d=\frac{\mu_a-\mu_b}{\sigma}$とおく。$d=0.6,n=9$とする。このときの$\beta$を計算してみる。$N(0,1)$において、$-z_{\alpha/2} -0.6\sqrt{n} \leq x \leq z_{\alpha/2} +0.6\sqrt{n}$の区間で積分する。

\begin{lstlisting}
A,B = norm.interval(0.95,0.,1)
N = 9
d = 0.6
a,b = A+d*np.sqrt(N),B+d*np.sqrt(N)
print(a,b)
norm.cdf(b,0,1)-norm.cdf(a,0,1)
\end{lstlisting}

答えは、$0.564$

\subsection{最尤モデルでの$\beta$の計算}
\subsubsection{データを元にしたモデルとモデルの類似度}
統計モデルAを$M(\mu=170)$とし、統計モデルBを$M(\bar{X})$とする。ここで、$\bar{X}$は、無作為抽出によって得られた標本の平均であり、標本の大きさを$100$とする。
モデルA,Bの間の検出力が計算可能である。
$d=\frac{170-\bar{X}}{6.8}$、$n=100$であるので、$\bar{X}=168$を得たとすると、
\begin{lstlisting}
A,B = norm.interval(0.95,0,1)
N = 100
d = (170-168)/(6.8)
a,b = A+d*np.sqrt(N),B+d*np.sqrt(N)
print(a,b)
norm.cdf(b,0,1)-norm.cdf(a,0,1)
\end{lstlisting}
その検出力は、$0.163$


\section{過誤のまとめ}
これまでの議論をまとめる。モデル$M_a$からサンプリングを行った標本について、モデル$M_a$に関する標本であるかを判定する。モデルから生成された標本であるが、偏った統計量ならば、モデルから生成されていないと判断する。この頻度を$\alpha$とした。このように、モデル$M_a$から生成されたのに、統計量の出現頻度から、このモデルから生成したものではないと言う誤った判断を行う事になる。この判断の間違いを第1'の過誤と呼ぶ。

次に、モデル$M_b$からサンプリングによって得られた標本が、別のモデル$M_a$からサンプリングされものであるかどうかを判定することを考える。この場合、統計量が$M_a$の信頼区間含まれているかどうかを確認し、含まれていない場合には、モデル$M_a$からサンプリングされた標本ではないと判定する。問題は、統計量が信頼区間に含まれている場合である。この場合、実際には、$M_a$からサンプリングされていないのにもかかわらず、誤って$M_a$からサンプリングされた標本であると判断することになる。この誤った判断を第2'の過誤と呼ぶ。以上のことをまとめると、表\ref{table:type_error}のようになる。

\begin{table}[hbtp]
 \caption{モデル$M_a$による自己標本批判}
 \label{table:type_error}
 \centering
 \begin{tabular}{ccc}
  \hline
  &  $M_a$の信頼区間に &  $M_a$の信頼区間に \\
  & 標本の統計量が入っていない & 標本の統計量が入っている \\
  \hline \hline
  モデル$M_a$の標本  & モデル$M_a$の標本ではないと判定  & モデル$M_a$の標本と判定 \\
  & (第1'の過誤) & \\
  %  & $\alpha$ &  $1-\alpha$\\
  モデル$M_b$の標本  & モデル$M_a$の標本ではないと判定  & モデル$M_a$の標本であると判断 \\
  & & (第2'の過誤) \\
  %& $1-\beta$ & $\beta$ \\
  \hline
 \end{tabular}
\end{table}


\begin{SMbox}{正解と回答の違い}
 あるデータ群に対してそのデータの特徴を元に、YesまたはNoとアノテーションをつける。
 データからそのYesまたはNoを予測する手順を開発する。
 その手順によって得た回答と、正解(真の値)の一致と不一致は以下の通りになる(表\ref{table:Yes_no_answer})。
 回答と一致したら、True、一致しないならFalse。
 Yesと予測したらPositive、Noと予測したらNegativeとする。
 回答がYesな問題に、Yesと答えることは(手順が正しい予測を行なった)、True Positiveといい、Noと答える(手順が間違えた予測を行なった)ことはFalse Negativeという。回答がNoな問題に、Yesと答えることを、False Positive、また、Noと答えることをTrue Negativeという。

統計モデル$M_a$により、標本を$M_a$のものと$M_b$のものに分ける作業をおこなう。 具体的には、モデル$M_a$の標本にYesを対応ずけ、モデル$M_b$の標本にNoを対応付ける。標本を元に、YesまたはNoを判定する手順をモデル$M_a$において信頼区間に統計量が入っているかいなかをもとに判断する。この問題において回答がFPとなったものが第1'の過誤であり、FNとなったものが第2'の過誤である。
\end{SMbox}

\begin{table}[hbtp]
 \caption{正解と回答の違い}
 \label{table:Yes_no_answer}
 \centering
 \begin{tabular}{ccc}
  %\hline
  &  負例(真の値) & 正例(真の値)  \\
  \hline \hline
  正例(予測値) &  偽陽性(FP)  & 真陽性(TP)\\
  &予測が外れた & 予測が当たった\\
  負例(予測値) & 真陰性(TN) & 偽陰性(FN)\\
  & 予測が当たった & 予測が外れた\\
  \hline
 \end{tabular}
\end{table}

\subsection{サンプルサイズの設定}
統計モデル$M_a$により、標本を$M_a$のものと$M_b$のものに分ける作業をおこなう。
このとき、なるべくなら第2'の過誤を少くしたい。
これは、$1-\beta$をなるべく大きく設定ことで解決できる。
この作業では、すでにモデルが存在しているので、モデルの母数は固定である。
また、有意水準についても$\alpha$と決定しているものとする。
$1-\beta$を満せるように変更できる変数は、サンプルサイズのみである。
ここでは、具体的な計算を行う。

%$\beta$の数値、モデルの母数、有意水準を固定したとき、必要なサンプルサイズを計算することができる。%ここでは、$\mu_a,\mu_b$が固定されている状況を考える。
%検出力$1-\beta$は$1$に近いほど、
%統計検定量を基準にして$M_a,M_b$の違いを。
%あらかじめ決めたおいた基準の$1-\beta$を設定し、それ以上の$1-\beta$となるサンプルサイズを推測する。
%サンプルサイズが小さければ、$M_a$と$M_b$の違いは統計検定量を基準にして曖昧であり、サンプルサイズが大きくなると、統計量の出現頻度の違いが明らかになる。
\subsubsection{サンプルサイズ}
$M_a$と$M_b$の母数平均の差$d$と検出力を指定したときに、$M_a,M_b$間の検出力をある値以上にするための最小のサンプルサイズが計算できる。
$\beta=0.1,d=0.8$とし、この$\beta$を満たすように$N$を計算した。

\begin{lstlisting}
A,B = norm.interval(0.95,0.,1)
beta = 0.1
d = 0.8
for N in range(10,200,2):
    a,b = A+d*np.sqrt(N),B+d*np.sqrt(N)
    beta_ = norm.cdf(b,0,1)-norm.cdf(a,0,1)
    if beta_ < beta:
        break
print(N)
\end{lstlisting}
二つのモデルの間の類似度を$0.9$以上にするために必要な最小のサンプルサイズは、$18$であることがわかる。

\section{自己否定の誤推定}
統計モデルからサンプリングされた標本を、統計量により元のモデルからサンプリングされたかどうかを判断するとき、そのモデルからサンプリングされた標本ではないと想定以上に判断してしまうことがある。言い替えると、複数の標本のうち$\alpha$をモデルの標本ではないと判断したいにもかかわらず、適切な処理や計算を行わないことで、$\alpha$以上におおくの標本にこのモデルの標本ではないと判断をくだすことになる。
そのような処理や計算方法について説明をおこなう。

%このような過度な推定は、二つの要素に分解できる。
%不適切な統計量を使用することで、棄却域と統計量の違いにより生じる$\alpha_1$、そして、様々な種類の途中処理を噛ませて計算した結果生じる$\alpha_2$である($0<\alpha_2 \leq 1$)。

%$\alpha_2$は、$\alpha\times 2$以上になる場合、軽視されることはないが、
%$\alpha_1$が同程度の隔たりになる場合においては無視され、$\alpha_1$は$\alpha_2$よりも軽視されがちである。
%この過誤は2つの要因に分解でき、\footnote{$\alpha_2$は$\alpha_1$に関係するので実際には、分解できない。気持ちとしては、$\alpha_2$は、$\alpha_1$を変数に持つ関数である$\alpha=\alpha_2(\alpha_1)$。}、
%$\alpha_2=\alpha$となっていれば、有意水準$\alpha$の検定ができる。

%統計モデルに対して不適切な統計量を使ってモデルの検証を試みると、第一の過誤が変化することがわかっている。


\subsection{どんな統計モデルでも$T$統計量で調べよう}
%$\alpha_1$は、統計モデルと、その統計量の関数になっており、言い換えれば、統計量が統計モデルの中で設計通りの振る舞いをしているかを測る指標である。
正規モデルと統計量$T$を使うと、$T$が信頼区間の中にある確率は$\alpha$ である。一方で、指数モデルを使い、統計量$T$を使った場合、統計量$T$が信頼区間の中にある確率は$\alpha$よりも多くなる。
ここでは、統計モデルの分布の仮定が正規分布以外の場合においても、$T$統計量を使ってモデルのサンプルを一定の有意水準でモデルの物であるか検証可能かを調べる。
%%このことを数値計算により確かめる。

次の統計モデル$M_E(\lambda)$を構築する。
\begin{enumerate}
    \item $X_1,X_2,\cdots,X_n ,i.i.d\sim F$
    \item $F$は指数分布
    \item 指数分布の母数は$\lambda$
\end{enumerate}
母数$\lambda=1$とした統計モデルを$M_E(1)$とする。
$M_E(1)$からサンプリングした確率変数$x_1,x_2,\cdots,x_n$から次の統計量を計算する。
\begin{eqnarray*}
    T &=& \frac{\bar{x}-1}{\sqrt{\frac{\sigma^2}{n}}} \\
 &=& \sqrt{n}\frac{\bar{x}-1}{\sigma}%\sqrt{\frac{\sigma^2}{n}}}
\end{eqnarray*}

正規モデルから得た標本であれば、$T \sim t(n-1)$である。
今回は、指数モデルであるため、このようにはならないはずである。
しかし、これが成り立つと考えて、計算をおこなうとどのようになるだろうか。

\subsubsection{数値計算}
$T$値が$t(n-1)$の棄却域に入っている頻度を数値計算により計算する。
$M_E(1)$または、正規モデル$M(\mu=0,\sigma^2=1)$からある一定のサンプルサイズの標本を$10^5$回取得する。$T$値を計算し、$T$値より偏った値が得られる確率$p$を計算する。
以上の数値計算を、二つの条件において行う。
\begin{itemize}
 \item (実験$1$)サンプルサイズを$n=10$とし、$p$値を計算する。その$p$値の分布の偏りを調べる。
 \item (実験$2$)サンプルサイズ$n$を4つの条件$n=(3,10,30,50,100)$とし、それぞれのサンプルサイズ毎に、上で説明した数値計算を行い、$p$が想定している有意水準$\alpha=0.05$を越えない割合を計算する。%($|T|>1.96$となるサンプルの割合)。
\end{itemize}
%以上を数値計算を行なった。

正規モデル$M(0,1)$の場合、$T$値は$t(n)$分布に従ので、$p<0.05$となる頻度も、$5\%$程度になることが期待される。
一方で、指数モデル$M_E(1)$の場合、$T$は$t(n-1)$分布に従わない。このことから、実験$1$では、$p$値の分布が偏り、一様分布からずれるそして、実験$2$では、$p<\alpha$となる標本が、$0.05$とは異ることが予想できる。

\begin{figure}
 \begin{center}
  \includegraphics[width=15cm]{./image/04_/N_10_Exponn_T_test.pdf}
  \caption{サンプルサイズ$10$での$p$値の累積分布。実践は、正規モデル。破線は、指数モデル。}
  \label{fig:model_dependent_T_test}
 \end{center}
\end{figure}


\begin{figure}
 \begin{center}
  \includegraphics[width=15cm]{./image/04_/t_test_expon_norm.pdf}
  \caption{それぞれの分布から得た標本の$T$値から計算した$p$値で、$p<0.05$以下になる割合。黒い破線が指数分布での数値計算の結果。灰色の破線が正規分布での数値計算の結果。}
  \label{fig:t_test_expon_norm}
 \end{center}
\end{figure}

実験$1$の結果を図\ref{fig:model_dependent_T_test}に示した。正規モデルであれば、$T$値は正規分布するので、$p$値の分布は一様分布する。$p$値の累積分布が傾き$1$の直線の上にあるので、このことが確かめられる。
一方で、指数モデルにおける$p$値の累積分布は、傾き$1$の直線の上にはないことがわかる。
このことから、指数モデルでは期待した通りのことが起きないことが判った。

実験$2$の結果を図\ref{fig:t_test_expon_norm}に示した。正規分布から標本を得た場合、$p<0.05$になる割合は、サンプルサイズに依存せず、$5\%$程度であり、期待通りである(図\ref{fig:t_test_expon_norm}灰色の点)。
一方で、指数分布から標本を得た場合、$p<0.05$になる割合はサンプルサイズに応じて変化しており、また、どのサンプルサイズでも$p<0.05$となる割合は$5\%$より多い(図\ref{fig:t_test_expon_norm}黒色の点)。

このことから、指数モデルのでは、設定した有意水準$0.05$よりも高い頻度で標本がモデル由来でないと判定される。
%$p<\alpha$としても、$p>0.05$であることがわかり、統計量を正しく選ばなかったことで、自己否定の過誤がとがわかる。

\if 0 
これはいらない
\begin{SMbox}{サンプルサイズがxx以上あるから$t$検定}
        サンプルサイズがある値以上あるので、中心極限定理により、$t$検定が利用できるというものもある\footnote{http://id.ndl.go.jp/bib/024660739}。このロジックが読み込めなかったので、その謎を明らかにすべく我々はアマゾンの奥地へ向かった。

        %サンプルサイズが1以上であれば、$t$検定を行うことは原理的には可能である。
        データが指数分布的であるときに、$t$検定を使うときに生じる問題は上でみた通りであり、$p<0.05$となる標本の割合が多くなっているので、間違った推測をする可能性が高くなる。
        他の分布関数でもおそらく同じような現象が現れる。
        このことから、我々は「$t$検定が利用可能である」は正確ではなく、「$t$検定を使うことができるが、間違った推測である確率が高くなる」ということだと推察した。

        %サンプルサイズが大きくなれば、$\alpha_1$は小さくなる。
        業界によっては、サンプルサイズが$xx$以上であれば、過誤を無視して良いというふうに言われることもある。実際には、設計したモデルと
\end{SMbox}
\fi
 
\subsection{検定を繰り返し使おう}
%ここまでは、一つの標本に対して、統計モデル$M(\mu)$により推測できるかを考えていた。
%ここでは、$\sigma^2=10^2$とした正規モデル$M(\mu)$によって複数の標本について推測できるかを仮説検定を指標にし考える。
%ここでは、複数の標本について、$M(\mu)$により推測できるかを
正規モデル$M(\mu)$において、$100$個の標本を集め、それらの統計量が信頼区間に入っているかどうかを調べると、およそ$100\alpha$個については信頼区間の外にある。
ここで、標本を得るたびにその統計量が信頼区間にあるかどうかを調べてみると、やはり$100\alpha$個が信頼区間のなかにある。

これを拡張し、二組の標本を100個得て、それら両方の統計量が信頼区間の中に入っているのは$100\alpha$個だろうか。計算してみる。
具体的には、正規モデル$M(\mu)$において、複数の標本が信頼区間に入っているかをたしかめ、それらが正規モデルのサンプルであると考えることができるだろうか。
言い替えると、複数の標本のうち少なくとも一つは信頼区間に入っていないなら、正規モデルの標本ではないと判断すると、それは想定通り複数の試行の内$\alpha$程度がその判断に引っ掛るだろうか。

標本が$3$個あるとする。このとき、それぞれの標本の統計量$T$が信頼区間に入っている確率は、$(1-\alpha)$である。全ての標本の統計量$T$が信頼区間に入っている確率は、その積$(1-\alpha)\times(1-\alpha)\times(1-\alpha)=(1-\alpha)^3$である。
一方で、棄却される確率$p'$は、$1-(1-\alpha)^3$である。

\begin{table}[hbtp]
    \caption{標本数に応じた$p'$}
    \label{table:multiple_test_reject_prob}
    \centering
    \begin{tabular}{lcr}
      \hline
      標本数  & $\alpha=0.05$  &  $\alpha=0.01$ \\
      \hline \hline
       1 & $0.05$  & $0.01$ \\
       2 & $0.0975$ & $0.0199$\\
       3 & $0.142$ & $0.0297$\\
       4 & $0.185$ & $0.0394$\\
    \end{tabular}
  \end{table}
表\ref{table:multiple_test_reject_prob}は、標本数に応じた$p'$である。標本数が大きくなるについれて、$p'$が大きくなることがわかる。
これは、標本のペア数が多いと、標本がモデルのものではないと判定されやすくなることを示している。

%$\alpha_1$がレベル$\alpha$の検定になっていない場合、$\alpha_2$はさらに有意水準$\alpha$から隔たりの多い数値になる。

\subsubsection{数値計算}
具体的に数値計算を行う。サンプルサイズ$10$の標本を$4$つ(標本数$4$)得る。
この試行を$10^6$回繰り返す。
各試行において、すくなくとも一つの標本の$p$値が有意水準$\alpha$を超ていることを確かめる。

具体的に以下のコードを実行すればよい。
最後の行で、1から有意水準を超ていないサンプルの割合をひいて、少なくとも一つは有意水準を超ていた試行の割合を計算している。
\begin{lstlisting}
repeats =10**6
sampleN = 4
N = 10
alpha = 0.05
mu=170
sigma = 5.8
sample = norm(mu,sigma).rvs(size=(sampleN,N,repeats)) # sample.shape = (sampleN, N , repeats)
normal_sample = np.sqrt(N)*(np.average(sample,axis=1).T-mu)/sigma # normal_sample.shape = (repeats, sampleN)
p_values = norm.cdf(normal_sample)
flag = (p_values < alpha*0.5)|(p_values > 1-alpha*0.5)
1-(np.sum(np.sum(flag,axis=1) == 0))/repeats
\end{lstlisting}
結果は、およそ$0.1849$程度になり、解析解と一致することがわかる。

\subsection{最小の$p$値を採用しよう}
固定のサンプルサイズの標本をいくつか得て、それぞれの標本において$p$値を計算し、その中で最小の$p$となるものを採用するという操作を数値実験により行う。
具体的には、平均分散をそれぞれ$\mu=170,\sigma^2=5.8^2$とし、正規分布からサンプルを生成する。
またサンプルサイズを$N=10$とし、標本のペア数を$2$とする。それぞれのサンプルにおいて$p$値を計算し、$p$値の中で最小の$p$を採用する。
これを$10^3$回繰り返す。

\begin{lstlisting}
repeats =10**6
sampleN = 4
N = 10
alpha = 0.05
mu=170
sigma = 5.8
sample = norm(mu,sigma).rvs(size=(sampleN,N,repeats)) 
normal_sample = np.sqrt(N)*(np.average(sample,axis=1).T-mu)/sigma
p_values = norm.cdf(normal_sample)

min_p_values = np.min(p_values,axis=1)
first_p_values = p_values[:,0]
\end{lstlisting}

図\ref{fig:minimum_p_value_choice}には、数値計算の結果を示した。
通常$p$値は一様分布するので、累積分布は傾き$1$の直線の上にのる。実際に数値計算でも同様の結果が示されている。
一方で、最小の$p$値を選択すると、図\ref{fig:confidence_interval_model}に示したように、傾き1の直線の上に乗らない。
これは、$p$値が一様ではなく、$p$値が小さなものが通常よりも大きな頻度で生じていることを示唆している。
このことから、優位水準$\alpha$を定めると、$p<\alpha$となる$p$値は$\alpha$よりも大きな頻度で生じており、
優位水準$\alpha$で検定ができないことが示される。

\begin{figure}
  \begin{center}
    \includegraphics[width=15cm]{./image/04_/Minimum-p-values-choice-exmepriment.pdf}
    \caption{最小の$p$値を採用したときの累積分布。実線は2つの標本の中から最小の$p$を選んだときの累積分布。破線は一つのサンプルから$p$を計算した場合の累積分布}
        \label{fig:minimum_p_value_choice}
    \end{center}
\end{figure}

\subsubsection{多重検定との関係}
この処理は一つ前の節で説明した多重検定と一致する。
具体的には、すくなくとも一つの標本において$p<\alpha$となるは、最小の$p$値を採用すると同じである。
このことを数値計算により確かめておく。
\begin{lstlisting}
repeats =10**6
sampleN = 4
N = 10
alpha = 0.05
mu=170
sigma = 5.8
sample = norm(mu,sigma).rvs(size=(sampleN,N,repeats))
normal_sample = np.sqrt(N)*(np.average(sample,axis=1).T-mu)/sigma
p_values = norm.cdf(normal_sample)

np.sum(np.min(p_values,axis=1)<alpha)/repeats
\end{lstlisting}
結果は、$0.184$であり、解析解\footnote{解析解という用語は正しいのか不明}と一致する。



\subsection{サンプル追加による$p$値の変化}
標本にサンプルを追加しながら検定をおこなうと、期待している結果が得られるだろうか(元ネタ\cite{simmons2016false})。ある標本において、サンプルを追加する毎に検定を実行する。
この試行を複数回繰り返し、一度でも有意水準を下回った標本の個数を数える。
ただし、$p<\alpha$になるまで繰り返すと必ず有意となるので、最終的なサンプルのサイズはあらかじめ決定しておく。

数値実験を行う。
初期サンプルサイズを$N=10$または$N=20$、最終サンプルサイズ$N_{max}=50$、標本数$10^6$とする。サンプルを$\Delta s$個追加し、検定を実行する。
$\Delta s$は、1,5,10または20とする。それぞれの$\Delta s$に応じた検定回数は$41,9,5,3$回である\footnote{多重検定とは条件が異なるので、$\Delta s=41$だとしても、$p<\alpha$となる頻度が$1-(1-\alpha)^{41}$とならない。}。


図\ref{fig:multiple_test_additional_sample}が数値実験の結果である。
サンプルサイズを追加する毎に検定をすると、$p<\alpha$となる頻度が$\alpha$以上になる。
どの場合においても$p<\alpha$となるのは$\alpha$程度であってほしいが、これは数値実験の結果と一致しない。

\begin{figure}
  \begin{center}
    \includegraphics[width=15cm]{./image/04_/multiple_test_additional_sample.pdf}
    \caption{サンプルの追加個数と検定を実行。サンプルの追加個数を$\Delta s$としその違いによる$p<0.05$となる頻度。初期サンプルサイズが$10$(黒色)と$20$(灰色)。}
        \label{fig:multiple_test_additional_sample}
    \end{center}
\end{figure}

以下に数値実験用のコードを残しておく\footnote{わかりにくいコードになってしまった。後で書き換えたい。}。
\begin{lstlisting}
sampleN = 10**6
N = 10
maxN = 50
mu = 170
sigma = 5.8

delta = 1

def nan_index(N,maxN,delta):
    index = np.array([~(np.arange(maxN)>=i) for i in np.arange(N,maxN+delta,delta)] )
    nan_index = np.ones(index.shape)*index
    binary_index = index.astype(np.float64)
    binary_index[~index] = np.nan
    return binary_index

def stopping_rule(delta):
    nan_array = nan_index(N,maxN,delta)
    sample_num = np.count_nonzero(~np.isnan(nan_array),axis=1)


    # sample
    norm_ = norm(mu,sigma)
    sample = norm_.rvs(size=(sampleN,maxN))

    rep_sample = np.tile( sample.reshape((-1,1,maxN)), reps= (len(nan_array),1))
    restrict_sample = rep_sample*nan_array
    sample_mean = np.nanmean(restrict_sample,axis=-1)

    Z = np.sqrt(sample_num)*(sample_mean-mu)/sigma
    p = norm.cdf(Z)
    p_ = ((p < 0.025) | (p>0.975))
    true_once = (np.cumsum(p_,axis=1)>=1)
    return np.sum(true_once[:,-1])/sampleN
N = 10
result_10 = [stopping_rule(delta) for delta in [1,5,10,20]]
\end{lstlisting}

\subsection{いつかは有意になる}\label{large_sample_size_significant}
非常にわずかなモデルのちがいでも検定を使うと、モデルの標本ではないと言えてしまうことを確認する。
まず、二つの正規モデルを構築する$M_a=M(170,5.8^2),M_b=M(171;5.8^2)$とする。
母数平均が殆ど一致していることを、母数平均の規格化量$D$を計算することで確認しておく。
\begin{equation*}
 D=\frac{|\mu_a-\mu_b|}{\sigma} = 1/5.8=0.172
\end{equation*}
これらのモデルはほぼ同じような予測を行うことは理解できる。
%この$D$が小さいか大きいのかはモデルだけでは判断できない\footnote{データとの比較をするさいは重要な量の一つである}。ただ、きるだろう。

モデル$M_a$において生成したサンプルがサンプルサイズを大きくすることで、$p<\alpha$になる様子を確認しておく。
具体的には、$M_a$においてサンプルサイズ$300$の標本を10個生成する。
それぞれの標本において、サンプルサイズを$1\sim 300$までにし、それぞれのサンプルサイズにおける$p$値を計算する。

\begin{lstlisting}
def nan_index(N,maxN,delta):
    index = np.array([~(np.arange(maxN)>=i) for i in np.arange(N,maxN+delta,delta)] )
    nan_index = np.ones(index.shape)*index
    binary_index = index.astype(np.float64)
    binary_index[~index] = np.nan
    return binary_index

def calc_p():
    maxN = sample_size
    nan_array = nan_index(1,maxN,1)
    sample_num = np.arange(1,maxN+1)
    norm_ = norm(mu_a,sigma)
    sample = norm_.rvs(size=(sample_size,num_of_sample))

    rep_sample = np.tile( sample.reshape((-1,1,maxN)), reps= (len(nan_array),1))
    restrict_sample = rep_sample*nan_array

    sample_mean = np.nanmean(restrict_sample,axis=-1)
    Z = np.sqrt(sample_size)*(sample_mean-mu_b)/sigma # モデルM_bのサンプルか?
    return Z

nan_array = nan_index(1,sample_size,1)
mu_a=170
mu_b=171
sigma = 5.8
sample_size=300
num_of_sample = 10

Z = calc_p()
\end{lstlisting}

図\ref{fig:time_series_p_value_depends_sample_size}にはサンプルサイズに応じた$p$値を表示した。
サンプルを増やしていくことで、$p$値が徐々に小くなることがわかる。
殆ど同じようなモデルであってもサンプルサイズを増やしていけば、あるモデルのサンプルではないと主張できる。
このことは、データをモデルを比較するさいに非常に重要な事項である。
\begin{figure}
 \begin{center}
  \includegraphics[width=15cm]{./image/04_/p_sample_size.pdf}
  \caption{サンプル応じた$p$値の変化}
  \label{fig:time_series_p_value_depends_sample_size}
 \end{center}
\end{figure}



\subsection{いつかは有意にならない}
ある正規モデル$M_a$において、サンプルを生成し、一つづつサンプルを追加し、追加毎に$p$値を計算する。
結果を図\ref{fig:time_series_p_value}に示してある。
この図の通り、サンプルサイズを追加すると、一時的には$p$値があるしきい値(図の点線$\alpha=\frac{0.05}{2}$または$1-\frac{0.05}{2}$)を下回ることがあるが、その後サンプルを加えると、有意水準よりも大きな値となることがある。
よく言われる「検定はサンプルサイズを大きくするといつかは有意水準を下回る」に反する\footnote{こう言われているのは、理論と実験の話を混ぜているためである。正確には、モデルと実験を比較するさいに、実験でサンプルサイズを逐次追加すると、モデルと実際の乖離があきらかになりやすいため、有意になりがちである。既に前の小節で確認した。}。

\begin{figure}
  \begin{center}
    \includegraphics[width=15cm]{./image/04_/recurssive_test.pdf}
    \caption{サンプルの追加に応じた$p$値の変化}
        \label{fig:time_series_p_value}
    \end{center}
\end{figure}




\begin{defi}
  $p$値の分布が偏ることで、設定していた優位水準とは異る頻度で$p<\alpha$になっていることを「有意水準$\alpha$で検定ができないと言う」。
\end{defi}

\if 0
有意水準$\alpha$で検定ができていないのは、
\begin{itemize}
  \item 
\end{itemize}
\fi

\begin{comment}
これはむりかも
\section{類似度の誤推定}
統計モデルの間の類似度を検出力といった。
統計モデルに対して、不適切な統計量を与えたとき、検出力を歪める。
これを類似度の過誤といい、その確率を$\beta'$で表す。
直接または数値計算を行い$\beta$を計算することがおそらくできそうだが、面倒なので行わない。
\end{comment}


\subsection{まとめ}
ここまでで、統計検定量から複数の標本のうち$100\alpha\%$程度をはじきだす仕組みを学んだ。
この仕組みでは、ある標本がモデル由来かどうかを調べることはできない。
また、仕組みを適用するさいに、途中処理を加えてしまうと、複数の標本のうち$100\alpha\%$程度をはじきだすということができないことも明らかになった。
モデルと我々のデータを比較するさいにもこの点に注意が必要になる\footnote{前述した通り、多重検定には注意をはらうが、他の留意点については特に気にとめない。特に、検定まえ検定が非難されるのは、まずは検定では何もわからないという点で、次に多重検定の問題がある。}。

母集団から得たデータはモデルから生成されていると考えることができないし、母集団から得た標本の$100\alpha\%$を検出することが目的ではない。
我々が考えている系は統計的仮説検定の前提と異っており、得られるものと得たい物も完全に異っている。
このことから、データとモデルを比較するための論理を作る必要がある。




\section{データとモデルの比較}
ここで、いくつかのことを定義しておく。
\begin{defi}
    統計モデルと標本を比較して、モデルが母集団のことを予測できないとさまざまな指標をもとに判断するとき、統計モデルを却下すると宣言する。
    %ある標本から求められた統計量以上に大きな値が得られる確率を$p$値と呼ぶ。
    %絶対にダメと判断されないときは、統計モデルを採択(棄却の対義語)すると宣言しない。
    %統計モデルが棄却されるのは、統計モデルの仮定によって変化する。本書の範囲内であれば、統計モデルの母数、分布関数、独立同一の分布関数からサンプリングされたことによる。
    %最尤統計モデルにおいて、棄却されない統計モデルの母数の範囲を信頼区間といい、棄却されるモデルの母数の範囲を棄却域という。
\end{defi}
%ある閾値$\alpha$を決めて、それよりも小さな$p$値をもつ標本について、モデルから得られたものではないと判断する。ここで、$\alpha$を有意水準という。

\begin{SMbox}{検定統計量や$p$を計算するだけで解析完了}
 データとモデルの乖離具合を示す指標を計算するだけになってしまう。
 標本がモデルにより推測可能かを調べることで、より多くの予測を引き出すことができる。
 では、検出力$\beta$やサンプルサイズ、中心間の距離の規格化量$D$も記載すればいいということか?そうではない。
\end{SMbox}


ここで、母集団から無作為抽出した標本(モデルから生成された標本ではない)を正規モデルにより、予測できるかを考える。
上記の議論と同様に、標本から、統計モデルにあった統計量を計算し、統計量よりも偏った値が出現する確率($p$値)を計算する。
$p$値が小さければ、モデルにより予測できないと考え、値が0から遠いほど、もしかしたらモデルで予測できるのかもしれないと考える\footnote{$p$値だけで判断してはいけない}。
標本を元に、モデルにより予測ができないかを考えている。
%このとき、モデルを却下すると宣言する。
%$p$値が$\alpha$よりも小さいとき、流石にこのモデルでは予測できないでしょうと判定する。
%$p$値が$\alpha$よりも大きい場合でも、そのままこのモデルで予測できるとは宣言しない。他の指標やデータとモデルをグラフにより比較し、予測できそうかを考察する必要がある。

%この標本は、モデルからサンプリングしたものではない。
%標本の統計量が、モデルの上で得られやすいものかを調べる。
%$M_a$を棄却する判断をする閾値は、言い換えると、統計モデル$M_a$の棄却される母数(棄却域$R$)の出現確率を$\alpha$とした。


以上のことは、托卵行動に例えることができる\footnote{ピーと鳴く鳥に$p$値を例えたお話を作りだすことが目的で特に意味はない。将来的には以下の文章は消そう。}。
モズは、カッコウに対して卵を託す托卵を行い、カッコウは、モズの卵とは気が付かず、そのまま育てる。
ここで言い換えたいのは、カッコウは統計モデルであり、卵は標本そして、モズは科学者である。
統計モデルは、モデルからのサンプリングされた標本を巣穴に置いている。
卵の情報を要約した統計量が、モデル由来であることをモデルはその統計量の出現頻度を推測できる。
出現頻度が$p$値である。
モデルの巣に自然から無作為抽出した標本を科学者が置く。
その標本の統計量の出現頻度をモデルは推測できる。
得られた推測から、標本がモデルの卵であることを判定するのは科学者である。
この手順だけでは科学者はモデル鳥と標本卵を比較しているだけであり、標本卵を構成しているデータそのものとモデル鳥を比較していないということに注意しなければならない。


\begin{figure}
    \begin{center}
        \includegraphics[bb=0 0 1024 768,width=15cm]{./image/01_/conceptual_diagram/conceptual_diagram.003.png}
        \caption{統計量を使ったモデルとデータの比較に関する概念図}
        \label{fig:conceptual_diagram_test}
    \end{center}
\end{figure}


\begin{SMbox}{偶然の差が生じたかを確かめたい}
    「偶然の差が生じたかを確かめたい」や「こんなことが起こる確率は$5\%$くらい」という言葉を統計学の教科書で見たことがある。これらは、本書での説明とは異なる前提をもとに議論を進めており、本書と解釈の互換性はない。
    %「統計モデルの上で統計量が現れる確率が十分小さいことを確かめたい」や「統計モデル上でそのような統計量が得られる確率が$5\%$」を省略して書いたものです。

    科学では、実験で得られたデータは、同様の実験を行った場合、同様のものが得られるということが前提になっている。このことを現象に再現性があると言う。
    再現性のないデータを現状の統計学で扱うことや、現実の現象が得られる確率を議論することは困難である。

    本書の前提を元にすれば、「こんなこと(これ以上に偏った統計量値)が(モデル内で)起こる確率は$5\%$くらい」ということを省略して「こんなことが起こる確率は$5\%$くらい」と言うことはできる。また、現実において起こりやすいのかどうかについては議論できない。
\end{SMbox}


\begin{SMbox}{統計的有意性}
 \begin{quote}
  統計的有意性とは、ある影響が、偶然のみによって生ずるとは考えにくいことが統計的解析によって示されたことを意味します(生物学的有意性 を参照のこと)。有意性のレベルとは、その影響がどの程度偶然によって説明し得るかを示すものです。有意性が0.05(5%)のレベルとは、その影響が単なる偶然により生ずる可能性は 1/20しかなく、0.01(1%)のレベルとは、1/100 しかないことを意味します。ほとんどの生物学的現象は個々人すべてに必ず一様に起こるというわけではありませんから、一つの研究または実験で観察された影響は、ある程度の不確実性、あるいは不正確性を伴います。統計的解析においては、観察された影響をその確実性について評価し、それが偶然に生じ得る確率はどの程度か(有意性のレベル)を決定します。偶然によって生ずる確率が低い場合には「統計的に有意」と呼ばれ、真の影響を示すとみなされます。

  \url{https://www.rerf.or.jp/glossary/stats/}
 \end{quote}
 本書では、モデル内での話として定義した。上記方針と本書は異る。
\end{SMbox}


\subsection{$p$値を使った判断に関する注意}
$p$値を元に統計モデルとデータの不一致を考えるとき、$p$値はモデルとデータの乖離を示す指標の一つであると言うことを意識しなければならない。このことを忘れてしまい、次の間違った判断を行うことがある。
\begin{enumerate}
    \item $p$値が0に近いならば、統計モデルによりデータを予測できないと判断する
    \item $p$値が1に近いならば、統計モデルによりデータを予測できると判断する
\end{enumerate}
$p$値をもとに判断してはいけない。
%どちらも判断してはいけない。

%それぞれのデータがどのようなものなかのかを確認してみる。
%\subsubsection{$p$値が0に近い$\rightarrow$統計モデルによりデータを予測できないと判断}
%\subsubsection{$p$値が1に近い$\rightarrow$統計モデルによりデータを予測できると判断}




\begin{SMbox}{$p$値が小さければ、モデルの仮定のうち少なくとも一つが間違い}
    \ 
    \begin{quote}
        P値が小さければ、データと帰無仮説の矛盾している程度が大きいので、P値が小さければ帰無仮説は棄却するんだと統計の教科書には書かれています。実はそうではなくって、今お話ししたように小さいP値が何を意味するかというと、たくさんある統計モデルの仮定のうちどれか一つが間違っているあるいは、複数のものが間違っている。決して帰無仮説だけが間違いの対象ではなくって、先程のように、小さいP値が選択的に報告してあれば、結果としては誤った結果になります。・・・・
        %ランダム化もランダムサンプリングもなされていなければ、そもそも、データに対して確率計算をすることも意味がないことですから、そういうデータでなければ、P値を計算する意味すらなくなってしまう。
        \footnote{京都大学大学院医学研究科 聴講コース 臨床研究者のための生物統計学「仮説検定とP値の誤解」佐藤 俊哉 医学研究科教授 \url{https://www.youtube.com/watch?v=vz9cZnB1d1c} }
    \end{quote}
    $p$値が小さければ、モデルの仮定のうち少なくとも一つが誤っているというものがある。私はこの意見に賛成できない\footnote{講義の録画のため先生の意見が正しく伝えきれてないというのもあるかもしれない}。

    モデルの中で標本の統計量以上偏った値の出現確率を計算したものが$p$値である。$p$値が小さかったことは、モデル上でそのような統計量が出現しにくいということである。
    このことから、ある母数を持つモデルによりデータの平均値を予測しにくいことを示唆すのが$p$値である。

    正規分布や独立同分布ではないことを$p$値は示唆しない。
 $p$値によって、統計モデルの仮定の間違いを主張できるような値ではない。
    %ただ、モデルとデータの比較を行なった後、データが目的にあっているのかを調べなければならない。
    %モデルの仮定をデータが満たしていることを$P$値では測れない。モデルの仮定をデータが満たすことはほとんどない。

 また、$p$が十分小さいとしても、そのモデルは予測には十分使えるということもありえる。
\end{SMbox}


\begin{SMbox}{モデルの仮定を満たせるのか}
    \ 
    \begin{quote}
    最初の原則。最初に述べられている原則ですが、P値はデータと特定の統計モデルが矛盾する程度を示す指標の一つであるというふうに書かれています。ここでですね、統計モデルは何かって言うと、統計モデルは必ず一連の仮定のもとで構成されています。どんな仮定かと言いますと、統計の教科書をみますと、「データが正規分布している」とか、「平均値が等しい」などが統計モデルに必要な仮定とされているのですが、まず、一番大切なことは、データを撮るときに、先程の試験のように、薬剤のランダム割り当てが行われているとか、対象者を剪定するときにランダムサンプリングがなされているか、こういったことも統計モデルの仮定に含まれています。
    それから当然、研究計画がきちんと守られているかも統計モデルが必要とする前提の一つです。例えば、先程の臨床試験で言えば、
    結果の解釈も変わってきます。最後まで対象者が追跡できているのか。追跡不能とからつだくがあったとすると、統計モデルの後世に影響を与えます。もちろん解析方法も妥当な結果を与える解析方法でなければいけない。
    こういったことを満たしていなければ、統計モデルの仮定を満たしているとは言えない。
    %もちろん、全ての解析結果が報告されている。これは統計モデルに必要な仮定とは言えないですが・・・
    \footnote{京都大学大学院医学研究科 聴講コース 臨床研究者のための生物統計学「仮説検定とP値の誤解」佐藤 俊哉 医学研究科教授 \url{https://www.youtube.com/watch?v=vz9cZnB1d1c} }
    \end{quote}

    この意見は統計モデルに関する仮定と実験計画の二つの要素が混じっている。実験計画を統計モデルの仮定を満たすように設計するという意見だと考えられる。
    この意見に賛成しない。

     まず、統計モデルの仮定が自然において対応するものが、本書においてはない。また、「平均値が等しい」という仮定であるが、ある平均値をもつ統計モデルとデータを比べるさいに、データの平均値が異なる場合においても、統計モデルを使ってそのデータの出現頻度などを推定することが可能である。
     このことは、モデルの仮定をデータが満たさなければならないことを示唆していない。

     次に、実験計画については、科学者がみたい効果を見るために設定しているのもである。ランダムサンプリングしているのは、対象に偏りがないようにし、その集団内でのばらつきを計測するためである。
     対象の選定に偏りがあった場合、本当に推測したかったことが推測できない。例えば、成人以上を対象にした試験なのに、60歳だけしかからサンプリングできなかったなら、成人に対しての言及はできない。
     また、偏りのあるデータを偏りを前提としていない統計モデルにより解釈するのはこんなんである。
     この困難さを回避するためにも実験デザインを守った無作為抽出であった方が良い。
     
     %この考え方は本書の方針とは異なる。
     %モデルに対してではなく、科学者がみたいものが見れなくなることを意味する。
     %モデルは偏ったデータが得られたことを考慮して構築していない。
     %モデル自体に偏りを設定すればよいはずであるが、


     %医学における研究が予測精度を高めるということを目的にして統計学を使っていないので、意見が一致しない。
\end{SMbox}




\subsection{有意水準$\alpha$で検定できない例}
すでに説明したように、$p$値使った判定には様々な制限がある。
次のような限界がある。

\begin{itemize}
  \item どんな母集団に対してでも特定の統計量を使う(母集団に関する知識の欠如)
  \item 複数の標本に対して検定を実行する
  \item 有意になるまでサンプルを取得する。
  \item 複数の標本のなかで最小の$p$になった標本を採用する
\end{itemize}
%これら以外にも、様々な

このような限界を無視した$p$値の使用をp-Hackingと言う。
$70\%$程度の研究者たちが3番目に挙げた方法でデータを取得したことを認めている[\cite{john2012measuring}]。数値実験により確かめたとおり、有意水準$\alpha$により検定ができない。
ここに上げられていない方法でもp-Hackingは可能であり、文献\cite{stefan2023big}にまとめられている。
\if 0
\begin{itemize}
  \item 
  \item $p$値をまるめる。$p=0.051$を$p=0.05$として報告する。
\end{itemize}
\fi
\subsection{$p$値を使うことが常に最適な判断材料}
$p$値を使うことが常に最適な判断材料になることは非常に稀であり、$p$値だけで結論が下せるようなことは生物学においては稀である。
数理統計学で出ている結果は、全てモデルの中の話であり、現実がモデルと一致しているならば、モデルの予測通りの推論が行える。
もちろんそんなことはない。
また、モデルがデータの予測に利用できるということがわかっていれば、モデルの予測が現実の一部を捉えることができるという期待がもてる。
このモデルとデータとの対比を全く行わずに検定は運用されている。
$p$値を使った、データとモデルの比較方法はすでに様々な論文において批判されている\cite{points_of_significance}。


\subsection{いつかは有意になる}
すでに小節\ref{large_sample_size_significant}おいて説明したように、非常に僅かな違いでも、$p<\alpha$となりやすいことを示しておく。TODO



\begin{SMbox}{有意水準は$0.05$でよし}
    よくある受け答えを引用しておく\cite{greenland2016statistical}。
    \begin{quote}
        Q: Why do so many colleges and grad schools teach p = 0.05?

        A: Because that's still what the scientific community and journal editors use.

        Q: Why do so many people still use p = 0.05?

        A: Because that's what they were taught in college or grad school.
    \end{quote}
 ここでの$p=0.05$は、$\alpha=0.05$のことで、有意水準を$0.05$で教える理由について聞いている。

本書では、モデルにおける計算を具体的に行うために$\alpha=0.05$を利用し、データとモデルを比較するさいにはこの基準を使わない。
\end{SMbox}



\begin{comment}
 
\begin{SMbox}{統計的仮説検定を使った研究は禁止するべき}
 よくある受け答え。
 \begin{quote}
  A:統計的仮説検定を使った研究は禁止するべき

  B:禁止ではなくきちんと使えるように教えていくべき

  最初に戻る。
 \end{quote}

\end{SMbox}

\end{comment}


\begin{SMbox}{ある仮説を正しいと論証するより、正しくないと論証する方が簡単?}
 ある仮説を正しいと論証するよりも正しくないと論証する方が簡単という主張がある。
 
 否定するのが簡単な命題は否定しやすいが、我々が考える命題は単純に否定することすら難しい。

 次に、統計的仮説検定を使った枠組みにおいて、正しいか、正しくないかという二値的な判断を行えない。
 本書で扱う事象にたいしては、データとモデルを比較しているだけであり、モデルをデータの一部を説明するようにモデルを構築しているだけである。
 ここから直ちに仮説に対して真偽をあつかうことはできない。モデルの推定から、何のような傾向があるかなどを示すことはできる。
\end{SMbox}

\subsection{モデルの性能}
ある基準$\alpha$を前もって設定できない。なぜなら、前情報がわからない状況で構築されたモデルによって標本を予測できないと判定を下すことができない。
たとえ、設定したとして、$p \leq \alpha$であったとしても、偶然棄却域にあったのかを区別できない。
言い替えると、そのモデルで十分予測可能だったとしても、偶然できないと判定されることがある。


%必要なサンプルサイズは、モデル間での差異を明かにするためのパラメータの一つである。サンプルサイズ$n$の標本を集めたとき、その中の$100\alpha$をすてるため


\begin{SMbox}{$p<0.05$なら差があるとする}
 すでに説明したとおり、$p<0.05$だとしても、それはなんらかに差があるということではないということもある。 ある特定の統計モデルにおいて、統計検定量が出現しにくいことから、そのモデルでは標本を予測するのは難しいのではないかと考える根拠の一つにすぎない。 生物統計学の教科書にでてくる「検定を使えば差があるかないかがわかる」という記述は、生物研究において$p$値の取り扱いとしてこのコンセンサスがあるということを示しているだけである。(生物学上のコンセンサスとして)差があるということにはできるが、それがなんらか意味があるのかについてはなんら調べがついていない。また、統計学的有意ということもあるが、前提が整っていない統計的仮説検定において統計的に有意と宣言するのはおかしなことである。
 統計量として$p$値のみが記述されているようなら、その論文に対し批判的に読むべきだろう。
\end{SMbox}
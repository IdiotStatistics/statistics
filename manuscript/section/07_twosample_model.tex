

\chapter{統計モデル2}
ここでは、二つの確率変数から得られた確率変数についてその性質を議論する。

\section{正規分布二つを含んだ統計モデル}
次の3つを仮定したモデルを正規$2$モデルと呼ぶ。
\begin{quote}
    \begin{enumerate}[(1)]
    \item $x_i$および、$y_i$はそれぞれ独立同分布
    \item その分布は、正規分布
    \item 正規分布の母数はそれぞれ$\mu_1,\sigma_1^2,\mu_2,\sigma_2^2$とする。
    \end{enumerate}
\end{quote}
この正規2モデルを$M(\mu_1,\mu_2,\sigma_1^2,\sigma_2^2)$と書く。$\sigma_1,\sigma_2$を特定の値に設定したモデルを$M(\mu_1,\mu_2)$または、$M(\mu_1,\mu_2;\sigma^2_1,\sigma^2_2)$とし、$\mu_1,\mu_2$を特定の値に設定したモデルを$M(\sigma_1^2,\sigma_2^2)$または$M(\sigma_1^2,\sigma_2^2;\mu_1,\mu_2)$とする。
データから最尤推定を行なった母数を持つモデル$M_{ML}=M(\mu_{1,ML},\mu_{2,ML},\sigma_{1,ML}^2,\sigma^2_{2,ML})$を最尤正規2モデルという。

\section{分散について事前知識のある場合}
分散が先行研究において明らかに成っているとき良い予測を行えるモデル$M(\mu_1,\mu_2;\sigma^2_1,\sigma^2_2)$について考える。最初に、$\sigma_1=\sigma_2$の場合、次に$\sigma_1\neq \sigma_ 2$それぞれにおける信頼区間および検出力を考える。

\subsection{信頼区間}
統計モデル$M(\mu_1,\mu_2;\sigma^2,\sigma^2)$により、$x_1,x_2,\cdots,x_n$,$y_1,y_2,\cdots,y_m$とサンプリングされたとする。次の統計量を定義する、
$$
Z=((\bar{x}-\mu_1)-(\bar{y}-\mu_2))\frac{\sqrt{mn}}{\sigma\sqrt{m+n}}
$$
ただし、$\bar{x}=\frac{1}{n}\sum_{i=1}^{n} x_i,\bar{y}=\frac{1}{m}\sum_{i=1}^m y_i$である。
$Z$は、$Z\sim N(0,1)$となることがわかっている。

信頼区間は、$Z$の大きさ$|Z|$によって決まる。
$\alpha=0.05$とすると、
\begin{eqnarray*}
&|Z|& < z_{0.025}\\
&\rightarrow & |(\bar{x}-\mu_1)-(\bar{y}-\mu_2)|\frac{\sqrt{mn}}{\sigma\sqrt{m+n}} < z_{0.025}\\
&\rightarrow& |(\bar{x}-\mu_1)-(\bar{y}-\mu_2)| <z_{0.025}\frac{ \sigma\sqrt{m+n} }{ \sqrt{mn }}\\
\end{eqnarray*}
式を展開すると、
\begin{equation*}
    (\mu_1-\mu_2)-z_{0.025}\sigma\sqrt{\frac{1}{n_1}+\frac{1}{n_2}} \leq \bar{X}-\bar{Y} \leq (\mu_1-\mu_2)+z_{0.025}\sigma\sqrt{\frac{1}{n_1}+\frac{1}{n_2}}.
\end{equation*}
を得る。
統計モデル$M(\mu_1,\mu_2)$によるサンプリングによって得られた平均値の差の大きさは、右辺よりも小さくなることが、$95\%$くらいの確率でモデル内でおこる。
実際に、何度も統計モデルからサンプリングを行なってみると、$95\%$くらいの確率でこの等式が成り立っている。計算機で試してみる。
\begin{lstlisting}
mu1 = 10
mu2 = 10
sigma = 5
m=10
n=20

norm1 = norm(mu1,sigma)
norm2 = norm(mu2,sigma)

sample1 = norm1.rvs(size=(m,1000))
sample2 = norm2.rvs(size=(n,1000))

xbar1 = np.average(sample1,axis=0)
xbar2 = np.average(sample2,axis=0)
A = np.abs(xbar1-xbar2) < norm(0,1).interval(1-0.05)[1]*sigma*np.sqrt(m+n)/np.sqrt(m*n)
len(np.where(A==True)[0])
\end{lstlisting}
およそ950程度の標本で、不等式が成立していることが確かめられる。

計算機で計算するには、次のコードが使える。
\begin{lstlisting}
def tTest(X,Y,sigma):
    x_bar,y_bar = np.average(X),np.average(Y)
    M,N = len(X),len(Y)
    Z = (x_bar-y_bar)*np.sqrt(M*N)/(sigma*np.sqrt(M+N))
    p=norm.cdf(Z,0,1)
    return 1-p
tTest(X,Y,5.7)
\end{lstlisting}


\begin{lstlisting}
def rejectRange(X,Y,sigma):
    M,N = len(X),len(Y)
    Z = sigma*(np.sqrt(M+N))/np.sqrt(M*N)
    za,zb= norm.interval(0.95,0,1)
    return Z*za,Z*zb
rejectRange(X,Y,5.7)
\end{lstlisting}

%\subsection{信頼区間}
%平均母数$\mu_1,\mu_2(\mu_1\neq \mu_2)$のときに、

$Z$の不等式を変形していくと、次がわかる
\begin{equation*}
    (\bar{X}-\bar{Y})-z_{0.025}\sigma\sqrt{\frac{1}{n_1}+\frac{1}{n_2}} \leq \mu_1-\mu_2 \leq (\bar{X}-\bar{Y})+z_{0.025}\sigma\sqrt{\frac{1}{n_1}+\frac{1}{n_2}}.
\end{equation*}
この式の意味は、何だっけ???TODO
%これは、$\bar{X},\bar{Y}$を得たときに、%$\mu_1-\mu_2$がこの範囲にあれば、$\mu_1=\mu_2$という統計モデルは棄却できない。



\subsection{検出力}
%二つの統計モデル$M(\mu,\mu),M(\mu_1,\mu_2)$において、を元に、検出力を計算する。
$M(\mu,\mu;\sigma^2,\sigma^2)$における統計量$Z$が、もう一つの統計モデル$M(\mu_1,\mu_2)$においての出現頻度を計算する。これは1標本のモデルと同様に検出力という。

$M(\mu,\mu)$において、
\begin{equation*}
    \frac{\bar{x}-\bar{y}}{U} \sim N(0,1)
\end{equation*}
である。ここで、$U=\sigma\frac{\sqrt{m+n}}{\sqrt{mn}}$である。
また、$M(\mu_1,\mu_2)$において、
\begin{equation*}
    \frac{\bar{x}-\bar{y}}{U}\sim N(\frac{\mu_1-\mu_2}{U},1)
\end{equation*}
である。上式の信頼区間$-z_{\alpha/2}\sim z_{\alpha/2}$が下式において出現するのは、確率分布が平行移動しているので、その区間を$[A,B]$とすると、ぞれぞれ
\begin{eqnarray*}
    A = -z_{\alpha/2}+\frac{\mu_1-\mu_2}{U}\\
    B = z_{\alpha/2}+\frac{\mu_1-\mu_2}{U}
\end{eqnarray*}
である。この区間に統計量が出現する頻度は、
\begin{equation*}
    \beta = \varPhi(B)-\varPhi(A)
\end{equation*}
により計算できる。ここで、$\varPhi(x)$は、標準正規分布の累積分布関数である。

\subsection{$\sigma$が異なるモデルでの検出力}
信頼区間は、
\begin{equation*}
    -z_{\alpha/2} \leq \frac{\bar{x}-\bar{y}}{U} \leq z_{\alpha/2}
\end{equation*}
ここで、$U=\sqrt{\frac{\sigma^2_1}{n_1}+\frac{\sigma^2_2}{n_2}}$である。
また、$M(\mu,\mu)$における統計量を$Z$とすると、$Z = \frac{\bar{x}-\bar{y}}{U}\sim N(0,1)$でり、また、モデル$M(\mu_1,\mu_2)$において$\frac{(\bar{x}-\mu_1)-(\bar{y}-\mu_2)}{U} \sim N(\mu_1-\mu_2,1)$であることから、
\begin{eqnarray*}
    A &=& \frac{(a-(\mu_1-\mu_2))}{U} \\
        &=& (\mu_1-\mu_2)/U-z_{\alpha/2}
\end{eqnarray*}
同様に、
\begin{eqnarray*}
    B &=& \frac{(b-(\mu_1-\mu_2))}{U} \\
        &=& (\mu_1-\mu_2)/U+z_{\alpha/2}
\end{eqnarray*}
よって、
\begin{equation*}
    \beta = \varPhi(B)-\varPhi(A)
\end{equation*}
である。



\section{母分散の事前知識がないときの統計モデル}
正規モデル$M(\mu_1,\mu_2,\sigma^2,\sigma^2)$について考える。

\subsection{信頼区間}
$t_{m+n-2}$を自由度$m+n-2$の$t$分布上の上側$100\alpha$点とする。言い換えると、$t$分布の確率密度関数を$p^t$とすると、$p^t(T > t_{m+n-2,\alpha}) = \alpha$となる点$t_{m+n-2,\alpha}$である。

このとき、正規モデル$M(\mu_1,\mu_2,\sigma^2,\sigma^2)$からサンプリングを行った$x_1,x_2,\cdots,x_n,y_1,y_2,\cdots,y_m$について、
\begin{equation*}
    |t_0| = \frac{|\bar{x}-\bar{y}|}{S\sqrt{\frac{1}{n}+\frac{1}{m}}}\sim t_{n+m-2}
\end{equation*}
が成り立つ。ただし、
\begin{eqnarray*}
    S^2 &=& \frac{\sum_{i=1}^n(x_i-\bar{x})+\sum_{i=1}^m (y_i-\bar{y})}{n+m-2} \\
    \bar{x} &=& \frac{1}{n}\sum_{i=1}^n x_i\\
    \bar{y} &=& \frac{1}{m}\sum_{i=1}^n y_i\\
\end{eqnarray*}
である。
以上より、信頼区間は、
\begin{equation*}
    |t_0| \leq t_{m+n-2,\alpha/2}
\end{equation*}
である。

\subsection{検出力}
検出力の計算には、$\sigma$が事前研究により明らかでなければならない。
ここでは$\sigma$が特定できているモデルにおける検出力を調べる。
統計モデル$M(\mu_1,\mu_2)$において、次の統計量が非心$t'$分布に従うことがわかっている。
\begin{equation*}
     t_0\sim t'(n+m-2,\lambda)
\end{equation*}
$\lambda = \sqrt{\frac{nm}{n+m}}\Delta$、$\Delta =\frac{\mu_1-\mu_2}{\sigma}$であり、$t'(n+m-2,\lambda)$を自由度$n+m-2$、非心パラメータ$\lambda$の非心$t$分布と言う。
モデル$M(\mu,\mu)$での信頼区間は、$|t_0|<t_{n+m-2,\alpha/2}$だったので、検出力は、$P^{t'}$を非心$t'$分布の確率密度関数だとすると、
\begin{eqnarray*}
    1-\beta &=& 1-P^{t'}( |t| \leq t(n+m-2,\alpha/2)) \\
    &=& P^{t'}( t \leq -t(n+m-2,\alpha/2))+P^{t'}(t \geq t(n+m-2,\alpha/2))
\end{eqnarray*}
である。
ここで、確率密度関数に関する近似式
\begin{equation*}
    P^{t'}(t'\leq w) \approx \varPhi \left( \frac{w(1-\frac{1}{4\phi})-\lambda}{\sqrt{1+\frac{w^2}{2\phi}}} \right)
\end{equation*}
が成り立つ[\cite{2003サンプルサイズの決め方}]\footnote{私は証明を読んでいない。いつか読む。}。ただし、$\varPhi$は、標準正規分布の累積分布関数であり、$\phi=n+m-2$である。

\cite{2003サンプルサイズの決め方}より、例題を解いてみる。
$\alpha=0.05,n=10,m=8,\mu_1=5.6,\mu_2=5.0,\sigma=1.0,n+m-2 = 16,\lambda=\sqrt{n\times m/(n+m)} \times \frac{\mu_1-\mu_2}{\sigma} = 1.265 $とする。検出力は、
\begin{eqnarray*}
    1-\beta &=& P^{t'}(t \leq -t(16,0.05))+P^{t'}(t \geq t(16,0.05)) \\
    &=& P^{t'}(t \leq -2.12)+P^{t'}(t\geq 2.12) \\
    &=& P^{t'}(t \leq -2.12)+(1-P^{t'}(t\leq 2.12)) \\
    &\approx& \varPhi(\frac{-2.12(1-1/(4\times 16))-1.265}{1+(-2.12)^2/(2\times 16)}) +1-\varPhi(\frac{2.12(1-1/(4\times 16))-1.265}{\sqrt{1+2.12^2/(2\times 16)}}) \\
    &=& \varPhi(-3.139)+1-\varPhi(0.770) \\
    &=& 0.222
\end{eqnarray*}

\subsubsection*{数値計算}
数値計算でも確かめてみる。
手順は、次の通りである。
モデルA$M(5.6,5.6;\sigma^2=1.0^2)$からサンプリングを$N=10^4$回行い、統計量$t0$を計算する。その$95\%$信頼区間$[A,B]$を求める。
モデルB$M(5.6,5.0;\sigma^2=1.0^2)$からサンプリングを$N$回行い、統計量$t1$を計算する。$t1$の中で、信頼区間$[A,B]$の外側にあるものが検出力$1-\beta$である。

\begin{lstlisting}
n=10
m=8
mu1=5.6
mu2=5.0
sigma = 1.0
phi = n+m-2
N = 10000

sample1 = norm(mu1,sigma).rvs(size = (n,N))
sample2 = norm(mu1,sigma).rvs(size = (m,N))
xbar = np.average(sample1,axis=0)
ybar = np.average(sample2,axis=0)
S2 = (np.sum((sample1-xbar)**2,axis=0)+np.sum((sample2-ybar)**2,axis=0))/float(n+m-2)
S = np.sqrt(S2)
t0 = (xbar-ybar)/(S*np.sqrt(1/n+1/m))
A,B = np.quantile(t0,q=[0.025,0.95+0.025])

sample1 = norm(mu1,sigma).rvs(size = (n,N))
sample2 = norm(mu2,sigma).rvs(size = (m,N))
xbar = np.average(sample1,axis=0)
ybar = np.average(sample2,axis=0)
S2 = (np.sum((sample1-xbar)**2,axis=0)+np.sum((sample2-ybar)**2,axis=0))/float(n+m-2)
S = np.sqrt(S2)
t1 = ((xbar-ybar))/(S*np.sqrt(1/n+1/m))

print(len(np.where((t1 < A ) | (t1 >B))[0])/N)    
\end{lstlisting}
$0.22$に近い値が得られる。



%\subsection{分散未知の場合}
%[統計学 講義](http://www3.u-toyama.ac.jp/kkarato/2020/statistics/handout/Statistics[B]-2020-25-0722.pdf)






\if 0
\section{正規分布を含んだ統計モデル}
我々は普段、男女の身長をみているから男性と女性を同じ統計モデルを用いて推測することは、到底無理だと言える。だが、あえて統計モデルを一つ構築して、男女の身長について推測を試みる。
\begin{quote}
    \begin{enumerate}[(1)]
\item i.i.d
\item 正規分布
\item 正規分布母数$\mu,\sigma=5.7$
\end{enumerate}
\end{quote}
この統計モデルを$M(\mu)$とかく。データを元にして、標準偏差は男女の差がほとんどなく、$5.7$程度であった。男性に対して制度の良い推測を行えていた$M(170)$を使って、女性の身長について推測してみる。女性で、$170cm$以上はどのくらいの頻度で現れるのか、$P(x>170)=0.5$である。一方で、データでは、$0.0179$である。また、平均身長は、データでは$157.8cm$程度である。このことから、統計モデルとデータが乖離していることがわかる。

\fi
\if 0
このように、二つの標本が異なることがわかっている、データでは、同じ統計モデルで予測するとデータに一致しないことがわかる。
\fi

\if 0
以上は、サンプルサイズが十分に大きい場合の平均値と分散を知っているから一つの統計モデルを使って男と女の身長を推測することが難しいことを示唆した。


一方で、サンプルサイズが十分ではない、言い換えれば、分布の特徴が掴みにくい場合、2標本を一つの統計モデルで推測するとまずいことがわかるのだろうか?
男と女、それぞれ5人から計測を行った、上が男、下が女の標本である。
\begin{lstlisting}

[161.92083043 162.3764036  170.60849829 166.64996998 164.83863778]
[159.58195311 152.93245418 159.18632478 145.77567615 152.13809195]
\end{lstlisting}



数式を書き換えれば、$|\bar{x}-\bar{y}|$の大きさによって決まることがわかる。
無作為抽出したデータの平均値$\bar{X},\bar{Y}$の差の絶対値がこの範囲に収まらなければ、この統計モデルは、現実を捉えていないと判断され、棄却される。

無作為抽出されたデータを用いて検定にかけてみると、$p=0.004$となり、$p=0.05$よりも小さいことがわかった。
%統計モデルの仮説が妥当であることを確認する。仮定(1)は満たされるように無作為抽出を行い計測したので、問題ないと考える。仮定(2)については、サンプルサイズを大きくしたときに、正規分布による予測がよく当たることを知っているので、この仮説も大きく外れているとは言い切れない。最後に、仮説(3)が間違っていることが示唆される。
%以上によって、帰無仮説が棄却される。

\fi
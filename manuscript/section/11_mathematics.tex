
\subsection{ガンマ分布の性質}
$x\sim Ga(\alpha,\beta)$について、以下が成り立つ
\begin{eqnarray}
    k x &\sim& Ga(\alpha,\beta/k) \\
    \frac{1}{k} x &\sim& Ga(\alpha,k\beta)\\
    \chi^2_{2n}&=&Ga(n,2) 
\end{eqnarray}
以上を使うと、
\begin{eqnarray*}
    n\bar{X} &\sim& Ga\left(n,\frac{1}{\lambda}\right) \\
    \frac{n}{2}\bar{X} &\sim& Ga\left(n,\frac{2}{\lambda}\right) \\
    \frac{n}{2\lambda}\bar{X} &\sim& Ga\left(n,2 \right)=\chi^2_{n}
\end{eqnarray*}
また、ガンマ分布とベータ分布の関係より、$X_1 \sim Ga(\alpha_1,\beta),X_2\sim Ga(\alpha_2,\beta)$ならば、$\frac{X_1}{X_1+X_2}\sim Beta(\alpha_1,\alpha_2)$である。以上より、
$Z=\frac{n\bar{X}}{n\bar{X}+m\bar{Y}}\sim Beta(n,m)$である。このことから、棄却域($z_1 \leq Z \leq z_2$)を求めることができる。具体的には、
\begin{eqnarray}
    \int_0^{z_1}\frac{1}{B(n,m)}z^{n-1}(1-z)^{m-1}dz &=& \alpha/2 \\
    \int_{z_2}^{\infty}\frac{1}{B(n,m)}z^{n-1}(1-z)^{m-1}dz &=& \alpha/2.
\end{eqnarray}
この解$z_1,z_2$を計算すれば良い。


%https://stats.stackexchange.com/questions/81151/likelihood-ratio-for-two-sample-exponential-distribution
%https://stats.stackexchange.com/questions/81151/likelihood-ratio-for-two-sample-exponential-distribution



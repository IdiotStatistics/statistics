
\chapter{数理統計の補足}


\begin{comment}
 
\subsection{2標本・指数分布}
$x_1,x_2,\cdots,x_n,i.i.d. \sim \rm{Exp}(\theta_1),y_1,y_2,\cdots,y_n,i.i.d. \sim \rm{Exp}(\theta_2)$とする。
帰無仮説$H_0$を、$H_0:\theta_1=\theta_2$とし、対立仮説$H_1$を、$H_1 : \theta_1 \neq \theta_2$とする。帰無仮説のもとで、尤度関数$L_{H_0}$は、

\begin{equation}
 L_{H_0} = \theta^{-n_1-n_2}\exp\{-\theta^{-1}T\}
\end{equation}
ただし、$T=\sum_{i=0}^n x_i+\sum_{i=0}^n y_i$である。$\frac{\partial H_0}{\partial\theta}=0$となる$\theta$は、

\begin{equation}
    \frac{\partial H_0}{\partial\theta} = \{ -(n_1+n_2)+\theta^{-1}T \}\theta^{-n_1-n_2-1}\exp(-\theta^{-1}T).
\end{equation}
より、$\theta_0=\frac{T}{n_1+n_2}$である。
$\theta_0$を$L_{H_0}$に代入すると、
\begin{equation}
    L_{H_0} = \theta_0^{-n_1-n_2}\exp(-n_1-n_2).
\end{equation}

同様に、対立仮説のもとで、尤度関数$L_{H_1}$は、
\begin{equation}
    L_{H_1} = \theta_1^{-n_1}\exp\left(-\frac{n_1}{\theta_1}\bar{x}\right)\theta_2^{-n_2}\exp\left(-\frac{n_2}{\theta_2}\bar{y}\right)
\end{equation}
$\frac{\partial H_1}{\partial\theta}=0$となる$\theta_1$を計算する。
\begin{equation}
    \frac{\partial H_1}{\partial\theta_1}=\left( -n_1\theta_1^{-n_1-1} \exp\left(-\frac{n_1}{\theta_1}\bar{x}\right)+n_1\bar{x}\theta_1^{-n_1-2}\exp\left(-\frac{n_1}{\theta_1}\bar{x}\right)\right)\theta_2^{-n_2}\exp\left(-\frac{n_2}{\theta_2}\bar{y}\right).
\end{equation}
$ \frac{\partial H_1}{\partial\theta_1}=0$より、$(-n_1+n_1\bar{x}\theta_1^{-1})\theta_1^{-n_1-1}=0$より、$\hat{\theta}_1=\bar{x}$である。同様に、$\hat{\theta}_2=\bar{y}$。
以上によって、$L_{H_1}$は、
\begin{equation}
    L_{H_1}(\hat{\theta}_1,\hat{\theta}_2) = (\hat{\theta}_1)^{-n_1}\exp(-n_1)(\hat{\theta_2})^{-n_2}\exp(-n_2)
\end{equation}
である。

尤度比は、
\begin{eqnarray}
    \varLambda = \frac{L_{H_1}}{L_{H_0}} &=& \frac{ (\hat{\theta}_1)^{-n_1}(\hat{\theta}_2)^{-n_2} \exp(-n_1-n_2)}{ \theta_0^{-n_1-n_2}\exp(-n_2-n_2) }\\
    &=& \left(\frac{\theta_0}{\hat{\theta_0}}\right)^{n_1} \left(\frac{\theta_0}{\hat{\theta_1}} \right)^{n_2}
\end{eqnarray}
尤度比検定より、$-2\log \varLambda \sim\chi^2_1$である。


\end{comment}


\subsection{ガンマ分布の性質}
$x\sim Ga(\alpha,\beta)$について、以下が成り立つ
\begin{eqnarray}
    k x &\sim& Ga(\alpha,\beta/k) \\
    \frac{1}{k} x &\sim& Ga(\alpha,k\beta)\\
    \chi^2_{2n}&=&Ga(n,2) 
\end{eqnarray}
以上を使うと、
\begin{eqnarray*}
    n\bar{X} &\sim& Ga\left(n,\frac{1}{\lambda}\right) \\
    \frac{n}{2}\bar{X} &\sim& Ga\left(n,\frac{2}{\lambda}\right) \\
    \frac{n}{2\lambda}\bar{X} &\sim& Ga\left(n,2 \right)=\chi^2_{n}
\end{eqnarray*}
また、ガンマ分布とベータ分布の関係より、$X_1 \sim Ga(\alpha_1,\beta),X_2\sim Ga(\alpha_2,\beta)$ならば、$\frac{X_1}{X_1+X_2}\sim Beta(\alpha_1,\alpha_2)$である。以上より、
$Z=\frac{n\bar{X}}{n\bar{X}+m\bar{Y}}\sim Beta(n,m)$である。このことから、棄却域($z_1 \leq Z \leq z_2$)を求めることができる。具体的には、
\begin{eqnarray}
    \int_0^{z_1}\frac{1}{B(n,m)}z^{n-1}(1-z)^{m-1}dz &=& \alpha/2 \\
    \int_{z_2}^{\infty}\frac{1}{B(n,m)}z^{n-1}(1-z)^{m-1}dz &=& \alpha/2.
\end{eqnarray}
この解$z_1,z_2$を計算すれば良い。


%https://stats.stackexchange.com/questions/81151/likelihood-ratio-for-two-sample-exponential-distribution
%https://stats.stackexchange.com/questions/81151/likelihood-ratio-for-two-sample-exponential-distribution




\chapter{数理統計の補足}

\section{正規分布の検定1}

また、母数分散$\sigma$について次が成り立つ。
\begin{theo}\label{normal_sigma_chi2}
    $X_1,X_2,\cdots,X_n \sim N(\mu,\sigma^2)$について、次が成り立つ。
    \begin{equation*}
        Y = (n-1)(\frac{S_x}{\sigma})^2 \sim \chi^2_{n-1}
    \end{equation*}
    ここで、$S^2_x=\frac{1}{n-1}\sum_{i=1}^n(x_i-\bar{x})^2,\bar{x}=\frac{1}{n}\sum_{i=1}^n x_i$である。
\end{theo}


%https://ds.machijun.net/clear-exercise-of-statistics/%E7%AC%AC7%E7%AB%A0-%E6%8C%87%E6%95%B0%E6%AF%8D%E9%9B%86%E5%9B%A3ex%CE%BC%EF%BC%89%E3%81%AE%E6%AF%8D%E5%B9%B3%E5%9D%87%E3%81%AE%E4%BF%A1%E9%A0%BC%E5%8C%BA%E9%96%93%E3%81%A8%E6%A4%9C%E5%AE%9Ap128/

\section{指数分布を含むモデル}

\begin{theo}
    $x_1,x_2,\cdots,x_n \sim Ex(\lambda)$とする。
    $x_1+x_2+\cdots+x_n \sim \gamma(n,\lambda)$である
\end{theo}

$n$を自然数とし、ガンマ分布$Ga(\frac{n}{2},2)$を特に、カイ2乗分布といい、$\chi ^2_n$で表す。

\begin{theo}
$n$を自然数とする。$G\sim\Gamma(\frac{n}{2},\beta),Y_n\sim \chi^2_n$とすると、$P(G\leq w) = P(Y_n \leq 2\beta w)$
\end{theo}
\begin{proof}
$w >0$に対して、
\begin{eqnarray*}
    P(G \leq w) &=& \int_0^w \frac{\beta^\frac{n}{2}}{\Gamma(n/2)}x^{n/2-1}\exp{(-\beta x)}dx \\
    &=&\int_0^{2\beta w} \frac{\beta^{\frac{n}{2}}}{\Gamma(n/2)}\left( \frac{t}{2\beta} \right)^{n/2-1}\exp{(-\beta t/2\beta)}\frac{dt}{2\beta} (x=t/(2\beta)) \\
    &=& \int_0^{2\beta w} \frac{1}{2^{n/2}\Gamma(n/2)}t^{n/2-1}\exp{(-t/2)}dt\\
    &=&P(Y_n \leq 2\beta w)
\end{eqnarray*}
\end{proof}

以上より$n\bar{x}\sim \Gamma(n,\lambda)$である。
このとき、$\lambda$の信頼区間を求める。$\lambda$の下限は、
\begin{equation}
    P(G\leq n\bar{x}) = \frac{\alpha}{2}
\end{equation}
を満たし、$\lambda$の上限は、
\begin{equation}
P(G\leq n\bar{x}) = 1-\frac{\alpha}{2}
\end{equation}
を満たす。
下限の式を変形していく。
\begin{eqnarray*}
    \alpha/2 &=& P(G\leq n\bar{x})  \\
    &=& P(Y_{2n}\leq 2n \lambda_l \bar{x})\\
    &\rightarrow& 2n\lambda \bar{x} = \chi^2_{2n}(1-\alpha/2)\\
    &\rightarrow& \lambda = \frac{\chi^2_{2n}(1-\alpha/2)}{2n\bar{x}}
\end{eqnarray*}
上限についても同様に、
\begin{eqnarray*}
    1-\frac{\alpha}{2} &= & P(G\leq n\bar{x}) \\
    &=& P(Y_{2n}\leq 2n\lambda \bar{x})  \\
    &\rightarrow& 2n\lambda \bar{x} = \chi^2_{2n}(\alpha/2)\\
    &\rightarrow&  \lambda = \frac{\chi^2_{2n}(\alpha/2)}{2n\bar{x}}
\end{eqnarray*}
以上によって、$\frac{1}{\lambda}$の信頼区間は、
\begin{equation}
    \label{exp_model_confidence_interval}
    \frac{2n\bar{x}}{\chi^2_{2n}(\alpha/2)} \leq \frac{1}{\lambda} \leq \frac{2n\bar{x}}{\chi^2_{2n}(1-\alpha/2)}
\end{equation}



\subsection{2標本・指数分布}
$X_1,X_2,\cdots,X_n \sim i.i.d Exp(\theta_1),Y1,Y_2,\cdots,Y_n \sim i.i.d Exp(\theta_2)$とする。帰無仮説$H_0$を、$H_0:\theta_1=\theta_2$とし、対立仮説$H_1$を、$H_1 : \theta_1 \neq \theta_2$とする。帰無仮説のもとで、尤度関数$L_{H_0}$は、

\begin{equation}
    L_{H_0} = \theta^{-n_1-n_2}\exp\{-\theta^{-1}T\}
\end{equation}
ただし、$T=\sum_{i=0}^n X_i+\sum_{i=0}^n Y_i$である。$\frac{\partial H_0}{\partial\theta}=0$となる$\theta$は、

\begin{equation}
    \frac{\partial H_0}{\partial\theta} = \{ -(n_1+n_2)+\theta^{-1}T \}\theta^{-n_1-n_2-1}\exp(-\theta^{-1}T).
\end{equation}
より、$\theta_0=\frac{T}{n_1+n_2}$である。
$\theta_0$を$L_{H_0}$に代入すると、
\begin{equation}
    L_{H_0} = \theta_0^{-n_1-n_2}\exp(-n_1-n_2).
\end{equation}

同様に、対立仮説のもとで、尤度関数$L_{H_1}$は、
\begin{equation}
    L_{H_1} = \theta_1^{-n_1}\exp(-\frac{n_1}{\theta_1}\bar{x})\theta_2^{-n_2}\exp(-\frac{n_2}{\theta_2}\bar{y})
\end{equation}
$\frac{\partial H_1}{\partial\theta}=0$となる$\theta_1$を計算する。
\begin{equation}
    \frac{\partial H_1}{\partial\theta_1}=\{ -n_1\theta_1^{-n_1-1} \exp(-\frac{n_1}{\theta_1}\bar{x})+n_1\bar{x}\theta_1^{-n_1-2}\exp(-\frac{n_1}{\theta_1}\bar{x})\}\theta_2^{-n_2}\exp(-\frac{n_2}{\theta_2}\bar{y}).
\end{equation}
$ \frac{\partial H_1}{\partial\theta_1}=0$より、$(-n_1+n_1\bar{x}\theta_1^{-1})\theta_1^{-n_1-1}=0$より、$\hat{\theta}_1=\bar{x}$である。同様に、$\hat{\theta}_2=\bar{y}$。
以上によって、$L_{H_1}$は、
\begin{equation}
    L_{H_1}(\hat{\theta}_1,\hat{\theta}_2) = (\hat{\theta}_1)^{-n_1}\exp(-n_1)(\hat{\theta_2})^{-n_2}\exp(-n_2)
\end{equation}
である。

尤度比は、
\begin{eqnarray}
    \varLambda = \frac{L_{H_1}}{L_{H_0}} &=& \frac{ (\hat{\theta}_1)^{-n_1}(\hat{\theta}_2)^{-n_2} \exp(-n_1-n_2)}{ \theta_0^{-n_1-n_2}\exp(-n_2-n_2) }\\
    &=& \left(\frac{\theta_0}{\hat{\theta_0}}\right)^{n_1} \left(\frac{\theta_0}{\hat{\theta_1}} \right)^{n_2}
\end{eqnarray}
尤度比検定より、$-2\log \varLambda \sim\chi^2_1$である。

$Ga(\alpha,\beta)$について、以下が成り立つ
\begin{eqnarray}
    k X &\sim& Ga(\alpha,\beta/k) \\
    \frac{1}{k} X &\sim& Ga(\alpha,k\beta)\\
    \chi^2_{2n}&=&Ga(n,2) 
\end{eqnarray}
以上を使うと、
\begin{eqnarray*}
    n\bar{X} &\sim& Ga(n,\frac{1}{\lambda}) \\
    \frac{n}{2}\bar{X} &\sim&(\frac{n}{2},\frac{1}{\lambda}) \\
    \frac{n}{2\lambda}\bar{X} &\sim& Ga(\frac{n}{2},2)=\chi^2_{n}
\end{eqnarray*}
また、ガンマ分布とベータ分布の関係より、$X_1 \sim Ga(\alpha_1,\beta),X_2\sim Ga(\alpha_2,\beta)$ならば、$\frac{X_1}{X_1+X_2}\sim Beta(\alpha_1,\alpha_2)$である。以上より、
$Z=\frac{n\bar{X}}{n\bar{X}+m\bar{Y}}\sim Beta(n,m)$である。このことから、棄却域($z_1 \leq Z \leq z_2$)を求めることができる。具体的には、
\begin{equation}
    \int_0^{z_1}\frac{1}{B{n,m}}z^{n-1}(1-z)^{m-1}dz = \alpha/2,      \int_{z_2}^{\infty}\frac{1}{B{n,m}}z^{n-1}(1-z)^{m-1}dz = \alpha/2.
\end{equation}
この解$z_1,z_2$を計算すれば良い。






%https://stats.stackexchange.com/questions/81151/likelihood-ratio-for-two-sample-exponential-distribution
%https://stats.stackexchange.com/questions/81151/likelihood-ratio-for-two-sample-exponential-distribution


\subsection{中心極限定理}
%中心極限定理は本書の内容を超えるので、あえて紹介を避けてきた。一般的な統計学の本には詳細が書かれているので、そちらを読んだ方が良い。
%中心極限定理は、モデル内において、確率変数の収束先を推測するのに使う。
\begin{theo}[中心極限定理]
    期待値$\mu$と分散$\sigma^2$を持つ独立分布に従う確率変数列$X_1,X_2,\cdots$に対し、$S_n=\sum_{k=1}^nX_k$とおくと、
    $S_n$は、期待値$0$、分散$1$の正規分布に分布収束する。
\end{theo}


\begin{SMbox}
中心極限定理は、間違った捉え方をされていることが多い。

少なくとも一人は、次のように考えていた。
\begin{hypoth}[中心極限仮説2]
    サンプルサイズを大きくすると、正規分布に近づく。
\end{hypoth}
これも間違いである。中心極限定理の前提のもと、標本平均を集めると正規分布に近づくことはありえる。
\end{SMbox}

「元のデータのバラつき方とその代表値としての平均値」という考えかたと、「元のデータのばらつき方とは関係ない、平均値自体のバラつき方」という考えかたがあるが、その区別が十分でないことが多い\footnote{\url{https://www.frontier-u.jp/wp-content/themes/frontier-u/pdf/toukei12018.pdf}}。
\begin{comment}
統計学のユーザーの中には、次のことが成立すると考えている\footnote{
    中心極限仮説が成り立つと考えている人は多い。というよりも、中心極限仮説が成り立つことにして、科学的な議論を進めるという方針を取っているのだと思われる。
    \\ \url{http://www.ner.takushoku-u.ac.jp/masano/class_material/waseda/keiryo/R10_inference.html#3_%E4%B8%AD%E5%BF%83%E6%A5%B5%E9%99%90%E5%AE%9A%E7%90%86} .
    \\ \url{https://yukiyanai.github.io/stat2/clt.html}.
    }。
\begin{hypoth}[中心極限仮説]
    一般のデータについて、データのサンプルサイズを大きくすると、その平均$\bar{x}$は、$\bar{x}\sim N(\mu,\sigma^2/n)$。
\end{hypoth}
反例がすぐに出てくるので、このような仮説は一般には成り立たない。例えば、データがコーシー分布から生成されている場合、成り立たない。
 
\end{comment}





\if 0
\subsubsection{ サンプルサイズが大きくなるとデータの構造がわかる}
サンプルサイズが増えると、正規分布を仮定できると書いてある文献もあるが、これは大数極限定理が成立する状況に限られるので、一般の母集団の特徴は正規分布を仮定した統計モデルで捉えるられるとは言い切れない。


サンプルサイズが増えることの利点として、おおよその分布の形がわかるということが挙げられる。
分布の形がわかれば、尤度ひを使った検定が可能になり、棄却すべき統計モデルがわかるようになる。
このことは、研究をさらに進めるさいに、どのような統計モデルを構築すれば良いのかの指針となりうる。
もう一つ利点をあげるなら、比較したい分布との違いを表示することで明らかにできるという点である。
言い換えると、少数のサンプルサイズのときに検定を使うことで、統計モデルの母数を検定することに意味があったのだが、
サンプルサイズが大きくなることで、プロットすれば分布の形の違いが現れるので、検定する意味がなくなる。
\fi

\if 0
信頼区間が狭くなるというふうに書いてあることもあるが、これも正しくないはず。
Cauchy分布を使って確かめてみる。
\fi 


\if 0
一方で、データの分布形を判断しようという意見も当然出ている。
\url{https://www.shindanyaku.net/statistics/vol_05/index.php}
\fi 

\if 0
統計学では、あえてデータが正規分布しているという統計モデルを設定し、その統計モデルを使って推測を行います。

https://www.ncchd.go.jp/press/2017/adultheight.html
https://jech.bmj.com/content/71/10/1014.long

\fi 

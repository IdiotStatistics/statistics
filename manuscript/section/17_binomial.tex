\chapter{2項分布モデル}
\section{モデル}
次の仮定をおいたモデル$M^b(p; n)$を構築する。
\begin{itemize}
 \item $x_1,x_2,\cdots,x_n$を確率変数とする。
 \item その確率変数は、二項分布に従う。
 \item 二項分布の母数を$n,p$とする。
\end{itemize}
2項分布の確率質量関数は、
\begin{equation*}
 P(X=k) = (n k)p^k(1-p)^{n-k}
\end{equation*}
で与えられる。
二つの状態、$A,B$を考える。確率$p$で状態$A$が出現し、確率$1-p$で状態$B$が出現するとする。
$n$回の試行の内、$k$回が$A$である確率が$P(X=k)$により計算できる。

二項分布に従う確率変数$X$の平均と、分散は、
\begin{eqnarray}
 E[X]= np \\
 Var[X] = np(1-p)
\end{eqnarray}
である。



\chapter{2項分布適用事例}

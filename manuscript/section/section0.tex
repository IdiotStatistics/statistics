
\section{前書き}\label{introduction}
頻度論では扱うことのできない現象に対して、統計学の知識を使い予測を行うことを目標にする。我々が扱う現象は、実験計画法により計測され、その計測を行うたびに、利用できる情報が増える。また、探索するほどさまざまな情報が手に入るので、常にモデルの改訂が必要となる。
既存の研究からモデルを構築し、そのモデルが新たに手に入れたデータを予測可能かを調べる。
予測ができないと判定されたならば、新たなモデルを構築する。そのモデルはデータに対して適合具合の高いものが候補となる。
このプロセスを繰り返すことにより、予測可能な範囲を増やしていく。




注意点
\begin{enumerate}
    \item 統計検定のみで帰無仮説を含むモデルを採択または棄却する
    \item $N=30$であれば中心極限定理よりある特定の検定が使える
    \item データの出現頻度が正規分布によりよく近似できる場合のみを考える。または、標本分布が正規分布であることを前提とする。
\end{enumerate}
これらの魔術を使わずに統計学を使う方法および、推測可能なことについて考える。

\if 0
\begin{enumerate}
    \item 仮説検定で推測可能な事象(type I,II error)
    \item モデルと現象の乖離により生じる大きな間違えを含む推定
    \item モデルの母数が現象を捉えていないことにより生じる事象
\end{enumerate}
\fi

孫引き引用をした箇所は孫引きしたと書いておいた。今後読んで、引用に修正することもある。

一般的な数理統計学の教科書\cite{2012統計科学の基礎,199005数理統計,1973確率,1963数理統計学,2009統計的機械学習,2005確率と統計,2016統計学,2017現代数理統計学の基礎,2020現代数理統計学}、統計モデルについては\cite{2012データ解析のための統計モデリング入門}。
生物学者が統計学を使うときの視点は\cite{2018統計思考の世界}に詳しいが、中心極限定理の説明が十分ではないと感じる。

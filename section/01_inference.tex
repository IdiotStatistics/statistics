\chapter{科学的推論}
科学におけるモデルおよび、数理統計学におけるモデルについて説明し、その違いを明らかにする。


\section{モデル}
モデル(模型)とは、現実を表していると思わせるような、作られたものであり、次の特徴を備えています。
\begin{enumerate}
    \item モデルは本物の特徴の一部を推測可能。
    \item (1)を行うために、複数の仮定により構築される。また、それらの仮定は、現状の知識では明らかではないまたは、現実とは乖離していることがある。推測可能なことを増やすために仮定を増やすことがある。
    \item モデルは本物の要素・特徴の全体を推測することはできない。
    \item モデルは本物ではない。
\end{enumerate}
  
例えば、スケールを一定にした車のプラモデルはモデルの一つです。本物の特徴の一つである大きさを推測可能にするため、スケール(例えば、1/24など)を決めて作られている。車のプラモデルのドアや車体の幅を計測し、スケール倍すれば、本物の大きさを推測できる。本物の車がなくても、スケールを維持した車のプラモデルを持っていれば、簡単に推測が可能になる。
手のひらに収まるモデルを作ることができ、どこからでも観察することが可能である。

一方で、車のプラモデルがあったとしても、モデルの仮定によって推測可能なことが増えることがある。
例えば、本物のパーツの重量に応じて、モデルのパーツの重さを変化させる。
こうすることで、プラスティックでできたモデルの重さを計測すると、そこから本物の重さの推測が可能になる。
他にも推測したいこと、例えば、素材の質感や色や重量、または車の速度といった特徴なども、モデルに仮定を追加することで、本物の様子を捉えることが可能になっていく。


本物の車を持って来れば、本物の様子を推測することが可能であるので、車は、車自身のモデルということができる。
また、モデルの仮定を増やせば、本物の車にモデルを近づけることが可能である。
しかし、車を車自身のモデルとすると、それまであった利便性が損なわれる。
おおよその車体の長さが知りたいのに、わざわざ長い測りが必要になることや、手に持って観察することもできない。
このように、仮定を増やし細部まで推測可能にするというのは、デメリットになることがある。

本書では、モデルは本物ではないが、推測に役にたつ物として利用する。モデルと本物が極めて一致するように感じられることもあると思うが、モデルは本物ではない。

\section{統計モデル}


\subsubsection{分野によって異なるモデルの解釈}
モデルが本物であるか否かは、学問領域によって認識が異なっている。
生物物理学の視点では、モデルは現実を推測するための偽物のことだと考えていることの方が多い。
モデルが自分の知りたいことをうまく予測してくれさえいればいいという立場である。
一方で、数学では、モデルを現実と捉える傾向がある。モデルにより世界が支配されていると考えているのである。例えば、ある数学者は、流体モデルに解が安定的に存在するかがわからないから飛行機に乗りたくないと思っていると言う雰囲気がある。
現代の統計学はどちらかと言うと数学者が作った枠組みを統計ユーザーが受け入れてしまったため、ユーザーたちは、数学者のように世界を捉ようとしているように見える。


\begin{mybox}
    %\begin{quotation}
    \paragraph{頻度主義・ベイズ主義}
    頻度主義・ベイズ主義は統計学の流派を表す言葉である。頻度主義者であると言う人はあまり見たことがないが、ベイズ主義者はよく見る。ベイズ主義者は頻度主義者を非難するような主張をすることが多々ある。それぞれの立場を正確に説明した文献がないので、何を意味しているのかを私は理解できていない。

    おそらく、頻度主義では、モデルと母集団を一致させて考えており、このことを念頭にすれば頻度主義的な議論が理解できると思う。この立場にたてば、中心極限定理がデータにも適応可能になり、あらゆるデータが正規分布で推定可能にることを主張できる\footnote{本当にそう考えているのか確信が持てない}。
    %\end{quotation}
\end{mybox}


\subsection{統計モデル}
\begin{enumerate}
    \item モデルは現象の一部を表していると思わせるような、作られたもの
    \item モデルは不要な要素を削ぎ落とすために、大胆な仮定
    \item その仮定は、数理統計の知識を使う
    \item モデルは現象のことを推測できる
\end{enumerate}

モデルを作るとき、各仮定は、前提である必要はありません。実際と全く異なることを仮定しても良いですし、現象を扱うとなるとそうならざる得ないこともあります。

\begin{mybox}
    %\begin{quote}
        \paragraph{なぜ正規分布を仮定できるのか}
        %\begin{quote}
            学生のころ先生とデータについて議論していて(生物学分野です)「そもそもなぜ正規分布が仮定できるのか…」とおっしゃって二人でしばらく固まったことを思い出します。実現可能性の考え方から学ぶのが良いのかなと思います。
        %\end{quote}
        %\footnote{\url{https://twitter.com/katzkagaya/status/1209656621523058691}}
    %\end{quote}
\end{mybox}


\subsubsection{オッカムの剃刀}

仮定の追加には合理的な理由が必要だと考えられます\footnote{仮定を追加した統計モデルはベイズ統計と書かれた本で学ことができます。}。

\begin{figure}
    \begin{center}
%\includesvg{../markdown/section1/statistics_model.svg}
\end{center}
\end{figure}

\subsubsection{標本}


\subsubsection{サンプルサイズ、標本数}
\begin{defi}
母集団から無作為抽出して得た標本に含まれるデータの個数をサンプルサイズ(標本の大きさ)といい、その数を$T$や$n$で表す。同じ実験を繰り返して行ない、複数の標本を作ると、その標本の個数を標本数という。また、標本(サンプルサイズ)の大きさが大きい場合、大標本、小さいとき、小標本と言う。
\end{defi}
例えば、無作為抽出しデータを$20$個得る実験を30回繰り返した場合、サンプルサイズ$20$の標本を$30$得たことになります。言い換えれば、標本数$30$で、サンプルサイズは$20$であると言う。

サンプルサイズを標本数と言う流儀の統計学もあるようなので注意が必要である。
\footnote{ 業界によって様々な慣習があるので、業界の慣習に(師匠の言うことに)従った方がいいようにも思う\url{https://www.jil.go.jp/column/bn/colum005.html}。方言により定義が異なることがある文献がまとめられている\url{https://biolab.sakura.ne.jp/sample-size.html}。}


\chapter{科学的仮説検定}
ここまで、推測とデータが一致することを利用し、統計モデルの良さを評価し、良いモデルを選択しようとした。
科学的仮説検定では、データに対して、絶対にダメな統計モデルを調べる方法である。
この方法は、数理統計学の統計的仮説検定の枠組みを利用する。統計モデルからサンプリングされた標本に対して、その統計量が、統計モデルに対応する分布関数に従うことがわかっている。このことから、標本がある母数を持つ統計モデル由来であるかを判定する。
言い換えるなら、ある母数を持つ統計モデルから標本がサンプリングされたのかい?そうじゃないのか?どっちなんだい?に答える方法の一つである。
これを科学においては、ある母数をもつ統計モデルによって推測してもいいのかい?そうじゃないのかい?どっちなんだい?統計モデルが「ピー」と答える。



\section{自己標本の批判}
統計モデルからサンプリングした標本の統計量が従う確率密度関数が理論的に求められる。
正規モデルであれば、
\begin{equation*}
    Z = \frac{\sqrt{n}(\bar{x}-\mu)}{\sigma} \sim N(0,1)
\end{equation*}
である。このことを利用すれば、$Z$の値から、統計量が出現しにくさがモデル上で計算できる。
例えば、$Z=0$であれば、これ以上の値が出る確率は$0.5$程度なので、よくある統計量であることがわかる。
$Z=1.96$であれば、これ以上の値が得られる確率は$0.025$程度なので、なかなかのレアさであることがわかる。
これらの統計量以上に偏った値の出現しにくさを$p$値という。

$p$値が小さいなら(統計量が出現しにくいなら)、統計量の元の標本もそのモデルから得られにくい標本であるという判断をする。
つまり、モデルから標本の得られにくさの指標の一つが$p$値であるとも言え、$p$値が小さいほど、その標本はそのモデルから得られにくい。
どの標本もモデルから生成されたものであるが、ある閾値$\alpha$を決めて、それよりも小さな$p$値をもつ標本について、モデルから得られたものではないと判断する。ここで、$\alpha$を有意水準という。
モデルが生成したはずの標本であるが、閾値を決めてモデルから生成されたものではないとするのである。
モデルが自身から得られた標本を批判するのであるから、自己の標本を批判するのである。

言い換えを明示的に書いておく。
\begin{center}
    標本$\rightarrow$ 統計量 $\rightarrow$ $p$値
\end{center}

ここで、母集団から無作為抽出した標本(モデルから生成された標本ではない)を正規モデルにより、予測できるかを考える。
上記の議論と同様に、標本から、統計モデルにあった統計量を計算し、それよりも偏った値が出現する確率($p$値)を計算する。
$p$値が小さければ、モデルにより予測できないと考え、値が1に近いほど、もしかしたらモデルで予測できるのかもしれないと考える\footnote{$p$値だけで判断してはいけない}。
$p$値が$\alpha$よりも小さいとき、流石にこのモデルでは予測できないでしょうと判定する。このとき、モデルを却下すると宣言する。
$p$値が$\alpha$よりも大きい場合でも、そのままこのモデルで予測できるとは宣言しない。他の指標やデータとモデルをグラフにより比較し、予測できそうかを考察する必要がある。

%この標本は、モデルからサンプリングしたものではない。
%標本の統計量が、モデルの上で得られやすいものかを調べる。
%$M_a$を棄却する判断をする閾値は、言い換えると、統計モデル$M_a$の棄却される母数(棄却域$R$)の出現確率を$\alpha$とした。


以上のことは、托卵行動に例えることができる。
モズは、カッコウに対して卵を託す托卵を行い、カッコウは、モズの卵とは気が付かず、そのまま育てる。
ここで言い換えたいのは、カッコウは統計モデルであり、卵は標本そして、モズは科学者である。
統計モデルは、モデルからのサンプリングされた標本を巣穴に置いている。
卵の情報を要約した統計量が、モデル由来であることをモデルはその推定量の出現頻度を推測できる。
出現頻度が$p$値である。
モデルの巣に自然から無作為抽出した標本を科学者が置く。
その標本の統計量の出現頻度をモデルは推測できる。
得られた推測から、標本がモデルの卵であることを判定するのは科学者である。

\begin{figure}
    \begin{center}
        \includegraphics[width=15cm]{./image/01_/conceptual_diagram/conceptual_diagram.003.png}
        \caption{科学的仮説検定の概念図}
        \label{fig:conceptual_diagram_test}
    \end{center}
\end{figure}
    

ここで、いくつかのことを定義しておく。
\begin{defi}
    統計モデルと標本を比較して、モデルが母集団のことを予測できないと判断するとき、統計モデルを却下すると宣言する。
    特に、データから推測される却下されないモデルの母数の範囲を信頼区間といい、それ以外の区間を却下区間という。
    統計モデルにおいて、標本の統計量以上に偏った(大きいまたは小さな)値が得られる確率を$p$値と呼ぶ。

    %ある標本から求められた統計量以上に大きな値が得られる確率を$p$値と呼ぶ。
    %絶対にダメと判断されないときは、統計モデルを採択(棄却の対義語)すると宣言しない。
    %統計モデルが棄却されるのは、統計モデルの仮定によって変化する。本書の範囲内であれば、統計モデルの母数、分布関数、独立同一の分布関数からサンプリングされたことによる。
    %最尤統計モデルにおいて、棄却されない統計モデルの母数の範囲を信頼区間といい、棄却されるモデルの母数の範囲を棄却域という。
    棄却される$p$値の閾値を有意水準$\alpha$と言い、一般に$\alpha=0.05$が使われる。
    言い換えれば、$\alpha$値は、統計モデルからサンプリングされた値について、これが元の統計モデルからサンプリングなのかどうかを判定する閾値\footnote{限界値}のことである。
    
    この定義から、統計モデルから得た標本だとしても$\alpha \%$の割合で、統計モデルが棄却される\footnote{実際のデータが$\alpha\%$の割合で棄却されるということではない。モデルから生成された標本であるのに、この統計モデルから生成されていないと判定を下される}。

    統計モデルの分布関数が変化すれば、その統計モデルにおける信頼区間・棄却域の式も変わる\footnote{中心極限定理を利用し、統計量の出現範囲を近似することが多い。サンプルサイズが多ければ中心極限定理が使えるのではなく、常に解析解と乖離があることを意識するべき}。
\end{defi}





\section{再生性}
\subsubsection{$(\star)$ $N(\mu,\sigma^2)$に従う確率変数であることを判定できるか}
$N(0,1)$に従う確率$x_1,x_2,\cdots,x_n$から計算した統計量、$z=\frac{\bar{X}-0}{\sqrt{\frac{1}{n}}}$は、$N(0,1)$に従い、$z$が$95\%$の確率で見つかる範囲は$[-1.96,1.96]$である。
同様に、$y_1,y_2,\cdots,y_n \sim N(1.96,1)$であるならば、$z=\frac{\bar{Y}-1.96}{\sqrt{\frac{1}{n}}}$は、$N(0,1)$に従う。

確率変数から、特定の母数を持つ正規分布に従わないことを示すことはできるだろうか。
具体的な問題設定として、
$y_1,y_2,\cdots,y_n$を正規分布に従う確率変数とする。そのとき、$y_1,y_2,\cdots,y_n$が$N(\mu,\sigma^2)$に従わないことを判断する良い方法はどのようなものだろうか。

ここで、$y_1,y_2,\cdots,y_m \sim N(1.96,1)$にもかかわらず、$N(0,1)$に従うと推測した場合、$z=\frac{\bar{Y}-0}{\frac{1}{\sqrt{n}}} \sim N(0,1)$であると考えられる。
$z$の分子の$\mu$が$0$になっていることに注意が必要である。
実際に、$y_1,y_2,\cdots y_{100}$を$N(1.96,1)$からサンプリングした標本を$100$個作ってみると、およそ$19$を中心に分布することがわかる。
このことは、$y_1,y_2,\cdots,y_m\sim N(0,1)$であるならば、$z$は、$[-1.96,1.96]$の間で$95\%$の確率で入るので、この推測が間違いであることが推測される。
以上の考察から、$y_1,y_2\cdots,y_n\sim N(0,1)$ではないと判断する。

\begin{figure}
    \begin{center}
        \includegraphics[width=15cm]{./image/02_/normal_distribution_test.pdf}
        \caption{(a)$N(1.96)$に従う確率変数を100個サンプリングし、その標本を1000個集めたときの$z=\sqrt{100}(\bar{X}-0)$のヒストグラム (b)$N(0,1)$に従う確率変数を100個サンプリングし、その標本を1000個集めたときの$z=\sqrt{100}(\bar{X}-0)$値のヒストグラム}
    \end{center}
\end{figure}


もう一つ例を挙げる。
$y_1,y_2,\cdots,y_n \sim N(170,5.8)$とする。このとき、この標本が$N(168,5.8)$によりサンプリングされたものではなくことを示すことはできるだろうか。
$z=\sqrt{n}\frac{\bar{y}-168}{\sigma}$を計算すればよい。
図には、$N(170,5.8)$に従う確率変数を100個サンプリングし、その標本を1000個集め、ヒストグラムを描いた。
これをみると、$0.5$を中心に分布が広がることがわかる。$z=\frac{\bar{X}-168}{\sqrt{\frac{5.8}{n}}}\sim N(0,1)$であるはずである。
複数回、標本を得た場合でも、$z$が$[-1.96,1.96]$の範囲に収まっている。このことは、$N(168,5.8)$ではないと判断できないことを示唆している。


\begin{figure}
    \begin{center}
        \includegraphics[width=15cm]{./image/02_/normal_distribution_test2.pdf}
        \caption{$N(170,5.8)$に従う確率変数を100個サンプリングし、その標本を1000個集めたときの$z=\sqrt{100}(\bar{X}-168)$のヒストグラム}
    \end{center}
\end{figure}


ある正規分布に従う確率変数$x_1,x_2,\cdots,x_n$が母数の異なる正規分布で得られる確率も計算できる。具体的には、$x_1,x_2,\cdots,x_n\sim N(\mu,\sigma^2)$とし、これが$N(\mu_1,\sigma_1^2)$で得られるとすると、そのときの統計量は、$z=\frac{\bar{x}-\mu_1}{\frac{\sigma_1}{n}}$である。この$z$は、$N(0,1)$に従うと考えられるので、$\phi(|z|>Z)$となる確率を計算すれば良い。

\begin{theo}
    確率変数$x_1,x_2,\cdots,x_n \sim N(\mu,\sigma^2)$ならば、$z=\frac{\bar{X}-\mu}{\sqrt{\frac{\sigma}{n}}} \sim N(0,1)$である。
    一方で、確率変数$x_1,x_2,\cdots,x_n \sim N(\mu,\sigma^2)$とする。$N(\mu_1,\sigma_1^2)$は正規分布とする。ただし、$\mu\neq \mu_1, \sigma =\sigma_1$このとき、$z=\frac{\bar{X}-\mu_1}{\sqrt{\frac{\sigma_1}{n}}} \sim N(0,1)$ではない。
\end{theo}
$\mu$と$ \mu_1$が極めて近い値のとき、$z=\frac{\bar{X}-\mu_1}{\sqrt{\frac{\sigma_1}{n}}} $も$N(0,1)$におけるよくある値になる言い換えれば、$\phi(|z|>Z)$は十分大きい。
一方で、$\mu$と$ \mu_1$が離れた値を取ると、$\phi(|z|>Z)$は小さな値になる。

\if 0
\subsection{$(\star)$ $Exp(\lambda)$に従う確率変数であることを判定できるか}
$x_1,x_2,\cdots,x_n \sim Exp(\lambda)$であるとき、$n\bar{x}\sim Ga(n,\frac{1}{\lambda})$である。
母数不明の指数分布に従う確率変数が、$x_1,x_2,\cdots,x_n \sim Exp(\lambda)$と仮定したとき、$n\bar{x}\sim Ga(n,\frac{1}{\lambda})$でないならば、$x_1,x_2,\cdots,x_n \sim Exp(\lambda)$ではないと判断できるだろうか。シミュレーションによって確認してみよう。

この論法は、母数が不明の指数分布に従う確率変数を得たとき、その指数分布の母数が特定の値ではないことを示すためにこの論法を利用する。ここでは、母数が$\lambda=1,2,5,10,100$からサンプルサイズ4の標本を$1000$生成し、それら標本の統計量$n\bar{X}$のヒストグラムと、ガンマ関数$Ga(100,1)$の確率密度関数を比較する。

\begin{figure}
    \centering
    \includegraphics[width=15cm]{./image/02_/Exp_Gamma_simulation.pdf}
    \caption{(a)$Ga(10,1)$の確率密度関数。(b-e)指数分布からサンプルサイズ$4$の標本を$1000$回生成し、その統計量$n\bar{x}$のヒストグラム}
    \label{fig:exp_gamma_simulation}
\end{figure}

図\ref{fig:exp_gamma_simulation}(a)は、指数分布$Exp(\lambda=1)$の確率密度関数を示している。
図\ref{fig:exp_gamma_simulation}b-eは、シミュレーションの結果を示している。
図\ref{fig:exp_gamma_simulation}(b)には、指数分布$Exp(1)$に従う確率変数の統計量$n\bar{x}$が確かに、$Ga(100,1)$に従うことが確かめれる。
図\ref{fig:exp_gamma_simulation}(c-e)では、指数分布の$\lambda$が$1/2,1/5,1/10$のときの統計量のヒストグラムである。これらと、図\ref{fig:exp_gamma_simulation}(a)を比較すると、分布が異なっているので、確かに、$Ga(100,1)$には従わないことがわかる。
\fi



\if 0 
これらの事象は、統計モデルの上で観測される、数学的な事実です。
数学を扱っている以上はこの事実は決して崩れることはありえません。
一方で、我々が扱う現象ではどうなるでしょうか。現象が数学な分布関数から生成されていることは決してありえません。
誰かがサイコロを振って、人々の身長を決めているのなら話は別ですが、
人の身長が、ランダムに正規分布によって決定されることはありませんね。

$M(168)$モデルの平均値は$168cm$、データでは$171cm$程度なので、$3cm$小さい。また、
$180cm$以上の人の割合を使ってモデルとデータの乖離を調べることができました。
$180cm$の人がたまたまいなかった場合は、$M(169.1),M(168)$のどちらも推測できているとは言い難いことになります。このことから、特定の値を使って乖離を判定することは難しいと考えられます。


$\phi(z>Z(\mu))を$p値として、絶対に選択してはいけない統計モデル$M(\mu)$の母数$\mu$を調べます。具体的には、指標$p$が$0.05$より小さい統計モデルを選択しないようにします。その母数の範囲は$162.14 >\mu, \mu > 174.31$です。この母数の統計モデルは$p=0.05$の基準で使わないことを統計モデルを棄却すると言います。逆に、$p>0.05$となるモデルは、積極的に正しいとは考えません。明らかに間違いではないけども正しくもないという判断をします。


$p$値を使う方法がとられます。p値とは統計モデルとデータの乖離度合いを示す指標です。p値は$0~1$の値をとり、$0$に近いと統計モデルとデータが乖離していると判断します。
\fi 




\begin{SMbox}{偶然の差が生じたかを確かめたい}
    「偶然の差が生じたかを確かめたい」や「こんなことが起こる確率は$5\%$くらい」という言葉を統計学の教科書で見たことがある。これらは、本書での説明とは異なっており、本書と互換性はない。
    %「統計モデルの上で統計量が現れる確率が十分小さいことを確かめたい」や「統計モデル上でそのような統計量が得られる確率が$5\%$」を省略して書いたものです。
    
    科学では、実験で得られたデータは、同様の実験を行った場合、同様のものが得られるということが前提になっている。このことを現象に再現性があると言う。
    再現性のないデータを現状の統計学で扱うことや、現実の現象が得られる確率を議論することは困難である。
\end{SMbox}

    

\section{統計量をもとにしたモデル間類似度}
母数の異なる二つの統計モデル$M_a,M_b$について考察する。母数から標本を得て、それぞれの統計モデルを統計量を元にモデル間の類似度を計算する。

\subsection{検出力の定義}
$M_a$の棄却できない統計量の範囲(信頼区間$A$)に$M_b$の統計量が出現する確率を$\beta$とする。$\beta$を検出力という\footnote{検出力を検定力または統計力と呼ぶこともある。\\ \url{https://id.fnshr.info/2014/12/17/stats-done-wrong-03/}}。
%$\alpha$は統計モデルとデータを比較したとき、そのモデルを棄却する指標である。
$\beta$は、二つの異なるモデルを比較するための指標で、一方のモデルで棄却できない母数がもう一方のモデルで出現する確率である。
$M_a$に対する$M_a$の検出力は、$1-\alpha$であり、$M_a$を棄却する閾値を低く設定すると、$\beta$は大きな値になる。
二つの統計モデルの母数がよく一致するならば、$\beta$は$1-\alpha$に近い値を取り、一致していないならば、$\beta$は0に近い値を取る。
具体的に、$\alpha,\beta$を式で書くと、
\begin{eqnarray*}
    P_a(\mu \in R_a) = \alpha\\
    P_b(\mu \in A_a) = \beta
\end{eqnarray*}
ここで、$R_a,A_a$はそれぞれ統計モデル$M_a$の棄却域、信頼区間、$P_a,P_b$は、それぞれ統計モデル$M_a,M_b$における統計量に関する確率密度関数。

\subsection{正規分布モデルの検出力}
具体的に、$P_a(\mu \in R_a),P_b(\mu\in A_a)$を計算してみる。
正規モデルを構築する
\begin{quote}
    \begin{enumerate}[(1)]
\item i.i.d
\item $N(\mu,\sigma^2)$
\item 母数$\mu$。$\sigma$は既知とする(一般性を持たせるために、具体的な値は書かない。$\sigma=1$と読み替えて進めても良い)
\end{enumerate}
\end{quote}
このモデルを$M(\mu)$とし、$M_a=M(\mu_a),M_b=M(\mu_b)$とする。
$M_a$または、$M_b$からサンプリングされた確率変数$x_1,x_2,\cdots,x_n$の平均値は、それぞれ$\bar{x}_a\sim N(\mu_a,\sigma/n)$または$\bar{x}_b\sim N(\mu_b,\sigma/n)$である。
$M_a$の信頼区間$A_a$は、$|\bar{x}_a|<\mu_a+\sigma / \sqrt{n}z_{2.5\%}$である。
このとき、$P_a$を$N(\mu_a,\sigma)$の確率密度関数とすると、
\begin{equation*}
    P_a(\mu \in A_a) = \alpha
\end{equation*}
であるのは定義から明らか。
また、$P_b$を$N(\mu_b,\sigma)$の確率密度関数とすると、
\begin{equation*}
    P_b(\mu \in A_a ) = \beta
\end{equation*}
である。
$\mu_a$と、$\mu_b$が一致していれば、$P_b(\mu \in A_a ) = 1-\alpha$である。
$\mu_b$が$\mu_a$から離れていくと、$P_b(\mu \in A_a)=0$に近づいていく。


\begin{figure}
\begin{center}
    \includegraphics[width=15cm]{./image/04_/power_of_a_test_2.pdf}
    \caption{統計モデル$M_a,M_b$から計算された統計量$\bar{x}$の確率分布$P_a,P_b$。(a)灰色の範囲は$M_a$の信頼区間。(b)灰色の領域は、$1-\beta$の領域を示している。$\beta$の領域が小さいので、描画できなかった (c)$\mu_b$が$\mu_a$に近いときの$\beta$と$1-\beta$の領域。(d)灰色の範囲の面積が$\alpha$を示している。}
    \label{fig:power_of_test_alpha_beta}
\end{center}
\end{figure}


検出力と$\alpha$の領域を図示した(図\ref{fig:power_of_test_alpha_beta})。$M_a$の$95\%$信頼区間は、$|\mu|<\mu_a+z_{0.025}\frac{\sigma}{\sqrt{N}}$である。信頼区間は、図\ref{fig:power_of_test_alpha_beta}(a)において灰色で塗った$x$軸の範囲である。$\alpha$は図\ref{fig:power_of_test_alpha_beta}(c)の灰色で塗りつぶした領域の面積である。
検出力$1-\beta$は、$M_b$における$M_a$の信頼区間の外側の領域の面積なので、図\ref{fig:power_of_test_alpha_beta}(b)の濃い灰色の範囲である。

$\alpha$を0に近づけていくと、信頼区間は徐々に大きくなり、$\beta$は大きくなる。
$\alpha$を1に近づけていくと、信頼区間は徐々に狭くなり、$\beta$は小さくなる。



\begin{figure}
    \begin{center}
        \includegraphics[width=15cm]{./image/04_/power_of_a_test_3.pdf}
        \caption{統計モデル$M_a,M_b$から計算された統計量$\bar{x}$の確率分布$P_a,P_b$。(a)$\mu_a,\mu_b$のサンプルサイズ$1$の平均値がしたがう確率密度関数$N(\mu_a,\sigma^2/1),N(\mu_a,\sigma^2/1)$。(b)(a)と同じ$\mu_a,\mu_b$に対して、サンプルサイズを$30$にした場合の確率密度関数。(c)$\mu_a,\mu_b$が(a)よりも近いときの$\bar{x}$の確率密度関数。(d)(c)と同じ$\mu_a,\mu_b$に対してサンプルサイズを$30$にした場合の$\bar{x}$の確率密度関数。}
        \label{fig:power_of_test_alpha_beta_sample_size}
    \end{center}
    \end{figure}

    

$\alpha$、$M_a$の母数$\mu_a$、$M_b$の母数$\mu_b$を固定したまま、サンプルサイズを変化させ,
$\beta$の変化を表す(図\ref{fig:power_of_test_alpha_beta_sample_size})。$\bar{x}$の確率密度関数($N(\mu,\sigma^2/n)$)の分散がサンプルサイズによって変化することは明らかである。このことから、サンプルサイズが大きくなると、信頼区間は徐々に狭くなり、$1-\beta$は大きくなる。サンプルサイズが小さいときは、$1-\beta$も小さくなる。

$\mu_a$を固定し、$\mu_b$を変化させたときの検出力$1-\beta$を図\ref{fig:power_of_test_N_mu0_variable}に示した。
サンプルサイズが大きければ、$1-\beta$も大きくなることがわかる。

\begin{figure}
    \begin{center}
        \includegraphics[width=15cm]{./image/04_/power_of_test.pdf}
        \label{fig:power_of_test_N_mu0_variable}
        \caption{$\mu_a$を変数にしたときの検出力(検出力関数)。}
    \end{center}
\end{figure}

$\beta$を定義したことにより、$\beta$の数値を決定し、$M_a,M_b$の違いが$\beta$になるために必要なサンプルのサイズが推測できる。ここでは、$\mu_a,\mu_b$が固定されている状況を考える。
検出力$1-\beta$は$1$に近いほど、$M_a,M_b$が違うと主張できる。
あらかじめ決めたおいた基準の$1-\beta$を閾値を設定し、それ以上の$1-\beta$となるサンプルサイズを推測する。
サンプルサイズが小さければ、$M_a$と$M_b$の違いは曖昧であり、サンプルサイズが大きくなると、はっきりとモデルの違いがわかる。




\subsection{$\beta$の計算}
正規モデル$M_a,M_b$を使って、$\beta$を計算してみる。
$M_a$の信頼区間は、
\begin{equation*}
    -z_{0.025}\leq \frac{\sqrt{n}(\bar{x}-\mu_a)}{\sigma}\leq z_{0.025}
\end{equation*}
より、
\begin{equation*}
    A_a = \{ \mu ; \mu_a -\frac{\sigma}{\sqrt{n}}z_{0.025} \leq \mu \leq \mu_a +\frac{\sigma}{\sqrt{n}}z_{0.025} \}
\end{equation*}
である。ここで、$a=\mu_a -\frac{\sigma}{\sqrt{n}}z_{0.025},b = \mu_a +\frac{\sigma}{\sqrt{n}}z_{0.025} $とおく。棄却域は$A_a$以外の$\mu$である。$M_b$の標本平均$\bar{x}_b$は、$N(\mu,\frac{\sigma^2}{n})$に従うので、$A_a$の区間で、$N(\mu_b,\frac{\sigma^2}{n})$の面積を計算すれば良い。
ここで、$\frac{\sqrt{n}(\bar{x}_b-\mu_b)}{\sigma}\sim N(0,1)$である。
このことを利用すると、
$a,b$は、$N(\mu_b,\frac{\sigma^2}{n})$の確率変数だとすると、
\begin{eqnarray*}
    A &=& \frac{\sqrt{n}(a-\mu_b)}{\sigma} \\
    &=& \frac{\sqrt{n}(\mu_a-\frac{\sigma}{\sqrt{n} z_{\alpha/2}})}{\sigma}\\
    &=& -z_{\alpha/2}+\frac{\sqrt{n}}{\sigma}(\mu_a-\mu_b)
\end{eqnarray*}
同様に、
\begin{eqnarray*}
    B &=& \frac{\sqrt{n}(b-\mu_b)}{\sigma} \\
    &=& \frac{\sqrt{n}(\mu_a-\frac{\sigma}{\sqrt{n} z_{\alpha/2}})}{\sigma}\\
    &=& z_{\alpha/2}+\frac{\sqrt{n}}{\sigma}(\mu_a-\mu_b)
\end{eqnarray*}
である。以上より、確率密度関数$N(0,1)$において、$-z_{\alpha/2}+\frac{\sqrt{n}}{\sigma}(\mu_a-\mu_b) \leq x\leq  z_{\alpha/2}+\frac{\sqrt{n}}{\sigma}(\mu_a-\mu_b)$の間で積分すれば良い。

$d=\frac{\mu_a-\mu_b}{\sigma}$とおく。$d=0.6,n=9$とする。このときの$\beta$を計算してみる。$N(0,1)$において、$-z_{\alpha/2} -0.6\sqrt{n} \leq x \leq z_{\alpha/2} +0.6\sqrt{n}$の区間で積分する。

\begin{lstlisting}
A,B = norm.interval(0.95,0.,1)
N = 9
d = 0.6
a,b = A+d*np.sqrt(N),B+d*np.sqrt(N)
print(a,b)
norm.cdf(b,0,1)-norm.cdf(a,0,1)
\end{lstlisting}

答えは、$0.564$


\subsubsection{サンプルサイズ}
$d$と検出力を指定したときに、$M_a,M_b$の類似度を検出力以上にするためのサンプルサイズが計算できる。
$\beta=0.1,\d=0.8$とし、この$\beta$を満たすように$N$を計算した。

\begin{lstlisting}
A,B = norm.interval(0.95,0.,1)
beta = 0.1
d = 0.8
for N in range(10,200,2):
    a,b = A+d*np.sqrt(N),B+d*np.sqrt(N)
    beta_ = norm.cdf(b,0,1)-norm.cdf(a,0,1)
    if beta_ < beta:
        break
print(N)
\end{lstlisting}
計算を実行すると、$18$であることがわかる。



\subsection{最尤モデルでの$\beta$の計算}
\subsubsection{データを元にしたモデルとモデルの類似度}
統計モデルAを$M(\mu=170)$とし、統計モデルBを$M(\bar{X})$とする。ここで、$\bar{X}$は、無作為抽出によって得られた標本の平均であり、標本の大きさを$100$とする。
モデルA,Bの間の検出力が計算可能である。
$d=\frac{170-\bar{X}}{6.8}$、$n=100$であるので、$\bar{X}=168$を得たとすると、
\begin{lstlisting}
A,B = norm.interval(0.95,0,1)
N = 100
d = (170-168)/(6.8)
a,b = A+d*np.sqrt(N),B+d*np.sqrt(N)
print(a,b)
norm.cdf(b,0,1)-norm.cdf(a,0,1)
\end{lstlisting}
その検出力は、$0.163$


\section{自己否定の過誤}
統計モデルの中で、統計モデルを統計量により検査するときに、モデル自身を絶対にダメなモデルと判断してしまうことを自己否定の過誤\footnote{データとモデルを比べたときに、誤ってモデルが間違いと判定することを第一の過誤と一般の教科書は紹介している。誤ってモデルが間違いと判断するとはどのようなことなのかの定義がないので、この定義の意図がわからない。}と言う。
この過誤は2つの要因に分解でき、\footnote{$\alpha_2$は$\alpha_1$に関係するので実際には、分解できない。気持ちとしては、$\alpha_2$は、$\alpha_1$を変数に持つ関数である$\alpha=\alpha_2(\alpha_1)$。}、不適切な統計量を使用することで、棄却域と統計量の違いにより生じる$\alpha_1$、そして、検定を繰り返して生じる$\alpha_2$である($0<\alpha_2 \leq 1$)。
$\alpha_2=\alpha$となっていれば、有意水準$\alpha$の検定ができる。
$\alpha_1$は、統計モデルと、その統計量の関数になっており、言い換えれば、統計量が統計モデルの中で設計通りの振る舞いをしているかを測る指標である。
正規モデルを使い、統計量$T$を使った場合、$\alpha_1 \approx	 0 $であるが、指数モデルを使い、統計量$T$を使った場合、$\alpha_1$が指定した$\alpha$よりも多くなる。これを見ていく。
$\alpha_2$は、$\alpha\times 2$以上になる場合、軽視されることはないが、
$\alpha_1$が同程度の隔たりになる場合においては無視され、$\alpha_1$は$\alpha_2$よりも軽視されがちであることも説明する。
%統計モデルに対して不適切な統計量を使ってモデルの検証を試みると、第一の過誤が変化することがわかっている。

\subsection{どんな統計モデルでも$T$統計量で調べよう($\alpha_1$)}
統計モデルの分布の仮定が正規分布以外の場合においても、$T$統計量を使ってモデル自身を検証できるのかを調べる。次の統計モデル$M_E(\lambda)$を構築する。
\begin{enumerate}
    \item $X_1,X_2,\cdots,X_n $はi.i.d
    \item 指数分布
    \item $\lambda$
\end{enumerate}
母数$\lambda=1$とした統計モデルを$M_E(1)$とする。
$M(1)$からランダムサンプリングした確率変数$x_1,x_2,\cdots,x_n$から次の統計量を計算する。
\begin{equation*}
    T = \frac{\bar{X}-1}{\sqrt{\frac{\sigma^2}{n}}}
\end{equation*}
ここで、$T \sim t(n-1)$とする。
$T$値が$t(n-1)$の棄却域に入っている頻度を数値計算により計算する。
具体的に、平均$1$の指数分布または、平均$1$、標準偏差$1$の正規分布からサンプルを得て標本を作る。その標本を$100000$回取得する。
このとき、$T$値を計算し、$T$値いじょの値が得られる確率$p$を計算する。その$p$が$p<0.05$となる割合を計算する。以上をサンプルサイズを変化させてシミュレーションを行なった。
平均$1$、標準偏差$1$の正規分布の場合、$T$値は$t(n)$分布に従ので、$p<0.05$となる頻度も、$5\%$程度になることが期待される。
一方で、平均$1$の指数分布の場合、$T$は$t(n-1)$分布に従うとはいえない。このことから、$p<0.05$となる頻度は計算してみなければわからない。


\begin{figure}
    \begin{center}
        \includegraphics[width=15cm]{./image/04_/t_test_expon_norm.pdf}
        \caption{正規分布または指数ぶんぷから得た標本の$T$値から計算した$p$値で、$p<0.05$以下になる割合}
    \end{center}
\end{figure}

シミュレーションの結果、正規分布から標本を得た場合、$p<0.05$になる割合は、サンプルサイズに依存せず、$5\%$程度であり、期待通りである。
一方で、指数分布から標本を得た場合、$p<0.05$になる割合はサンプルサイズに応じて変化しており、また、どのサンプルサイズでも$p<0.05$となる割合は$5\%$より多い。

このことから、指数モデルの$\alpha_1$は、$\alpha_1>0.05$であることがわかり、統計量を正しく選ばなかったことで、自己否定の過誤が期待した$0.05$よりも大きくなっていることがわかる。

\if 0
\subsubsection{いつでも正規モデルでいこう}
データが非対称に分布しているのに、統計モデルに正規分布を指定した場合、推定が正しく行えないことを確認しておこう\footnote{元ネタ。
    小標本 t 検定の誤解:中心極限定理と一般化線形モデル 井口豊(生物科学研究所,長野県岡谷市)\url{https://biolab.sakura.ne.jp/small-sample-t-test-glm.html}}。
次のような統計モデルを構築する。
\begin{enumerate}
    \item $X_1,X_2,\cdots,X_n $はi.i.d
    \item 正規分布
    \item 正規分布の母数$\mu$,$\sigma^2$の値は不明
\end{enumerate}
正規分布の母数$\mu=1$とした統計モデルを$M(1)$と記述する。
この$X_1,X_2,\cdots,X_n$について次の統計量が$t(n)$分布に従うことがわかっている。
\begin{equation*}
    T = \frac{\bar{X}-1}{\sqrt{\frac{\sigma^2}{n}}} \sim t(n)
\end{equation*}
このとき、データが、既知の確率分布から得られた場合に、$p$値がサンプルサイズによってどのように変化するのかを調べる。
具体的に、平均$1$の指数分布または、平均$1$、標準偏差$1$の正規分布からサンプルを得て標本を作る。その標本を$100000$回取得する。
このとき、$T$値を計算し、$T$値いじょの値が得られる確率$p$を計算する。その$p$が$p<0.05$となる割合を計算する。以上をサンプルサイズを変化させてシミュレーションを行なった。

平均$1$、標準偏差$1$の正規分布の場合、統計モデルの仮定と一致するので、$T$値は$t(n)$分布に従う。よって、$p<0.05$となる頻度も、$5\%$程度になることが期待される。
一方で、平均$1$の指数分布の場合、統計モデルの仮定と一致しない。このことから、$T$は$t(n)$分布に従うとはいえない。このことから、$p<0.05$となる頻度はわからない。


\begin{figure}
    \begin{center}
        \includegraphics[width=15cm]{./image/04_/t_test_expon_norm.pdf}
        \caption{正規分布または指数ぶんぷから得た標本の$T$値から計算した$p$値で、$p<0.05$以下になる割合}
    \end{center}
\end{figure}

シミュレーションの結果、正規分布から標本を得た場合、$p<0.05$になる割合は、サンプルサイズに依存せず、$5\%$程度であり、理論と一致する。
一方で、指数分布から標本を得た場合、$p<0.05$になる割合はサンプルサイズに応じて変化しており、また、どのサンプルサイズでも$p<0.05$となる割合は$5\%$より多い。
このように、データが正規分布とかけ離れているにもかかわらず、正規モデルを構築し、そこから統計量を計算しても、的外れになることがあることを示唆している\footnote{$n$を大きくしたとき、中心極限定理より、$p<0.05$となる割合も$5\%$に近づくと解釈することがある。本当だろうか。具体的には、次の定理が成り立つのだろうか。
\begin{quote}
\begin{theo}
    $X_1,X_2,\cdots,X_n \sim Exp(\lambda)$とするとき、$T=\frac{\bar{X}-1/\lambda}{\sqrt{\frac{S^2}{n}}}$ここで、$S^2=\frac{1}{n-1}\sum_{i=1}^n(X_i-\bar{X})^2$である。$T\sim t(n-1)$または、$t$がなんらかの統計分布に従う。または、$E[T]<\infty,Var[T]<\infty$
\end{theo}
\end{quote}
このことが成り立つなら、中心極限定理も成立し、$n$が十分大きいときに、分布関数を近似できそうである。
}
\fi
\begin{SMbox}{サンプルサイズがxx以上あるから$t$検定}
        サンプルサイズがある値以上あるので、中心極限定理により、$t$検定が利用できるというものもある\footnote{http://id.ndl.go.jp/bib/024660739}。このロジックが読み込めなかったので、その謎を明らかにすべく我々はアマゾンの奥地へ向かった。

        %サンプルサイズが1以上であれば、$t$検定を行うことは原理的には可能である。
        データが指数分布的であるときに、$t$検定を使うときに生じる問題は上でみた通りであり、$p<0.05$となる標本の割合が多くなっているので、間違った推測をする可能性が高くなる。
        他の分布関数でもおそらく同じような現象が現れる。
        このことから、我々は「$t$検定が利用可能である」は正確ではなく、「$t$検定を使うことができるが、間違った推測である確率が高くなる」ということだと推察した。

        %サンプルサイズが大きくなれば、$\alpha_1$は小さくなる。
        業界によっては、サンプルサイズが$xx$以上であれば、過誤を無視して良いというふうに言われることもある。実際には、設計したモデルと
\end{SMbox}

\section{検定を繰り返し使おう($\alpha_2$)}
次の統計モデルによって複数の標本について推測することを考える。
\begin{enumerate}
    \item $X_1,X_2,\cdots,X_n $はi.i.d
    \item 正規分布
    \item $\mu$,$\sigma^2=10$
\end{enumerate}
ここまでは、一つの標本に対して、統計モデル$M(\mu)$により推測できるかを考えていた。
ここでは、複数の標本について、$M(\mu)$により推測できるかを仮説検定を指標にし考える。
標本が$3$個あるとする。このとき、それぞれの標本の統計量$T$が信頼区間に入っている確率は、$(1-\alpha)$である。全ての標本の統計量$T$が信頼区間に入っている確率は、その積$(1-\alpha)\times(1-\alpha)\times(1-\alpha)=(1-\alpha)^3$であり、この確率で統計モデルは棄却されない。
一方で、棄却される確率は、$1-(1-\alpha)^3$である。
\begin{table}[hbtp]
    \caption{標本数に応じた$\alpha_2$}
    \label{table:multiple_test_reject_prob}
    \centering
    \begin{tabular}{lcr}
      \hline
      標本数  & $\alpha=0.05$  &  $\alpha=0.01$ \\
      \hline \hline
       1 & $0.05$  & $0.01$ \\
       2 & $0.0975$ & $0.0199$\\
       3 & $0.142$ & $0.0297$\\
       4 & $0.185$ & $0.0394$\\
    \end{tabular}
  \end{table}
表\ref{table:multiple_test_reject_prob}は、標本数に応じた$\alpha_2$である。標本数が大きくなるについれて、$\alpha_2$が大きくなることがわかる。

$\alpha_1$がレベル$\alpha$の検定になっていない場合、$\alpha_2$はさらに有意水準$\alpha$から隔たりの多い数値になる。




\section{類似度の過誤}
統計モデルの間の類似度を検出力といった。
統計モデルに対して、不適切な統計量を与えたとき、検出力を歪める。
これを類似度の過誤といい、その確率を$\beta'$で表す。
    
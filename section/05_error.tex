
\section{モデルと標本の乖離による過誤}
データとモデルを統計量により比較したとき、その乖離について測ることのできないことをまとめる。

\subsection{モデルの確率密度関数と標本の分布の乖離による過誤}
経験がないことで、適当なモデルを構築し、非常に少数のサンプルサイズしか得られないことで、そのモデルの妥当性について検証できないまま、モデルとデータを仮説検定により比較し、判断することにより生じる間違いである。
例えば、データの分布が非対称に分布しているのに、正規分布を含んだ統計モデルを構築し、$T$統計量により検定をおこなったとする\footnote{データが非対称に分布していることから、正規分布を含んだ統計モデルでは推論できないことがわかるので、統計検定を使う意義はなくなると思う}。前の節でみたように、分布関数に対して適切な統計量を選ばなければ$\alpha_1$が設定した有意水準$\alpha$とならないので、期待していた推論が行えないことが多くなる。


\if 0
\subsection{標本の分布がわからない事による過誤}
TODO: よくわからない。
以上により仮説検定は、既知の母集団と変異を与えた母集団との違いを図る方法の一つの方法であると言える。
母集団分布が既知でないならば統計モデルをかんで構築する必要があり、そこから変異を捉えることを試みると、
モデルの仮定と標本の特徴が一致しないことにより、過誤が生じる。

\subsubsection{独立同分布ではない事による過誤}
モデルでは、確率変数は独立であるが、自然現象の多くはなんらかの相関を持っていることが多い。
人間の身長であっても、同じ社会の中にいれば、各個人がなんからの相関を持つとも考えられる。
そんなことは、おいておいて、それぞれ独立だと思って解析してみると言うのが統計学を使った方法であるので、
自然とモデルの間にはやはりギャップが存在し、モデルの想定よりもあらい推定になることがあり得る。
\fi
\if 0
\paragraph{$p<0.05$にした理由}
https://biolab.sakura.ne.jp/statistics-5-percent.html
\fi



\if 0
\subsection{サンプルサイズが小さければ$t$検定}
西内啓 著「統計学が最強の学問である(実践編)」(ダイヤモンド社)
\url{https://biolab.sakura.ne.jp/small-sample-t-test-glm.html}
\fi 

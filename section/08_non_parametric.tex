
\section{ノンパラメトリック検定}
\begin{mybox}
    \paragraph{正規分布じゃないからノンパラメトリック!}
    \begin{quote}
        サンプルサイズが大きい場合は、正規性の検定を行い、正規性でないと確認してからノンパラメトリック検定を使うと書いた文献もある\footnote{\url{https://katosei.jsbba.or.jp/view_html.php?aid=1196}.}。
        正規分布に近いとは言えないこともある場合、ノンパラメトリック検定を勧める文献もある \url{ http://j-ca.org/wp/wp-content/uploads/2016/04/5102_51kyo2_so.pdf . 臨床研究における統計学的解析 ─推定と検定の正しい使い方─}
        これらとは反対に、正規性検定をノンパラメトリック検定を利用するときの基準として使うべきでないと主張する論者もいる\footnote{https://biolab.sakura.ne.jp/normality-test-nonparametric.html}。
        ノンパラメトリック検定を勧められたときには、その検定の仮定を調べて、自分のデータと仮定が妥当であることが確認できるならば、ノンパラメトリック検定を使えば良い。
    \end{quote}
\end{mybox}

\subsubsection{Shapiro-Wilk検定}
Shapiro-Wilk検定で使う統計モデルは、次の二つの仮定により構成されている。
\begin{quote}
    \begin{enumerate}[(1)]
\item それぞれが独立に得られる
\item 正規分布
\end{enumerate}
\end{quote}
この検定を使うと、統計モデルの仮定(1)がクリアな場合、(2)が仮定できないと判断されます。つまり、$p$値が小さければ、正規分布ではないことが主張できます。一方で$p$値が大きい場合では、積極的に正規分布であるとは主張できません。

\begin{mybox}
    \begin{quotation}
t検定を使う前提条件としてShapiro-Wilk検定を使うことを推奨している文献があるようです。


統計モデルは真実ではありませんし、各仮定が前提である必要はありません。どのくらい正規分布から離れているかを判定するのは解析者に依存します。

統計モデルには、各変数が独立であることを仮定しますが、この仮定を前提にする統計的な手法の利用は推奨されてません。統計モデルの仮定のうち一つは前提としようと努力をするのですが、独立性については無視する立場があるのも確かです。我々は、統計モデルの仮定は本当に正しいことを要求しません。
\end{quotation}
\end{mybox}

% https://biolab.sakura.ne.jp/welch-anova-statwing.html

% http://ibis.t.u-tokyo.ac.jp/suzuki/lecture/2015/dataanalysis/L8.pdf
% https://biolab.sakura.ne.jp/normality-test-nonparametric.html




\subsubsection{ウィルコクソン(Wilcox)の符号順位検定}
ウィルコクソンの符号順位検定で使う統計モデルを構成する仮定は次の通りです
\begin{quote}
    \begin{enumerate}[(1)]
\item $i=0,\cdots,n$に対して、$Z_i=Y_i-X_i$とする。$Z_i$は互いに独立である。
\item $Z_i$は連続的母集団に従い、共通の中央値$\lambda$に関して対称である。
\end{enumerate}
\end{quote}  

% https://twitter.com/genkuroki/status/1444531128530976770
% https://www.jstage.jst.go.jp/article/psj/30/1/30_30.006/_pdf




\subsubsection{マン・ホイットニーのU検定}
\begin{quote}
    \begin{enumerate}[(1)]
\item 両標本が同じ分布関数から生成された(帰無仮説)
\item 等分散
\end{enumerate}
\end{quote}
ウィルコクソンの順位和検定
% https://ja.wikipedia.org/wiki/マン・ホイットニーのU検定
% https://oku.edu.mie-u.ac.jp/~okumura/stat/brunner-munzel.html



\subsubsection{対応ある t 検定}
1群の(差を作った)検定なので、等分散の仮定は必要ではない。中身が0と比べつ
% https://biolab.sakura.ne.jp/paired-t-test-two-sample-anova.html


\subsubsection{Walchの検定}
% https://biolab.sakura.ne.jp/maxima-welch-test.html
% https://oku.edu.mie-u.ac.jp/~okumura/stat/ttest.html


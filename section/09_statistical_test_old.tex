\chapter{仮説検定の実際}
実際に利用されている仮説検定について説明する。ここではあえて間違っているとされる方法も含めている。

正しい仮説検定について説明してある論文は多い。それらの中でも、数学は同じだが、手順には差異がある。


\section{仮説検定における前提}
仮説検定とは、仮説を採用するかを決定する方法である。数多くの研究で、科学的な検証がなされたことを示すために、仮説検定が利用されている。

その手順は、データの出現頻度を予測するという方法論を十分に利用しきれていない\footnote{予測を一切考えないので、モデルという考えが存在しないように見える}。
%私が、仮説検定を行うことはない。

%Fisherが有意検定を提案した後に、Neyman-Pearsonが仮説検定を提案し、その後、それらを組み合わせた仮説検定が利用されることになった。

科学的仮説検定では、データが統計的法則により予測可能であるかを示すため、モデルを構築した。
仮説検定では、モデルの代わりに仮説を使う。モデルの母数に関する仮説のみを帰無仮説と定義し、帰無仮説で指定した母数ではないを対立仮説と呼ぶ。
帰無仮説の元、標本の統計量以上に偏った値が得られる確率($p$値)を計算する。
$p$値が$0.05$よりも小さいならば、対立仮説を採択し、$p>\alpha$ならば判断を保留する。

仮説検定の枠組みでは、データが前提を満たさなければならないと考えられていることが多い\footnote{仮説検定を使う研究者にとって、モデルを使った予測であるということは意識されない。モデルの仮定ではなく、仮説検定をするための前提のことである}。
例えば、データは独立同一の分布関数から得られている\footnote{これを確かめる方法はあるのだろうか。言い換えるなら、現象が数学的分布関数により生じていることを確かめる必要がある}。また、正規分布を仮定しているのであれば、データの分散と帰無仮説の分散が等しいなどである。
そのため、仮説検定を行う前に、いくつかの仮説検定を行い、これらの前提を確かめる。正規分布の仮定は、$Shapiro$検定を使う。その後、正規分布であれば、等分散検定などを行う。
これらの前段階の検定では、$p$値が設定した$\alpha$よりも小さければ、対立仮説を採択し、$p>\alpha$であれば、帰無仮説を採用する\footnote{検定により対立仮説や帰無仮説を採択することはできないが、仮説検定においてはできるという立場をとる。}\footnote{検定ではモデルを決定できない。仮説検定においてはそれができるということにして、仮説の論証がなされている。}\footnote{採択すると言い切ったが、前段階の検定においては、採択または棄却と判断してるといっておいた方が現状にあっている。}。
さらに、最終的な仮説検定においては、$p<\alpha$ならば帰無仮説を棄却し、$p>\alpha$ならば、判断を保留する
\footnote{仮説検定の手順は分野によって少しずつ異なるので、指導教員に手順を聞くことを勧める。留年したくないなら、魔術を信仰した方が良い。やれと言われたことをやらなければ論文は通らない。}
\footnote{科学的仮説検定と仮説検定には互換性がない。前者は、モデルとデータの乖離を検討するという考えである、後者は仮説が検証できると盲信し、検証する方法である。}
\footnote{複数回の検定を行うので、多重検定の問題もあり、想定された$\alpha$水準を満たされないことが指摘されている。}
\footnote{仮説検定が廃止されたとしても、過去の研究においては仮説検定が使われており、それら過去の研究を理解する必要がある。この理由から仮説検定を理解しなければならない}。



\begin{SMbox}{p値の解釈}
$p$値は分野によって多様な解釈がなされることがある\cite{published_papers/18436201,2020医療統計解析使いこなし実践ガイド}。
\if 0
例えば、ASAの声明[\cite{ASA_JA}]を引用しているにもかかわらず、
$p$値は、証明したい仮説が真である場合、研究で行った前提条件が担保されている場合、研究で得られた結果が実際に得られる確率を示している\cite{2020医療統計解析使いこなし実践ガイド}。

などと書かれる場合がある。
ここで、「実際に得られる確率」が何を指しているのかが不明確であるが、「統計モデル上で実験で得られた統計量が得られる確率」を意図すると読み替えることはできるのだろうか。
\fi

よい解釈として以下の6つの原則が示されている\cite{published_papers/18436201}
\begin{enumerate}
    \item $p$値はデータと特定のモデルが矛盾する程度を示す指標の一つである。
    \item $p$値は調べている仮説が正しい確率や、データが偶然飲みで得られた確率を図るものではない。
    \item 科学的な結論や、ビジネス、政策における決定は$p$値がある値を越えたかどうかのみに基づくべきではない。
    \item 適正な推測のためには、全てを報告する透明性が必要である
    \item $p$値や統計的優位性は、効果の大きさや結果の重要性を意味しない。
    \item $p$値は、それだけでは統計モデルや仮説に関するエビデンスの、よい指標とはならない。
\end{enumerate}

\if 0
    $p$値や信頼区間を報告することがASAの声明では求められている。私は、それら以外の情報として、ランダムサンプリングされているということ・再現可能性・正規分布を含んだ統計モデルなどを研究者がどの程度信じているのかということも報告するべきだと考えている。統計モデルの仮定が現象から著しく外れているのならば、統計モデルを使った推論は無意味である。
    また、研究者の統計学への心情がわかれば、報告に価値があることを理解しやすくなる。
\fi 
\end{SMbox}
\begin{SMbox}{p値への誤解}
    誤解とされる解釈はも引用しておく\cite{idiot_statistics2014}\footnote{原典は\cite{GOODMAN2008135}である。孫引き引用である}。
    以下の解釈は、統計ユーザーの流派によらず間違いであるとされることが多い\footnote{これらの誤解を採用している科学者もいないとは言えない。教科書でも誤解を広めていることがある}。
    \begin{enumerate}
        \item $p=0.05$ならば、帰無仮説が真である確率は$5\%$しかない。
        \item $p\geq 0.05$のような有意でない結果は、グループ間に差がないことを意味する
        \item 統計的に有意な発見は客観的に重要である
        \item $p$値が$0.05$より大きい研究と小さい研究は矛盾する
        \item $p$値が同じ研究は帰無仮説に対して同等の証拠を提供する。
        \item $p=0.05$は、帰無仮説のもとで$5\%$しか起こり得ないデータを観察したことを意味する
        \item $p=0.05$と$p\leq 0.05$は同じことである。
        \item $p$値は不等式の形で書かれるものである(例えば、$p=0.015$のときは$p\leq 0.02$とする)。
        \item $p=0.05$は、帰無仮説を棄却したとしたら、第一種の誤りの確率が$5\%$しかないことを示す。
        \item 有意水準$p=0.05$のもとで、第一種の誤りの確率は$5\%$になる。
        \item ある方向を向いた結果やその方向の結果があり得ない差異を気に留めないのであれば、片側の$p$値を用いるべきである。
        \item 科学に関する結果や処方の方針は$p$値が有意であるかどうかに基づくべきである。
    \end{enumerate}
\end{SMbox}


\begin{SMbox}{モデルの仮定を満たせるのか}
    \ 
    \begin{quote}
    最初の原則。最初に述べられている原則ですが、P値はデータと特定の統計モデルが矛盾する程度を示す指標の一つであるというふうに書かれています。ここでですね、統計モデルは何かって言うと、統計モデルは必ず一連の仮定のもとで構成されています。どんな仮定かと言いますと、統計の教科書をみますと、「データが正規分布している」とか、「平均値が等しい」などが統計モデルに必要な仮定とされているのですが、まず、一番大切なことは、データを撮るときに、先程の試験のように、薬剤のランダム割り当てが行われているとか、対象者を剪定するときにランダムサンプリングがなされているか、こういったことも統計モデルの仮定に含まれています。
    それから当然、研究計画がきちんと守られているかも統計モデルが必要とする前提の一つです。例えば、先程の臨床試験で言えば、
    結果の解釈も変わってきます。最後まで対象者が追跡できているのか。追跡不能とからつだくがあったとすると、統計モデルの後世に影響を与えます。もちろん解析方法も妥当な結果を与える解析方法でなければいけない。
    こういったことを満たしていなければ、統計モデルの仮定を満たしているとは言えない。
    %もちろん、全ての解析結果が報告されている。これは統計モデルに必要な仮定とは言えないですが・・・
    \footnote{京都大学大学院医学研究科 聴講コース 臨床研究者のための生物統計学「仮説検定とP値の誤解」佐藤 俊哉 医学研究科教授 \url{https://www.youtube.com/watch?v=vz9cZnB1d1c} }
    \end{quote}

    この意見は統計モデルに関する仮定と実験計画の二つの要素が混じっている。実験計画を統計モデルの仮定を満たすように設計するという意見だと考えられる。
    私はこの意見に賛成しない。

     まず、統計モデルの仮定が自然において満たされていることはほとんどない。また、「平均値が等しい」という仮定であるが、ある平均値をもつ統計モデルとデータを比べるさいに、データの平均値が異なる場合においても、統計モデルを使ってそのデータの出現頻度などを推定することが可能である。
     このことは、モデルの仮定をデータが満たさなければならないことを示唆していない。

     次に、実験計画については、科学者がみたい効果を見るために設定しているのもである。ランダムサンプリングしているのは、対象に偏りがないようにし、さまざまな対象で効果があるかを検証するためである。対象の選定に偏りがあった場合、本当に推測したかったことが推測できない。例えば、成人以上を対象にした試験なのに、60歳だけしかからサンプリングできなかったなら、成人に対しての言及はできない。
     これらのことは、モデルに対してではなく、科学者がみたいものが見れなくなることを意味する。

\end{SMbox}

\begin{SMbox}{P値が小さければ、モデルの仮定のうち少なくとも一つが間違い}
    \ 
    \begin{quote}
        P値が小さければ、データと帰無仮説の矛盾している程度が大きいので、P値が小さければ帰無仮説は棄却するんだと統計の教科書には書かれています。実はそうではなくって、今お話ししたように小さいP値が何を意味するかというと、たくさんある統計モデルの仮定のうちどれか一つが間違っているあるいは、複数のものが間違っている。決して帰無仮説だけが間違いの対象ではなくって、先程のように、小さいP値が選択的に報告してあれば、結果としては誤った結果になります。・・・・
        %ランダム化もランダムサンプリングもなされていなければ、そもそも、データに対して確率計算をすることも意味がないことですから、そういうデータでなければ、P値を計算する意味すらなくなってしまう。
        \footnote{京都大学大学院医学研究科 聴講コース 臨床研究者のための生物統計学「仮説検定とP値の誤解」佐藤 俊哉 医学研究科教授 \url{https://www.youtube.com/watch?v=vz9cZnB1d1c} }
    \end{quote}
    P値が小さければ、モデルの仮定のうち少なくとも一つが誤っているというものがある。私はこの意見に賛成できない。

    モデルの中で標本の統計量以上の値の出現確率を計算したものが$P$値である。$P$値によって、仮定の間違いを主張できるような値ではない。少なくともある母数を持つモデルで予測してはいけないことを示唆するのが$P$値である。正規分布や独立同分布ではないことを示唆することはおそらくない。

    %モデルの仮定をデータが満たしていることを$P$値では測れない。モデルの仮定をデータが満たすことはほとんどない。
\end{SMbox}



\if 0
p値が小さいとき、統計モデルの少なくとも一つの仮定と実際のデータとが合わないことが考えられます。そこでまず、統計モデルの仮定を満たしているのかを確認します。統計モデルの仮定(1)を確認します。身長データを無作為抽出して計測したため、身長データはそれぞれ独立であると考えられます。次に統計モデルの仮定(2)です。これは、Q-Qプロットを使います。もし、データが正規分布ではなさそうであれば、この統計モデルは使えません。
以上のことが確かめられたら統計モデルの仮定(3)が実際のデータと異なると結論付けます。少なくとも、ある母数では実際のデータと乖離していると主張するわけです。
\fi

%\section{科学的仮説検定}

\if 0


この手順を身長の検証をするためになぞってみます。
\begin{enumerate}
    \item 普段の観察から、身長はある平均値の周りに対象に分布していることがなんとなくわかっているので、対象な分布関数の中から関数を選ぶ。また、サンプルサイズが大きいときの標本を見ると、正規分布でよく推定できることが知られている。以上のことから、正規分布で推測を試みる。
    \item 次の統計モデルを構築する。
    \begin{quote}
        \begin{itemize}
            \item 確率変数は独立同一分布に従う
            \item 正規分布関数
            \item 正規分布関数の母数$\mu,\sigma=5.7$
        \end{itemize}
    \end{quote}
    \item $\mu=170$
    \item 次のことがわかっている$x_1,\cdots,x_n$が正規分布$N(\mu,\sigma^2)$に従う確率変数であるならば、$\frac{\bar{x}-\mu}{\frac{\sigma}{\sqrt{n}}}\sim N(0,1)$である。ここで、$\bar{x}=\frac{x_1+x_2+\cdots+x_n}{n}$
    \item $\alpha=0.05$
    \item 母数からサンプルをランダム抽出し、標本とする。
    \item 無作為抽出できていかたと、標本を正規分布で推測しても良さそうかを検証する
    \item 標本の平均$\bar{X}$を計算し、統計モデルでの出現頻度を計算する。
    \item $p<\alpha$ならば、統計モデルの仮定のうち少なくとも一つがデータと乖離していると考える
    \item 仮説(1),仮説(2)はそれほど悪くない仮説であると思われるので、母数に関する仮定が間違っていると考えられる。以上から、母数は$\mu$ではないと宣言する。
\end{enumerate}

もしも、予備実験から実験状況に変化がなければ、帰無仮説を含んだ統計モデルは棄却されにくい。実験状況に変化があれば、帰無仮説を含んだ統計モデルは棄却されやすくなる。


\fi






\if 0 

このとき、$Z(\bar{x},\mu)$が$N(0,1)$においてよくある値でない場合に、棄却します。
一方で、$Z(\bar{x},\mu)$が$N(0,1)$においてよくある値だったとしても、統計モデルを採用するとは言いません。
また、$Z(\bar,\mu)$の絶対値が$0$よりも十分大きな値を取れば、$N(0,1)$において出にくいということがわかります。これは、$|\bar{X}-\mu|$つまり、統計モデルの母数$\mu$と平均値$\bar{X}$の絶対値が大きければ、$Z(\bar{X},\mu)$の出現頻度は低く、絶対値が十分$0$に近ければ、$Z(\bar{X},\mu)$の出現頻度は高いことを意味します。

ここまでは、全て数学的フィクションである統計モデルの話をしました。では、現実のデータが
では、p値が語っていることを考えるいきます。


以上のことから、この検定を使うには、少なくともQ-Qプロットが必要であることが分かります。


一般に、統計モデルが棄却されない母数の領域を信頼区間といい、棄却される母数の領域を棄却区間という。
特に、$p=0.05$を基準とし、その基準における統計モデルが棄却されない区間を$95\%$ ($100*(1-0.05)$)信頼区間という。


信頼区間と対応する言葉として、採択域(棄却域)というものがある。棄却域は、確率変数を標準正規分布へ変数変換した後での信頼区間である。これは明らかに信頼区間と一対一対応する。

https://twitter.com/genkuroki/status/1270179975195316224


信頼区間は、データによって棄却されない母数の範囲のことである。

信頼区間は、統計モデルが棄却されるパラメータかどうかを表しているので、現実の推論を全く行っていない。

身長を統計モデルにより扱うことで、$p<0.05$では帰無仮説が棄却され、推測を行う統計モデルとはいまいちだということは分かったと思う。一方で、$p>0.05$となった場合でも積極的に採択しないことはなぜだろうか。次は、その理由を探る。
\fi



\if 0
\begin{SMbox}{統計モデルを積極的に採用しない理由}
設定した$\alpha=0.05$よりも大きな$p$値をもつ統計モデル$M(162.2)$は、それなりに推測するでしょうか。この統計モデルにより$P(x=170)$は、極めて少数であり、サンプルサイズを大きくしたときと乖離していることが分かります。このことから、全ての現象に対して特定の$p$値を元に棄却するモデルを決めることの無意味さを感じることができます。
\end{SMbox}
\fi


\if 0    
\begin{SMbox}{帰無仮説のもとで偶然には起こり得ないことが起こった}
    帰無仮説のもとで偶然には起こり得ないことが起こったと言うふうに書くと、現実に起こりにくいと言うふうに印象付けられてしまう。
    非現実である統計モデルの上で、実験で得られた統計量以上の値が得られる確率は十分小さいと言い換えた方が良い。
\end{SMbox}
\fi


\if 0
\begin{SMbox}{正規分布を前提にできる場合}
TODO:意味不明\\
非常に限定された条件で、標本が正規分布していることを前提として使えます。具体的には、標本が分布関数(Cauchy分布は当てはまらない)により生成されていることが前提となる場合です。このとき、中心極限定理によって、十分なサンプルサイズがある場合には、データが正規分布に近づきます。
言い換えれば、サンプルサイズを増やしていけば任意に小さな$p$値を得ることができます。
一方で、一般の現象は特定の分布関数によってデータが生成されているとは言うことはできません。この場合、正規分布に近づくことが正当化できる科学理論はありません。     
\end{SMbox}
\fi


\if 0
\section{一般的な仮説検定の手順}
一般的な仮説検定の手順をまとめる。
%
\begin{framed}
    \begin{itemize}
        \item 帰無仮説($\mu = \mu_0$)・対立仮説($\mu\neq \mu_0$)を設定する。
        \item 仮説が正しいと考えたとき、検定統計量従う分布を考える(4)。
        \item $\alpha$を設定する。
        \item 母集団からの無作為抽出により標本を得る。
        \item 検定統計量を計算し、その出現する確率$p$値を計算する。それが$\alpha$以下であれば、$\mu=\mu_0$ではないと結論づける(帰無仮説が棄却された。)。%ただし、前段階検定においては、$p>\alpha$であれば、帰無仮説を採択する。
    \end{itemize}
\end{framed}
\fi

\section{仮説検定の手順}
仮説検定の手順を確認します。
仮説検定では、仮説の前提が正しいことを決定する必要がある。
これは、特定の分布関数にデータが従っていることを前提にし、前提が正しいならば、帰結も正しいと考えており、正しいデータを使わなければ仮説を検証できないと考えているからである\footnote{実際には、前提は検証できないのだが、仮説検定においては、できると決定されている}。%そのようなケースは実際の現象においては非常に稀であると考えられる。
ゆえに、データと想定した仮説の前提を満たしていることを注意深く検証しながら、仮説検定を利用することが求められている\footnote{科学的仮説検定では、現象を予測するためにモデルを使ったので、モデルの仮説をデータが満たさなくても良い}。
%統計モデルがデータと乖離していれば、推定値が何を意味するかが捉えられなくなると考えられており、
データが前提を満たすように、得られたデータを仮説の前提と一致するように、数学的な変換を施すこともある\footnote{標本分布の形が失われる}。
%科学的仮説検定では、モデルを調整し、再度計測することを要求
%ので、統計モデルの改訂を要求されることもある。
%ただし、$p$値を小さくすることを目標に統計モデルを改訂してはいけない。

\begin{framed}
    \begin{enumerate}
        \item 仮説検定が使える前提が何かを確認する。前提は以下のようになることが多い。
        \begin{quote}
            \begin{itemize}
                \item 確率変数は独立同一分布に従う
                \item 分布関数(正規分布など)
            \end{itemize}
        \end{quote}
        %\item 統計モデルに母数を設定する.経験的に知っている値や、過去の論文や予備実験で明らかになった値である方が良い
        %\item 統計モデルからサンプリングした確率変数の統計量が従う分布関数を探す。統計モデルの性質によって、母集団から得られた標本から得られた統計量の出現確率が計算可能になる。一般の仮説検定は本などでこの分布関数を確認する。
        \item 有意水準$\alpha$を設定する(さまざまな業界で$0.05$が設定される)。
        \item 母集団から無作為抽出を行い、標本を得る
        \item 標本が仮説の前提を満たしていることを確認する。標本分布と仮説の前提の分布関数がある程度一致していることを調べる。正規分布を前提にしているなら、正規分布の検定を行う。
        %\begin{quote}
            %標本が統計モデルにより推測できないと思われる場合、分布関数を変更し、統計モデルを再構築する。

            %または、母集団の性質が別の分布関数になったのだから、仮説検定を使うまでもなく、変化があったことが主張可能である。例えば、正規分布で推測できると(過去の実験や研究結果から)思われてたデータが、実際には指数分布的だった場合など。この場合、計測機器・無作為抽出の方法などに異常がなかったかも確認すべきである。
        %\end{quote}
        \item 標本から統計量を計算し、その値以上に大きな値をとる確率を計算する($p$値)。
        \item $p$値が$\alpha$以下であれば、帰無仮説を棄却し、対立仮説を採択する。
        \item $p$が$\alpha$以上であれば、判断を保留する(最終検定前の検定では採択する)。%\footnote{対立仮説を採択しない流派もあるが、採択していると考えるのが妥当である。}。
    \end{enumerate}
\end{framed}




\begin{SMbox}{第一の過誤・第二の過誤・統計モデルが正しい}
    Neyman-Pearson流の統計学においては、$\alpha,\beta$を次のように定義する。帰無仮説が正しいとき、誤ってそのモデルを棄却してしまう間違いを第一の過誤といい、この確率を$\alpha$とする。また、対立仮説が正しいのに、帰無モデルを採択する間違いを第二の過誤といい、この確率を$\beta$とする。

    %私は、ある統計モデルが正しいとは考えないので、「モデルが正しいとき、間違えてそのモデルを棄却する間違い」ということを考える流派に同意できなかった。
    %仮説は棄却されることはあるが、正しいことや積極的に採択されることは稀であるというのはFisherの流派が近い。
    %ただし、Fisherは、「$p$値を帰無仮説が正しいという条件のもとで、手元にあるデータ、およびさらに極端なデータが得られる確率」と定義した[\cite{1573106361610039296}]。
    %ここでの正しいという言葉の意図が重要である。
    %私は、「統計モデルの仮説が自然を記述するのにほどほど良いと考えられるとき、手元にあるデータの統計量以上の隔たりが統計モデル内で得られる確率」という意味であると捉え直し、科学的仮説検定を提案した。

    %Neyman-Pearson流派の統計学の教科書を科学の分野で見つけることができていない。
    %Fisher流派とNeyman-Pearson流派の両方を書いている教科書は存在しているが、「帰無モデルが正しい」とはどういうことかを答えたものを見つけることができていない。

    %Fisher流派とNeyman-Pearson流派の両方を同じように扱うことが問題視されることもある[\cite{published_papers/18436201}]。
\end{SMbox}



\begin{SMbox}{過誤の概念に対する懸念}
第一の過誤・第二の過誤に関する批判として\cite{norleans2004臨床試験のための統計的方法}がある。
\cite{2010毒性試験に用いる統計解析法の動向}において引用されていた部分を引用しておく。
%私も彼らの意見に同意する。
\begin{quote}
過誤の概念は非現実的である。根本的な問題は、我々が真実を知らないことである。現実の臨床試験では、我々は実験から学び、真実を知りたいと願うのであって、真実がすでに知られており、我々の観察を判断するのに利用できる、というようなものではない。現在利用できる情報だけに基づく決定は、それ以上の情報が利用できるときには間違っていたことがわかることもあり得る。それ以上の情報が得られないとき、決定を行なった元になる情報でその決定の評価を行うことは理論的に不可能である。一つの試験では、試験さそのものから得られる情報が、利用できる唯一の情報である。利用できる情報の調査と競合する利害の注意深いバランスを考慮した後でのみ、仮説の棄却や採択の判断が行われる。その後の試験の情報が利用できるようになるまでは、現在の判断が正しいか誤りかを判断する情報は存在しない。従って、一つの試験にとっては、過誤の考え方は全く意味を持たない。
\end{quote}
%何も考えず、$\alpha=0.05,\beta=0.8$としてサンプルサイズを
\end{SMbox}



%\section{何度も検定しよう}
% 塩野義製薬の薬
% https://mainichi.jp/articles/20220818/k00/00m/040/138000c
%https://www.mhlw.go.jp/stf/newpage_26901.html
%https://www.jstage.jst.go.jp/article/jjb/36/Special_Issue_2/36_S123/_pdf/-char/ja
% https://bio.nikkeibp.co.jp/atcl/news/p1/22/07/20/09735/
% https://twitter.com/penguin_pharm/status/1549746189561606144
% https://www.sankei.com/article/20220902-Y2DSR2RUWVM63G5QE6LS22VHEI/
% https://www.buzzfeed.com/jp/naokoiwanaga/s-217622?utm_source=dynamic&utm_campaign=bfsharetwitter
% https://www.kansensho.or.jp/uploads/files/guidelines/teigen_220902.pdf
% https://www3.nhk.or.jp/news/html/20220902/k10013800921000.html
% https://www3.nhk.or.jp/news/html/20220902/k10013800921000.html
% https://www3.nhk.or.jp/news/html/20220907/k10013806901000.html
% https://www.m3.com/news/open/iryoishin/1062171
% https://www.mixonline.jp/tabid55.html?artid=73353


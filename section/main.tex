%\documentclass[uplatex]{jsarticle}
\documentclass[a4paper,11pt,dvipdfmx]{jsarticle}
\usepackage{listings,jlisting}
\usepackage[dvipdfmx]{graphicx}
\usepackage{url}
\usepackage{amsthm}
\usepackage{framed}
\usepackage{svg}

\newtheorem{theo}{定理}[section]
\newtheorem{defi}{定義}[section]
\newtheorem{lemm}{補題}[section]
\newtheorem{hypoth}{仮説}[section]
%\newtheorem{Proof}{証明}[section]
\renewcommand\proofname{\bf 証明}

% https://oku.edu.mie-u.ac.jp/tex/mod/forum/discuss.php?d=2390
\usepackage{tcolorbox}
\tcbuselibrary{skins}
\newenvironment{myitemize}{%
\begin{description}}{\end{description}}
\tcolorboxenvironment{myitemize}{blanker,
before skip=6pt,after skip=12pt,top=5ex,
bottom=5ex,
borderline west={1mm}{0pt}{black!50}}
\parindent=0pt\relax % 「Some text.」と「More text.」の先頭のインデントを0にする


\newtcolorbox{mybox}[1][]{colback=red!5!white,
blanker,
before skip=6pt,after skip=12pt,top=1ex,
bottom=1ex,
borderline west={1mm}{0pt}{black!50},
%colback=white!10!white,
#1}

\usepackage{enumerate}
\lstset{
language = python,   
breaklines = true,
numbers = left,
frame = tbrl,
tabsize = 4,
captionpos = t
}


%% https://www.kerislab.jp/posts/2019-01-14-vscode-latex/
%% 
% 数式
\usepackage{amsmath,amsfonts}
\usepackage{bm}
% 画像
\usepackage[dvipdfmx]{graphicx}

\begin{document}

\title{科学統計教程}
\author{Idiot}
%\date{2020/6/29}
\maketitle
%% TODO QQplot,検出力,確率密度関数,分散分析,t分布

\section{前書き}\label{introduction}



本書の特徴

注意点





\begin{enumerate}
    \item 統計検定のみで帰無仮説を含むモデルを採択または棄却する
    \item $N=30$であれば中心極限定理よりある特定の検定が使える
    \item データの出現頻度が正規分布によりよく近似できる場合のみを考える。または、標本分布が正規分布であることを前提とする。
\end{enumerate}
これらの魔術を使わずに統計学を使う方法および、推測可能なことについて考える。


\begin{enumerate}
    \item 仮説検定で推測可能な事象(type I,II error)
    \item モデルと現象の乖離により生じる大きな間違えを含む推定
    \item モデルの母数が現象を捉えていないことにより生じる事象
\end{enumerate}

孫引き引用をした箇所は孫引きしたと書いておいた。今後読んで、引用に修正することもある。

一般的な数理統計学の教科書\cite{2012統計科学の基礎,199005数理統計,1973確率,1963数理統計学,2009統計的機械学習,2005確率と統計,2016統計学,2017現代数理統計学の基礎,2020現代数理統計学}、統計モデルについては\cite{2012データ解析のための統計モデリング入門}。
生物学者が統計学を使うときの視点は\cite{2018統計思考の世界}に詳しいが、中心極限定理の説明が十分ではないと感じる。


\bibliography{ref} %hoge.bibから拡張子を外した名前
\bibliographystyle{junsrt}

\end{document}
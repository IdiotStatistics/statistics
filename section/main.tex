%\documentclass[uplatex]{jsarticle}
\documentclass[a4paper,11pt,dvipdfmx]{jsbook}
\usepackage{listings,jlisting}
\usepackage[dvipdfmx]{graphicx}
\usepackage{url}
\usepackage{amsthm}
\usepackage{framed}
\usepackage{svg}

\newtheorem{theo}{定理}[section]
\newtheorem{defi}{定義}[section]
\newtheorem{lemm}{補題}[section]
\newtheorem{hypoth}{仮説}[section]
%\newtheorem{Proof}{証明}[section]
\renewcommand\proofname{\bf 証明}

%\usepackage[showframe]{geometry}
\usepackage{tikz} 
%\usepackage{showframe}
\usepackage[breakable]{tcolorbox}
%\usepackage{needspace}

% https://oku.edu.mie-u.ac.jp/tex/mod/forum/discuss.php?d=2390
%\usepackage{tcolorbox}
\tcbuselibrary{skins}
\newenvironment{myitemize}{%
\begin{description}}{\end{description}}
\tcolorboxenvironment{myitemize}{blanker,
before skip=6pt,after skip=12pt,top=5ex,
bottom=5ex,
borderline west={1mm}{0pt}{black!50}}
\parindent=0pt\relax % 「Some text.」と「More text.」の先頭のインデントを0にする


\newtcolorbox{mybox}[1][]{colback=red!5!white,
breakable,
%blanker,
before skip=6pt,after skip=12pt,top=1ex,
bottom=1ex,
borderline west={1mm}{0pt}{black!50},
%left=50mm,
%colback=white!10!white,
#1}

\usepackage {blindtext} 

\newtcolorbox{brokenbox}[1][]{width=0.8\linewidth, colback=blue!20!white, colframe=blue!70!white, boxrule=0.3mm, toptitle=1.5mm, bottomtitle=1.5mm,breakable,enlarge left by=0.1\linewidth,#1}


\definecolor{myblue}{RGB}{139,66,58}
\tcbset{
shortmessagestyle/.style={
    breakable,
    %enhanced,
    %outer arc=0pt,
    %arc=0pt,
    colframe=myblue,
    colback=white,%myblue!20,
    %colback=red!5!white,
    before skip=6pt,after skip=12pt,top=2ex,
    bottom=2ex,left=2ex,
    boxrule=-1pt,
    borderline west={1mm}{0pt}{myblue},
    %attach boxed title to top left,
    %boxed title style={
        %colback=myblue,
        %outer arc=0pt,
        %arc=0pt,
    %},
    title=Example~\thetcbcounter,
    fonttitle=\sffamily,
    colupper=black,
    %colframe=black,fonttitle=\bfseries,
    colbacktitle=white,
    coltitle=black,
}}

\newtcolorbox[auto counter,number within=section]{SMbox}[2][]{
  shortmessagestyle,
  title=■#2,%~\thetcbcounter,
  #1
  }


  \definecolor{bubblegreen}{RGB}{103,184,104}
  \definecolor{bubblegray}{RGB}{241,240,240}
  \definecolor{bubbleblue}{RGB}{0,172,238}  
  \tcbset{
    commonoptions/.style={
      enhanced,
      enhanced jigsaw,    
      fonttitle=\bfseries,
      boxed title style={opacityback=0, opacityframe=0},
      sharp corners,
      arc=4mm,
      %opacityframe=0,
      after skip=0mm,
      colframe=bubbleblue,
      colback=white,
      %left=10mm,
      %left=20mm,
      %leftrule=10mm,
      before skip=6pt,after skip=12pt,top=2ex,
      },
    }
    \newtcolorbox{rightbubbles}[4][]{
        commonoptions,    
        attach boxed title to top right={yshift=-7mm, xshift=-2mm},
        top=6mm,
        right skip=1.5cm,
        %width=.8\textwidth,
        %valign=center,
        %left=10ex,
        %halign=flush center,
        %before=\par\smallskip\centering,
        finish={\node[anchor=north west, yshift=-2mm] at (frame.north east) {\includegraphics[width=10mm]{#4}};},
        title={#3:}, 
        rounded corners=north, rounded corners=southwest,
        colupper=white, colupper=black, colbacktitle=white,
        coltitle=black,
        before skip=1mm,
        #1}


\NewEnviron{rightbubbleslittle}[2]{\vspace{1mm}%
\tcbox[commonoptions,    
    finish={\node[anchor=north west, yshift=2mm] at (frame.north east) ;},%{\includegraphics{#2}};},
    rounded corners,
    colback=#1,
    colupper=white,
    before=\hfill,
    right skip=1.5cm,
    varwidth upper]
    {\BODY}
}    


\usepackage{enumerate}
\lstset{
language = python,   
breaklines = true,
numbers = left,
frame = tbrl,
tabsize = 4,
captionpos = t
}


%% https://www.kerislab.jp/posts/2019-01-14-vscode-latex/
%% 
% 数式
\usepackage{amsmath,amsfonts}
\usepackage{bm}
% 画像
\usepackage[dvipdfmx]{graphicx}

\begin{document}

\title{科学統計教程}
\author{Idiot}
%\date{2020/6/29}
\maketitle
%% TODO QQplot,検出力,確率密度関数,分散分析,t分布

\section{前書き}\label{introduction}



本書の特徴

注意点





\begin{enumerate}
    \item 統計検定のみで帰無仮説を含むモデルを採択または棄却する
    \item $N=30$であれば中心極限定理よりある特定の検定が使える
    \item データの出現頻度が正規分布によりよく近似できる場合のみを考える。または、標本分布が正規分布であることを前提とする。
\end{enumerate}
これらの魔術を使わずに統計学を使う方法および、推測可能なことについて考える。


\begin{enumerate}
    \item 仮説検定で推測可能な事象(type I,II error)
    \item モデルと現象の乖離により生じる大きな間違えを含む推定
    \item モデルの母数が現象を捉えていないことにより生じる事象
\end{enumerate}

孫引き引用をした箇所は孫引きしたと書いておいた。今後読んで、引用に修正することもある。

一般的な数理統計学の教科書\cite{2012統計科学の基礎,199005数理統計,1973確率,1963数理統計学,2009統計的機械学習,2005確率と統計,2016統計学,2017現代数理統計学の基礎,2020現代数理統計学}、統計モデルについては\cite{2012データ解析のための統計モデリング入門}。
生物学者が統計学を使うときの視点は\cite{2018統計思考の世界}に詳しいが、中心極限定理の説明が十分ではないと感じる。

\chapter{空想統計教程}
本書は統計学を使って科学的な推論・予測を行うプロセスを説明したものである。
ただし、このプロセスを通して研究を行っている人は少なく、類似書がほとんどない。
ゆえに本書が説明する全ては、空想である。
なぜこのような空想が必要であるのかを本章で説明する。
%私が知りたいことを知るためにどのような統計量が必要なのかを説明している。

\section{生物統計学の問題点}
一般的な生物統計学の本は、論文を査読プロセスに耐えさせるための方法論が記述されている。
そのため、生物学の各分野に特化されており、さまざまな特色がある。
例えば、
\begin{enumerate}
    \item 繰り返し検定を行っていき、最終的な検定方法を決定する。いわゆる検定フロー
    \item $p$値のさまざまな解釈間違い
    \item 統計モデルとの比較で言えそうにないことまで言う。
    \item $p<0.05$であるので対立仮説を採択する。
    \item 頻度論統計学を採用する。
\end{enumerate}
などである。
最後以外の特色は統計を"わかっている"科学者たちから否定的に批判され続けているはずであり、生物学者らもよく理解しているはずである。
ではなぜこれらの特色を使い続けているのか。
生物統計学の本を執筆しているのは、多くの論文を学術誌に投稿し、掲載された研究者であることが多い。
掲載には、2人程度の査読者が割り振られ、論文を査読され、そこで使われている統計がその学術において妥当であるかどうかが調査される。
この査読プロセスに耐えられるように統計を使うことが必要となり、その経験を元に執筆されるのが通常の生物統計学の書籍だろう。
目的が論文のアクセプトになっており、科学的に何をするべきなのかを考えることは二の次になっているのではないだろうか\footnote{空想である}。

生物統計学の本では、$t$検定を行う前に、正規性の検定を行うことを求めているものが多い。
正規性の検定を行っても正規分布からサンプリングしたか否かについて検証できないので、本書では推奨できない方法である。
これが通常の査読プロセスで要求されているものだとすると、正規性の検定を拒否した著者らの論文は出版されにくくなることが予想される。
論文が出版されなければその研究者の研究者としての寿命は短い。Publish or Perishである。
一方で、査読者の指摘通りの解析をして論文が出版された研究者はその後も研究者であり続けるはずであり、査読プロセスで得られた知見を元に生物統計学の本を書く。
もちろん$t$検定の前に正規性の検定を行うと解説する。
妥当とは言い切れない生物統計を学んだ研究者が査読に加わり、再び正規性の検定を求めていく。



%editorや査読者ではなく、自然をよく予測できるかどうかであるだろう。
%そのために必要となるデータや指標・統計量は何かを理解する必要がある。

帰無仮説を否定すれば、対立仮説を採択するというのも、生物統計の特色である。
言い換えると、イコールの仮説では$p<\alpha$だったから、イコールではない仮説が成立すると生物統計学では主張できることが多い。
この考えは本書では推奨しない。
イコールでない仮説についてなんら検証できていないからである。


最後にあげた特色、頻度論統計学が扱う対象は、繰り返し試行を行った結果得られる統計的性質を対象にしている。生物学のデータがこの対象に当てはまっていないにもかかわらず、生物学の多くの書籍では頻度論統計学を採用している。本書では、頻度論統計学の知見を元に、頻度論が対象にしていない範囲で推論を行う方法を説明する。言い換えれば、正規分布で予測できることがわからない対象に対してモデルを構築する考え方と推定の科学的方法について考察を行う。

\begin{SMbox}{生物統計学では言えないことも言える}
    生物統計学では、統計学では言えないことを言えないことを言っている。この理由を探してはいけない。やってるからやっているだけである。やってることを支えている理論はない。
    
    ただし、生物統計が扱うのは頻度論では扱えない範囲なので、統計学では言えないことでもある程度認めなければいけない事項がある。
\end{SMbox}

\section{ASAの$p$値に関する解釈}
$p$値に関する解釈は、アメリカ統計協会(ASA)が2016年に発表した声明に従う\cite{doi:10.1080/00031305.2016.1154108}。以下に引用しておく。
\begin{quote}
    P-values can indicate how incompatible the data are with a specified statistical model.
\end{quote}
日本計量生物学会が公開した翻訳では、
\begin{quote}
    $P$値はデータと特定の統計モデルが矛盾する程度をしめす指標のひとつである。
\end{quote}
この翻訳された文章ではincompitableを矛盾としている。本書では適合と翻訳する。
日本語の翻訳を書き換えると、$p$値は、データと特定のモデルの適合具合を示す指標の一つとして扱う。

%この声明に従った生物統計学の書籍は少ない。

ASAの関係者ではないので彼らが$p$値をどのように使ってほしいかを私は理解していない。ASAの発表を私の都合に合わせて使っているにすぎない。

\section{生物学的問い}
生物学的問いとして、初等的な生物統計学の書籍に提示されているのは、
\begin{enumerate}
    \item OO条件とXX条件では~という特性に$\cdots$という差がある
    \item A群とB群が同じである
    \item A群とB群について平均値に差があるか
\end{enumerate}
などがある。現在の生物統計学においては、これらの問いを有意差検定を用いて解決を試みている\footnote{失敗している}。
具体的には、母数が同一のモデルを仮定し、そのモデルにデータが適合するかを検証する。
これでは、統計検定量について適合しなさそうなモデルが明らかになるだけである。
本書では、これらの問いについてA群とB群に共通する特徴について、その特徴を測定した標本が適合するモデルをそれぞれの群に対して特定する。
さらに、それらのモデルの性質の違いを明らかにする。
ここまでが本書で扱う内容である。
生物学ではさらに、それらの適合モデルの違いが生物学的にどのような影響をもたらすのかについて考察する必要がある。
この点は、本書で扱っていない。



%\section{データが汚いので予測は手順を踏むだけ}

%言い換えるならば、統計学を使って一般のことについて推測しやすいデータを得られたときに言えることを学ぶことで、より難しいデータが得られたときに、言ってはいけないことを学ぶべきだろう。


\chapter{推論}
科学におけるモデルおよび、数理統計学におけるモデルについて説明し、その違いを明らかにする。

\section{モデル}
モデル(模型)とは、現実を表していると思わせるような、作られたものであり、次の特徴を備えています。
\begin{enumerate}
    \item モデルは本物の特徴の一部を推測可能。本物との乖離の程度も推測できる
    \item (1)を行うために、複数の仮定により構築される。また、それらの仮定は、現状の知識では明らかではないまたは、現実的には成立していないことがある。
    %推測可能なことを増やすために仮定を増やすことがある。
    % \item モデルは本物の要素・特徴の全体を推測することはできない。
    \item モデルは間違った推測をする。
\end{enumerate}
  
例えば、車のプラモデルはモデルの一つです。本物の特徴の一つである大きさを推測可能にするため、スケール(例えば、1/24など)を決めて作られている。ドアや車体の幅を計測し、スケール倍すれば、本物の大きさを推測できる。普段長さを測れない場所であっても、手のひらに収まるプラモデルであれば、どの部分でも推測が可能になる。言い換えれば、本物の車がなくても、スケールを維持した車のプラモデルを持っていれば、簡単に大きさに関する推測が可能になる。



\if 0
モデルの仮定によって推測可能なことが増えることがある。
例えば、本物のパーツの重量に応じて、モデルのパーツの重さを変化させる。
こうすることで、プラスティックでできたモデルの重さを計測すると、そこから本物の重さの推測が可能になる。
他にも推測したいこと、例えば、素材の質感や色や重量、または車の速度といった特徴なども、モデルに仮定を追加することで、本物の様子を捉えることが可能になっていく。
\fi

本物の車を持って来れば、本物の様子を推測することが可能であるので、本物の車は、車自身のモデルということができるが、車を車自身のモデルとすると、それまであった利便性が損なわれる。
%また、モデルの仮定を増やせば、本物の車にモデルを近づけることが可能である。
おおよその車体の長さが知りたいのに、わざわざ長い測りが必要になることや、手に持って観察することもできない。
このように、細部まで推測可能にするというのは、デメリットになることがあり、モデルとして利用することはない。

細部まで推測可能なモデルは使うことは稀であり、車のモデルとして、大きさの尺度を保っていない直方体のブロックを使うことがある。このモデルでも推測できることがある。3台の同じ車を縦列駐車するのに必要な長さなどは、直方体三つ分と推測が可能である。
モデルの作り込みの程度によって車の特徴に関して推測できることの種類が決まる。
%が異なる。統計学を利用するときは、目的について考えることは少なく、儀式的に採用した手順に従うことが多い。

真球を車のモデルし、車の大きさに関する推測を行うと、現実の大きさと推測は大きく乖離することが考えられる。
モデルが本物の推測に使えないということに判断を下すには、本物のデータとモデルの出す推測を複数の指標から比較し考察することになる。


軽自動車に対してその大きさを予測可能なモデルを使って、トラックの大きさを予測できる。
予測できるが、その予測値は実際のトラックの大きさと異なる。
モデルが車体長を3.4mであると予測される。実際のトラックの車体長は6mよりも大きい。
メートル単位でモデルと実際には差が生じる。
このように、モデルと実際を調べることで、このモデルではトラックの大きさを推測できないと判定できる。
実際には、どれくらいの誤差が生じたときに、モデルが使えないというのかは、予測したいことにより異なる。

%車のモデルという題材からモデルの特徴を挙げた

モデルは本物ではないが、推測に役にたつ物として利用する。モデルと本物が極めて一致するように感じられることもあると思うが、モデルは本物ではない。



\section{統計モデル}
統計モデルについて説明し、モデルを使って現実を推測することを概念図を用いて説明する。
まず、統計モデルは、数理統計の知識を使いモデルを構築され、現実を推測するために用いられる。簡単な統計モデルを例に挙げると、次のような仮説から構築される。

\begin{enumerate}
    \item (仮定1)確率変数が同一の分布から独立に得られる(i.i.d)
    \item (仮定2)その分布関数は、$f(x)$と書ける。
    \item (仮定3)分布関数の母数に関する仮説\footnote{三番目の仮説のみを統計モデルと主張する流派もある\cite{塩見_正衛2021}}
\end{enumerate}

\subsection{統計モデルとデータ}
%統計モデルは予測を
データに統計モデルがよく当てはまるよう指標を定め、その指標を小さくするようにモデルの母数を推定できる。



\subsection{データへの過剰適合}
モデルは改訂することにより、予測の精度をあげることができた。これは、何度も対象を観測することで、モデルと実際の当てはまりを定量的な評価が可能であるから、モデルの作り込みを防ぐことができる。
再現性の確保されている現象に対しては、データに当てはまるようにモデルに仮定を足していき、モデルの作り込みを行う。さらに新たなデータとモデルの予測とを定量的な指標を元に評価する。
一方で、何度も繰り返し観測可能でない現象を対象にした学問領域において、モデルの作り込みは現在得られているデータを過度によく予測するモデルとなることがある。その結果、構築したモデルが新に得たデータに対して予測精度が落ちてしまうことが多々ある。
そのため、データを見た後に、モデルに仮定の追加または変更はしない方が良い。




\subsection{統計モデルの仮定を自然が満たしているのか}
統計モデルにより推定したい対象またはデータが、統計モデルの仮定から外れていることは多々ある。
まず仮定1、独立性と同一の分布という仮定は、数学的厳密な定義がある。
その定義を現実の世界にの言葉に変換すること自体が難しい。
まず、各変数が独立とは、事象A,Bが同時に起きた確立$P(A,B)$がそれぞれが生じる確率の積に等しいということであり、$P(A,B)=P(A)P(B)$である。そもそも事象生じる頻度が$P$により決定されているということを考えることができない。それに加えて$P(A,B)=P(A)P(B)$なども、現実世界の事象に一致する概念がない。
%\footnote{同様のものとして、尤度も現実に対応する概念がない。よく尤もらしさと言い換えられるが、間違っている。尤度は$P(A)P(B)P(C)\cdots$である。それ以外の言い換えは間違いである。}。

間違っていることを承知の上で、科学的な言葉に変換して、妥当であるかを考察してみる。
あえて、得られたデータに相関が全くないと、捉えてみると、現実的には妥当ではないことの方が多い。例えば、人の身長を計測器により繰り返し観測すると、その計測器や扱う人の癖がデータに含まれ、それはデータの傾向を決定する因子となり、データ間には相関があると考えられる。
もし、相関がない実験デザインを設定できたとしても、人の身長はその背景にある社会や遺伝的な繋がりが因子となっており、相関が無いと言い切ることは難しい。

同一の分布とは、同一の数学的規則に自然が支配されていることを仮定していると考えられる。コインのトスでは、その裏表の出現確率を二項分布によるものと考えても問題が大きくならない。一方で、人の身長は、母父の大きさや成長過程における栄養の量などの因子によって成長すると考えられる。この現象が、サイコロのように乱数をふって決定されていると考えるのは妥当とは言い切れない\footnote{そう考えてもいいけど、あまり役に立たない}。
%統計モデルの数学的な仮定が科学と対応しているとは言い難い。
%モデルは、妥当とは言い切れない仮定により構築されている。


統計モデルを現実の推測に使えないということではない。モデルと現象を比べて予測するためだけにモデルを利用するのであるから、仮定が現実に存在するかはどうでも良い。
\begin{SMbox}{有用な近似が得られるからモデルを使う}
    Boxらは、統計学において正規分布や一次関数で推論することを次のように捉えている\cite{box1976science}。
\begin{quote}
    Equally, the statistician knows, for example, that in nature there never was a normal distribution, there never was a straight line, yet with normal and linear assumptions, known to be false, he can often derive results which match, to a useful approximation, those found in the real world. 
\end{quote}
    %データが正規分布的に分布しているということは、ある中心に対して対照的にデータが生じていることを示す。このとき、正規分布でデータを予測するのが良いのではないかと考える。
\end{SMbox}
%以下では、統計モデルを$M(\bm{a})$とし、ここで$\bm{a}$は、仮定3の統計モデルの母数であり、母数が複数あることも考慮し、ベクトルで表記しておく。

\begin{SMbox}{学問間に生じているモデルに関する認識の違い}
モデルが本物であるか否かは、学問領域によって認識が異なっている。
私は、モデルは現実を推測するための偽物のことだと考えている。
モデルが自分の知りたいことをうまく予測してくれさえいればいいという立場である。
一方で、数学では、モデルを現実と捉える傾向がある。モデルにより世界が支配されていると考えているのである。例えば、ある数学者は、流体モデルに解が安定的に存在するかがわからないから飛行機に乗りたくないと思っていると言う雰囲気がある。
%現代の統計学はどちらかと言うと数学者が作った枠組みを統計ユーザーが受け入れてしまったため、ユーザーたちは、数学者のように現象を捉ようとしているように見える。

実際にモデルに対する認識が研究者によって異なっていると感じている人はいる。
学習理論を研究しておられる渡邊 澄夫さんは、情報科学と物理学におけるモデルとして次のような見解を述べている。
\begin{quote}
    (注意)このことを聞いたとき,どのように感じるかは,人によって ずいぶん違います.情報科学の研究者の人たちは,「目的が違うのだから, 最適なものが違うのは当然であり,まったく不思議ではない」と感じる場合が多いようです. 一方,物理学の研究者の人たちは, 「真の法則が発見できるということと,最良の予測ができることとは,ぴったりと 同じであるべきである」と感じるようです.これは,おそらく,「モデル」という 概念や重みにおいて,情報科学と物理学では大きな隔たりがあることが原因ではないかと 思います.(例題:電子の質量が正確に予言できるのは,量子電磁力学が真の自然法則であるからと 考えられています).

    生物学・環境学・経済学に用いられる「モデル」は,上記の意味での情報科学におけるモデルに近いのか, 物理学における理論に近いのか,それとも,その中間に当たるのか,もっと違う種類のものなのか,は,かなり微妙な質問で,一様な回答はないものと思います. 数学者のかたが数理科学の研究に挑もうとされるときには,「モデル」という言葉が表すものが,分野において,場合において,このように様々に異なりうることを認識されておかれるとよろしいでしょう。
    \\ \url{http://watanabe-www.math.dis.titech.ac.jp/users/swatanab/Bayestheory2.html}
\end{quote}

統計や機械学習の分野で有名なBox氏は、「全てのモデルは間違いである(All models are wrong)」と、次のように説明している。

\begin{quote}
    For such a model there is no need to ask the question "Is the model true?". If "truth" is to be the "whole truth" the answer must be "No". The only question of interest is "Is the model illuminating and useful?".
\end{quote}


\end{SMbox}


\if 0
\begin{mybox}
    %\begin{quotation}
    \paragraph{頻度主義・ベイズ主義}
    頻度主義・ベイズ主義は統計学の流派を表す言葉である。頻度主義者であると言う人はあまり見たことがないが、ベイズ主義者はよく見る。ベイズ主義者は頻度主義者を非難するような主張をすることが多々ある。それぞれの立場を正確に説明した文献がないので、何を意味しているのかを私は理解できていない。

    おそらく、頻度主義では、モデルと母集団を一致させて考えており、このことを念頭にすれば頻度主義的な議論が理解できると思う。この立場にたてば、中心極限定理がデータにも適応可能になり、あらゆるデータが正規分布で推定可能にることを主張できる\footnote{本当にそう考えているのか確信が持てない}。
    %\end{quotation}
\end{mybox}
\fi


\if 0
\begin{brokenbox}[colback=yellow]
    \blindtext[5]
  \end{brokenbox}
\fi 
\subsection{数理モデルの機能}
数理モデルには予測・サンプリングという機能がある。%それぞれ説明していく。
\paragraph{予測}

次に説明するサンプリングを使うことで出現しやすい場所を数値的に計算することが必要となるモデルもある。

\paragraph{サンプリング}
サンプリングは、モデルを使ってデータを生成する方法である。モデルが説明したいデータの出現頻度をよく予測できるなら、モデルが生成したデータは実際に得られるデータと似たものになる。

\section{数理統計学におけるモデル}
数理統計学は、モデルが生成した有限個の確率変数からモデルの母数を推測する方法論を提供している。



\if 0
\subsubsection{オッカムの剃刀}

仮定の追加には合理的な理由が必要だと考えられます\footnote{仮定を追加した統計モデルはベイズ統計と書かれた本で学ことができます。}。

\begin{figure}
    \begin{center}
%\includesvg{../markdown/section1/statistics_model.svg}
\end{center}
\end{figure}
\fi

\section{統計学の用語}
統計学の言葉をいくつか借りて、本来の意味とは異なった定義で使う。


\subsection{母集団、無作為抽出、サンプリング}
母集団は興味のある対象全体の集団のことである。例えば、17歳男性の身長に関心があるならば、17歳男性の全員の集合が母集団である。日本人全体の身長に関心があるならば、日本人全員の集合が母集団である。

無作為抽出とは、偏りなく母集団からデータを取得することである。
無作為抽出することで、都合の良い結果が集まらないようにしている\footnote{無作為抽出しなければならないのはモデルの仮定1を満たすためだという主張を見かけたことがある(文献を探すべき)。モデルの仮定を現実が満たすようにすることはできないので、このように考えない方が良い。行き着く先はモデルの仮定を満たすように、検定を繰り返すようになる。もちろん検定ではモデルの仮定を満たしているかを決定することはできない。}。
本書では、モデルから確率変数を生成することをサンプリングとカタカナで記述し、現実の作業である無作為抽出と区別する\footnote{この使い分けは一般的でないし適切ではない。}。

\subsection{誤差・揺らぎ}
計測上の手順で生じるデータの差異の平均と各データの差分のことを誤差と呼ぶ。
誤差が生じるのは測定者の違いや、計測装置の精度に依存する。

揺らぎとは、ある集団における個体間の差異である。
例えば、ある畑で採集された野菜の重量の個体間の差異を揺らぎと呼ぶ。

本書では誤差は、揺らぎよりも十分小さいものとして扱い、揺らぎの性質についてモデルを構築する。

\subsection{標本、サンプルサイズ、擬似反復、標本数}
\begin{defi}
母集団から無作為抽出して得た標本に含まれるデータの個数をサンプルサイズ(標本の大きさ)といい、その数を$T$や$n$で表す。同じ実験を繰り返して行ない、複数の標本を作ると、その標本の個数を標本数という。
モデルからサンプリングした場合も、その確率変数の集まりを標本という。
モデルの標本において、標本の大きさが大きいものを大標本、小さいものを小標本と言う。
\end{defi}
例えば、無作為抽出しデータを$20$個得る実験を30回繰り返した場合、サンプルサイズ$20$の標本を$30$得たことになる。言い換えれば、標本数$30$で、サンプルサイズは$20$であると言う。


擬似反復は、同じ個体においてその特徴を複数回計測し、これを揺らぎとして集団の差異として捉えることである。
例えば、17歳男性の身長について計測することを計画する。
サンプルサイズとして、100個のデータ点から計測することにしたので、$10$人から$10$回身長を計測した。結果、$100$個の計測データが集まった。
このデータでは、通常の$17$歳男性の身長に関する統計モデルと乖離していると結論がつけられやすくなる。

%あるクラスで身長に関して計測することを考える。$A$さんに関しては10回計測し


サンプルサイズを標本数と言う流儀の学問もあるようなので注意が必要である
\footnote{業界によって様々な慣習があり(\url{https://biolab.sakura.ne.jp/sample-size.html})、業界の慣習に(師匠の言うことに)従った方が余計なトラブルを減らせると考えられる(\url{https://www.jil.go.jp/column/bn/colum005.html})。この言葉くらいは統一して記述したい。本書でも途中で間違った使い方をしてしまうかもしれないが、なるべく間違わないようにしたい}。


\section{モデルを使った推測}
d

\begin{figure}
    \begin{center}
        \includegraphics[width=15cm]{./image/01_/conceptual_diagram/conceptual_diagram.002.png}
        \caption{統計モデルによる現象の推測に関する概念図}
        \label{fig:conceptual_diagram_statistics}
    \end{center}
\end{figure}
    


%\chapter{数理統計学}
データの出現頻度を近似する式である確率密度関数、累積分布関数について説明し、様々な形の確率密度関数について説明する。
さらに、特定の分布に従う確率変数が、その分布関数から生成された確率変数であることを確かめる方法について説明する。
最後に、モデルの確率変数への当てはまりの良さの相対的な指標である尤度を導入し、尤度を最大にする母数を推定する方法を説明する。
さらに、モデルのパラメータの数に対するペナルティを導入した指標のAICを導入する。

\subsection{確率密度関数}

\subsection{累積分布関数}
aaa

\subsection{相補累積分布関数}
$1$から累積度数を引いたものは、相補累積分布関数と呼ばれ、ある値よりも大きな値をえる確率示し、数式では、
\begin{eqnarray}
    1-F(x) &=& f(X>x) \\
        &=& \int_{x}^{\infty} f(z)dz.
\end{eqnarray}
図\ref{fig:standard_normal_distribution}(c)に図示した。累積分布関数と相補累積分布関数のどちらかを表示するかは、分野によって異なる。
生物学の分野などでは、より大きな値を得る確率を重視することがあるので、累積分布関数よりも、相補累積分布関数が好まれることがあるように私は感じている。

\section{確率変数}
\subsection{確率変数がある分布関数に従う}
確率変数$x$が、ある分布関数に従うとは、

$x$をたくさん集めて、$x_1,x_2,\cdots,x_n$という標本を得たときに、その出現頻度がその分布関数に精度よく近似できる。


\section{正規分布}
正規分布の確率密度関数は、
\begin{equation}
p(x;\mu,\sigma)=\frac{1}{\sqrt{2\pi\sigma^2}}\exp\left(-\frac{(x-\mu)^2}{2\sigma^2} \right)
\end{equation}
ここで、$\mu,\sigma^2$は、正規分布のパラメータで、それぞれ母数平均、母数分散です。
母数平均は最も出現頻度の高い数値を表しており、この値を中心にし、対象に分布が広がります。言い換えれば、$\mu-a$と、$\mu+a$の出る確率は同程度になります。
母数分散は、数値のまとまり具合を示します。$\sigma$が大きくなるほど、$\mu$の近くの数値が出現する頻度は小さくなり、より離れた場所での出現頻度を高くします。
正規分布関数に確率変数が従うことを$X\sim N(\mu,\sigma^2)$とかく。



正規分布においてその母数を$\mu=0,\sigma=1$とするとき、標準正規分布といい、$N(0,1)$で表す。確率変数$Z$が標準正規分布に従うとき、その確率密度関数は
\begin{equation}
\phi(z) = \frac{1}{\sqrt{2\pi}}\exp(-\frac{z^2}{2})
\end{equation}
であり、図\ref{fig:standard_normal_distribution}(a)である。
標準正規分布の累積分布関数は、
\begin{eqnarray}
\Phi(x) &=& p(X<x; 0,1) \\
    &=& \int_{-\infty}^x \phi(z)dz \\
    &=& \frac{1}{2}(1+\rm{erf}\frac{x-\mu}{\sqrt{2\sigma^2}})
\end{eqnarray}
であり、図\ref{fig:standard_normal_distribution}(b)である。

相補累積分布関数は、
\begin{eqnarray}
    1-\Phi(x) &=& p(X>x; 0,1) \\
        &=& \int_{x}^{\infty} \phi(z)dz.
\end{eqnarray}


\begin{figure}
    \begin{center}
        \includegraphics[width=15cm]{./image/02_/standard_normal.pdf}
        \caption{標準正規分布(a)確率密度関数(b)累積度数分布(c)1-累積度数分布}
        \label{fig:standard_normal_distribution}

    \end{center}
\end{figure}
    

\subsection{正規分布に従う確率変数の出現しやすさ1}
標準正規関数に従う確率変数が$95\%$の確率で見つかる範囲を求めてみます。
標準正規関数は、0を中心にして、対称な関数なので、正負の値が同じ程度の確率で見つかります。言い換えれば、$0\sim a$までの積分値と、$-a\sim 0$までの積分値が同じになります。そこで、次の積分を考えて、その最小値となる値を見つけてみます。
\begin{equation}
\int_{-a}^{a} \frac{1}{\sqrt{2\pi}}\exp(-\frac{z^2}{2}) dz = 0.95
\end{equation}

\begin{lstlisting}
b,a = norm.interval(0.95,0,1) # 積分値が0.95になる範囲を計算
print(norm.cdf(b, loc=0, scale=1)-norm.cdf(a, loc=0, scale=1)) # 0.95になるかを確認
print(b,a) # その範囲を表示
\end{lstlisting}


$0<\alpha<1$に対して、$\Phi(z_\alpha) = 1-\alpha$となる$z_\alpha$を上側$100\%$点という。
$z_{0.05}=1.64,z_{0.025}=1.96$の値は後でよく使う。

より、一般的には、$\alpha(0\leq \alpha \leq 0)$を指定すると、その半分$\alpha/2$となる積分範囲の末端を$a_1$とします。数式で書くと、
\begin{equation}
    \int_{-\infty}^{a_1} \frac{1}{\sqrt{2\pi}}\exp(-\frac{x^2}{2})dx = \frac{\alpha}{2}.
\end{equation}
同様に、右側の範囲の末端を$a_2$とします。数式で書くと、
\begin{equation*}
    \int_{a_2}^{\infty} \frac{1}{\sqrt{2\pi}}\exp(-\frac{x^2}{2})dx = \frac{\alpha}{2}.
\end{equation*}
これを書き換えると、次と同値です。
\begin{equation*}
    \int_{-\infty}^{a_2} \frac{1}{\sqrt{2\pi}}\exp(-\frac{x^2}{2})dx = 1-\frac{\alpha}{2}.
\end{equation*}

\begin{figure}
\begin{center}
    \includegraphics[width=15cm]{./image/02_/z_value.pdf}
    %\caption{図1.p値cm}
  \end{center}
\end{figure}

標準正規分布$z\sim N(0,1)$において$95\%$の確率で確率変数が見つかる範囲を調べることはできましたが、正規分布$x\sim N(\mu,\sigma^2)$においてでは、どの範囲になるのでしょう。次の定理を使えば簡単に計算ができます。
\begin{theo}
    確率変数$x$が、$x\sim N(\mu,\sigma^2)$であるならば、$\frac{x-\mu}{\sigma}\sim N(0,1)$である。    
\end{theo}
\begin{theo}
$\alpha(0\leq \alpha\leq 1)$に対して、$\int_{-\infty}^{z}\frac{1}{\sqrt{2\pi}}\exp(-x^2/2)=\alpha$を満たすとき、$\int_{-\infty}^{\mu+\sigma z} \frac{1}{\sqrt{2\sigma^2}}\exp(-\frac{(x-\mu)^2}{2\sigma})=\alpha$である。同様に、$\int_{z}^{-\infty}\frac{1}{\sqrt{2\pi}}\exp(-x^2/2)=1-\alpha$を満たす$z$について、$\int_{\mu+\sigma z}^{\infty} \frac{1}{\sqrt{2\sigma^2}}\exp(-\frac{(x-\mu)^2}{2\sigma})=1-\alpha$である。
\end{theo}
言い換えれば、標準正規分布の軸上の点$z$を、$[-\infty,z]$の範囲での積分値を保ったまま、正規分布$N(\mu,\sigma^2)$上の点に変換するには、$\frac{x-\mu}{\sigma}=z$を$x$について解けば良いことになります。

この定理により、以下をとけば、値が$95\%$の確率で得られる範囲がわかります。
\begin{eqnarray*}
    \frac{x-\mu}{\sigma}=z_{0.025}\\
    \rightarrow x = \mu+\sigma z_{0.025}
\end{eqnarray*}
また、
\begin{eqnarray*}
    \frac{x-\mu}{\sigma}=-z_{0.025}\\
    \rightarrow x = \mu-\sigma z_{0.025}
\end{eqnarray*}
以上により、$x \sim N(\mu,\sigma^2)$が$95\%$の確率で見つかる範囲は、$[\mu-\sigma z_{0.025},\mu+\sigma z_{0.025}]$であることがわかります。
同様に$90\%$の確率で見つかる範囲は、$[\mu-\sigma z_{0.05},\mu+\sigma z_{0.05}]$です。

\subsection{より大きな値をとる確率}
$x$を標準正規分布の確率変数とし、($x\sim N(0,1)$)また、$x\leq 0$であるとします。。$x$以上の大きな値を取る確率は、$P(X>x)=1-\varPhi(x)$で計算できます。
同様に、$x < 0$であるときは、より小さな値を取る値が、$P(x<X)=\varPhi(x)$で同様に計算できます。
図\ref{fig:z_value_larger}には、$x$に対して、より異なった値を取る確率を書いています。

$x$の大きさ$|x|$よりも大きな値を取る確率は、以上の二つの和で次のようにかけます。
\begin{equation}
    P(|x|>z) = 1-\varPhi(|x|)+\varPhi(-|x|)
\end{equation}
式を見ると正の数で$x$より大きな値を取る確率と、負の数で$x$より小さな値を取る確率の和になっていることが確認できます。
$P(|x|>z)$はより極端な値を取る確率などと言う方もされます。

計算してみます。$x=1.64$であれば、$\varPhi(1.64)=0.95$より、それ以上に大きな値を得る確率は、$P(X>1.64)=0.05$です。また、$x=-1.64$であれば、$\varPhi(-1.64)=0.05$です。よって、$|x|=|1.64|$よりも大きな値を得る確率は$P(|1.64|>X)=0.1$です。


\begin{figure}
    \begin{center}
        \includegraphics[width=15cm]{./image/02_/z_value_larger.pdf}
        \caption{標準正規分布におけるより大きな値(より偏った値)を取る確率。(a)$z=1.64$より大きな値を取る確率は0.05。(b)$z=1.96$より大きな値を取る確率は$0.025$。(c)$z=2.24$よりも大きな値を取る確率は$0.0125$}
        \label{fig:z_value_larger}
      \end{center}
    \end{figure}

\subsection{$N(0,1)$での珍しい値は、$N(0,2)$では珍しくない?}
以上の議論により、$N(0,1)$において、$z=1.64$以上の値が出る確率はおよそ$5\%$である。
では、$N(0,2)$において$z=1.64$が出る確率はいくつだろうか。
$N(0,2)$において、$z=1.64\times2$以上に大きな値が出る確率は、およそ$5\%$である。
このことから、$N(0,2)$において$z=1.64$以上の値が出る確率は、$5\%$より大きいことがわかる。
具体的に、計算をしてみると、その確率は$0.206$程度であることがわかる。
\begin{lstlisting}
1-norm.cdf(1.64,0,2)
\end{lstlisting}

\subsection{$N(1.96,1)$で出てくる値は、$N(0,1)$において珍しい?}
$N(1.96,1)$において、$1.96$以上の値が出る確率は、$50\%$です。明らかに、よく出る値であることがわかります。
一方で、$N(0,1)$においては、$1.96$以上の値が出る確率は、$2.5\%$くらいなので、珍しい値になります。
このように、確率分布の母数が変化すると、珍しい値も変化します。




\subsection{正規分布に従う確率変数の出現しやすさ2}
確率変数のしやすさを表す基準として、$\sigma$を基準にして、定数$a$倍の範囲$[\mu-a\sigma,\mu+a\sigma]$を使う方法もあります。
標準正規分布では、分散が$1$なので、その$0.5$倍、$1$倍、$2$倍、$3$倍の範囲はそれぞれ$[-0.5,0.5]$,$[-1,1]$,$[-2,2]$,$[-3,3]$になります。この範囲に入る確率は、それぞれ$0.38$,$0.683$,$0.954$,$0.997$です。それぞれの範囲と確率は、図\ref{fig:sigma_interval_probability}に図示しました。

$\sigma$の定数倍の範囲に値が見つかる確率は、$\sigma$の大きさに依存しないことが証明できます。言い換えれば、$[-0.5\sigma,0.5\sigma],[-\sigma,\sigma],[-2\sigma,2\sigma],[-3\sigma,3\sigma]$の範囲に値がある確率は、上記と同じで、それぞれおよそ$0.38$,$0.683$,$0.954$,$0.997$になります。


\begin{figure}
    \begin{center}
        \includegraphics[width=15cm]{./image/02_/sigma_value.pdf}
        %\caption{図1.p値cm}
        \label{fig:sigma_interval_probability}
      \end{center}
\end{figure}

\begin{table}[hbtp]
    \caption{$\sigma$を基準にした値の出やすさ}
    %\label{table:data_type}
    \centering
    \begin{tabular}{lcr}
        \hline
        出現確率  & $N(0,1)$  &  $N(\mu,\sigma^2)$ \\
        \hline \hline
        0.38 & [-0.5,0.5]  & $[\mu-0.5\sigma,\mu+0.5\sigma]$ \\
        0.683 & [-1,1] & $[\mu-\sigma,\mu+\sigma]$\\
        0.954 & [-2,2] & $[\mu-2\sigma,\mu+2\sigma]$\\
        0.996 & [-3,3] & $[\mu-3\sigma,\mu+3\sigma]$\\
    \end{tabular}
\end{table}




\section{指数分布}
確率変数$X$が指数分布に従うことを$X \sim Exp(\lambda)$と書く。
指数分布の確率密度関数は、
\begin{equation*}
    f(x)=\lambda \exp(-\lambda x).
\end{equation*}
ここで、$\lambda$は、$\lambda>0$であり、指数分布の母数である。
期待値は$E[X]=\frac{1}{\lambda}$で、分散は、$V[X]=\frac{1}{\lambda^2}$である。
累積分布関数は、
\begin{equation*}
    F(x)=1-\exp(-\lambda x).
\end{equation*}
正規分布は、母数平均を中心として、左右対称に分布していた。言い換えれば、$\phi(\mu+x)=\phi(\mu-x)$である。一方で、指数分布は、左右非対称に分布が広がり、小さな値は大きな値よりも出現確率が高いので、$f(E[X]+a)\neq f(E[X]-a)$である。
また、正規分布では、母数平均と母数分散がそれぞれ独立なので、それぞれの特徴を独立に動かすことで、期待値や分散が独立に変化する。
指数分布では、母数が一つであり、母数を変化させると、期待値と分散は同時に変化する。



\begin{figure}
    \begin{center}
        \includegraphics[width=15cm]{./image/02_/expon_frequency.pdf}
        \caption{指数分布$\lambda=1/15$(a)確率密度関数(b)累積度数分布(c)相補累積度数分布}
        \label{expon_frequency}
    \end{center}
\end{figure}


\subsection{指数分布に従う確率変数の出現しやすさ}
指数分布の確率密度関数を区間$[a,b]$で積分したときに、$\alpha(0\leq \alpha \leq 1)$になる$[a,b]$を求めます。条件として、
\begin{eqnarray*}
    \int_0^{a}  \lambda\exp(-\lambda x )dx &=& \alpha/2\\
    \int_0^{b} \lambda\exp(-\lambda x )dx &=& 1-\alpha/2
\end{eqnarray*}
を満たすとする。
$a$について、とくと、
\begin{eqnarray*}
    \int_0^{a}  \lambda\exp(-\lambda x )dx &=& \alpha/2\\
     1-\exp(-\lambda a) &=& \frac{\alpha}{2}\\
     \rightarrow a&=& \frac{1}{\lambda} \log\frac{1}{1-\alpha/2}
\end{eqnarray*}
$b$については、同様に、
\begin{equation*}
    b = \frac{1}{\lambda}\log\frac{\alpha}{2}
\end{equation*}
以上より、この積分の条件で、$100(1-\alpha)\%$の確率で値を得る範囲は、$[\frac{1}{\lambda} \log\frac{1}{1-\alpha/2} ,\frac{1}{\lambda}\log\frac{\alpha}{2}]$である。
図\ref{fig:expon_simulation_sample}は、指数分布により、サンプルサイズ$1000$の標本を$100$回作って、各標本においてデータが区間$[\frac{1}{\lambda} \log\frac{1}{1-\alpha/2} ,\frac{1}{\lambda}\log\frac{\alpha}{2}]$に入った割合をシミュレーションし、そのヒストグラムを表示している。確かに、$95\%$くらいの割合でその区間にデータが入っている。


\begin{figure}
    \begin{center}
        \includegraphics[width=15cm]{./image/02_/expon_simulation_sample.pdf}
        \caption{指数分布$\lambda=1/10$からサンプルサイズ1000の標本を100回シミュレーションし、各標本においてデータが区間$[\frac{1}{\lambda} \log\frac{1}{1-\alpha/2} ,\frac{1}{\lambda}\log\frac{\alpha}{2}]$に入った割合を計算した。そのヒストグラム。}
        \label{fig:expon_simulation_sample}

    \end{center}
\end{figure}


\section{カイ二乗分布}
確率変数$X$がカイ二乗分布に従うことを$X \sim \chi^2_k$と書く。ここで、$k$はカイ二乗分布の母数で、自由度を示し、自然数を取る。
確率密度関数は、
\begin{equation*}
    f(x;k) = \frac{1}{2^{k/2}\Gamma(k/2)}x^{k/2-1}\exp\left(-\frac{x}{2}\right).
\end{equation*}
ここで、$\Gamma(k/2)$はガンマ関数を表す\footnote{$ \Gamma(z)=\int_0^{\infty }t^{z-1}\exp(-t)dt$である。 }。
累積分布間数は、
\begin{equation*}
    F(x) = \frac{\gamma(k/2,x/2)}{\Gamma(k/2)}.
\end{equation*}
ここで、$\gamma(k/2,x/2)$は、不完全ガンマ関数である\footnote{$\gamma(a,x)=\int_0^x t^{a-1}\exp^{-t}dt$である。ガンマ関数も、不完全ガンマ関数も計算できなくても問題はない。コンピュータを使えばすぐに計算してくれる。}。
この関数も左右非対称である。


\begin{figure}
    \begin{center}
        \includegraphics[width=15cm]{./image/02_/chi2_frequency.pdf}
        \caption{カイ二乗分布}
        \label{chi2_}
    \end{center}
\end{figure}

\subsection{カイ二乗分布に従う確率変数の出現しやすさ}
カイ二乗分布の確率密度関数を区間$[a,b]$で積分したときに、$\alpha(0\leq \alpha \leq 1)$になる$[a,b]$を求めます。条件として、
\begin{eqnarray*}
    \int_0^{a}  \frac{1}{2^{k/2}\Gamma(k/2)}x^{k/2-1}\exp\left(-\frac{x}{2}\right)dx &=& F(a)-F(0) = \alpha/2\\
    \int_0^{b} \frac{1}{2^{k/2}\Gamma(k/2)}x^{k/2-1}\exp\left(-\frac{x}{2}\right)dx &=& F(b)-F(0)= 1-\alpha/2
\end{eqnarray*}
を満たすとする。
代数的に$a,b$について解くことが難しいので、数値的に計算してみた結果を載せておく(表\ref{table:chi2_confidence})。この$a,b$をそれぞれ$\chi^2_k(\alpha),\chi^2_{k}(1-\alpha)$と書くことがある。


\begin{table}[hbtp]
    \caption{$\alpha=0.05$}
    \label{table:chi2_confidence}
    \centering
    \begin{tabular}{lcc}
    %\hline
    k  & $a$   & $b$   \\
    \hline \hline
    1 &  0.0009 &  5.02\\
    3 & 0.215 & 9.3484  \\
    5 &  0.831 & 12.832 \\
      \hline
    \end{tabular}
  \end{table}

\section{$t$分布}
確率変数$T$が$t$分布に従うとき、$T \sim t(\nu)$と表記する。
確率密度関数は、
\begin{equation*}
    f(t) = \frac{\Gamma((\nu+1)/2)}{\sqrt{\nu \pi}\Gamma(\nu/2) }(1+t^2/\nu)^{-(\nu+1)/2}.
\end{equation*}
ここで、$\nu$は、$0$より大きな実数である。
この関数を見ただけでは、すぐには判別するのは難しいかもしれないが、$f(t)$には$t$が関係する部分は$(1+t^2/\nu)$だけである。二乗の項があるので、偶数関数であることがわかり、$0$を中心にした対称な関数$f(t)=f(-t)$であることがわかる。
累積分布関数は著者には難しすぎるので、記述しない。wikipediaなどで調べれば正しそうな数式が書かれている。



\begin{figure}
    \begin{center}
        \includegraphics[width=15cm]{./image/02_/student_t_frequency.pdf}
        \caption{t分布}
        \label{student_t}
    \end{center}
\end{figure}

\subsection{$t$分布における珍しい値}
$t$分布における$|T|$以上の値が得られる確率が$\alpha$程度になる$|T|$のリスト。
例えば、$n=10$の$t$分布において$|T|=1.81$以上の値が得られる確率は、$0.1$程度である。


\begin{table}[hbtp]
    \caption{$t$分布における$|T|$以上の値が得られる確率が$\alpha$程度になる$|T|$のリスト}
    \label{table:student_t_confidence}
    \centering
    \begin{tabular}{cccc}
    %\hline
    n & p=0.1 & $p=0.05$ & $p=0.025$   \\
    \hline \hline
    1 & 6.31 & 12.70 & 25.45 \\
    5 & 2.01 &2.57  & 3.16\\
    10 & 1.81 &  2.22& 2.63 \\
      \hline
    \end{tabular}
  \end{table}
% https://bellcurve.jp/statistics/course/8968.html


\section{統計分布の関係}
同一の確率分布からサンプリングされた複数の確率変数$X_1,X_2,\cdots,X_n$を得たとき、それを要約した要約統計量がどのような分布関数に従うのかを考察する。
% $(\star)$のついた項目は、科学的(恣意的)な判断を含んでいる。

\subsection{正規分布の再生性}
$X \sim N(\mu_1,\sigma^2_1),Y\sim(\mu_2,\sigma^2_1)$とするとき、$aX+bY \sim N(a\mu_1+b\mu_2,a^2\sigma^2_1+b^2\sigma^2_2)$より、$a=\frac{1}{2},b=\frac{1}{2}$。すると、$\frac{X}{2}+\frac{Y}{2}\sim N(\frac{\mu_1+\mu_2}{2},\frac{\sigma^2_1}{2^2}+\frac{\sigma^2_2}{2^2})$である。$\mu_1=\mu_2,\sigma_1=\sigma_2$とすると、$\frac{X+Y}{2}\sim N(\mu_1,\frac{\sigma^2_1}{2})$が成り立つ。
このことを利用すると、$X_1,X_2,\cdots,X_n\sim N(\mu,\sigma^2)$とすると、$\bar{X}=\frac{X_1+X_2+\cdots+X_n}{n}\sim N(\mu,\frac{\sigma^2}{n})$である。よって$\frac{\bar{X}-\mu}{\sqrt{\frac{\sigma^2}{n}}}\sim N(0,1) $。また、$\bar{x}$の出現しやすい区間は、
\begin{equation*}
    -z_{0.025}<\frac{\bar{X}-\mu}{\sqrt{\frac{\sigma^2}{n}}} < z_{0.025}
\end{equation*}
である。式を変形すると、
\begin{equation*}
    \mu-z_{0.025}\frac{\sigma^2}{n}<\bar{x}<\mu+z_{0.025}\frac{\sigma^2}{n}
\end{equation*}
がわかる。
以上をまとめておく。

\begin{theo}
    $X_1,X_2,\cdots,X_n \sim N(\mu,\sigma^2)$とすると、$\frac{\bar{X}-\mu}{\sqrt{\frac{\sigma^2}{n}}}\sim N(0,1)$ ただし、$\bar{X}=\frac{X_1+X_2+\cdots+X_n}{n}$。また、$\bar{X}$の出現しやすい区間は、$\mu-z_{0.025}\frac{\sigma^2}{n}<\bar{x}<\mu+z_{0.025}\frac{\sigma^2}{n}$である。
\end{theo}



\begin{theo}
    $X_1,X_2,\cdots,X_{n_1} \sim N(\mu_1,\sigma_1^2),Y_1,Y_2,\cdots,Y_{n_2}\sim N(\mu_2,\sigma_2^2)$ただし、$\mu_1\neq \mu_2,\sigma_1\neq \sigma_2$とする。正規分布の再生性により、$\bar{X}\sim N(\mu_1,\frac{\sigma^2_1}{n_1}),\bar{Y}\sim N(\mu_2,\frac{\sigma^2_2}{n_2})$である。次が成り立つ。
    $\bar{X}-\bar{Y} \sim N(\mu_1-\mu_2,\frac{\sigma^2_1}{n_1}+\frac{\sigma^2_2}{n_2})$であり、
    \begin{equation*}
        \frac{(\bar{X}-\bar{Y})-(\mu_1-\mu_2)}{\sqrt{\frac{\sigma_1^2}{n_1}+ \frac{\sigma_2^2}{n_2}}}\sim N(0,1).
    \end{equation*}
\end{theo}



\subsection{指数分布の再生性}
指数分布$Exp(\lambda)$と、ガンマ分布$Ga(1,\frac{1}{\lambda})$は、同一の密度分布関数であり、それは$f(x) = \frac{1}{\lambda} \exp(-\frac{x}{\lambda})$である。ガンマ分布には、分布の再生性があり、$X\sim Ga(a_1,b),Y\sim Ga(a_2,b)$であるなら、$X+Y \sim Ga(a_1+a_2,b)$である。このことを、$n$個の確率変数$X_1,X_2,\cdots X_n \sim Exp(\lambda)(=Ga(1,\frac{1}{\lambda}) )$に適用すると、$X_1+X_2+\cdots+X_n \sim Ga(n,\frac{1}{\lambda})$である。以上によって、$n\bar{X}\sim Ga(n,\frac{1}{\lambda})$ただし、$\bar{X}=X_1+X_2+\cdots+X_n$である。
再生性については、確率母関数を利用することで証明できる。

\begin{theo}
    $X_1,X_2,\cdots,X_n \sim Ga(1,\frac{1}{\lambda})$ならば、
    $n\bar{X}\sim Ga(n,\frac{1}{\lambda})$
\end{theo}

\begin{proof}
    $Ga(1,\frac{1}{\lambda})$の確率母関数は、$M_X(t)=(1-\frac{1}{\lambda}t)^{-1}$である。確率変数$X_1+X_2+\cdots+X_n$の確率母関数は
    \begin{eqnarray}
        M_{n\bar{X}} = M_{X_1+X_2+\cdots+X_n} &=& M_{X_1}M_{X_2}\cdots M_{X_n} \\
        &=& (1-\frac{1}{\lambda}t)^{-1}(1-\frac{1}{\lambda}t)^{-1}\cdots(1-\frac{1}{\lambda}t)^{-1}\\
        &=& (1-\frac{1}{\lambda}t)^{-n}
    \end{eqnarray}
    以上より、$n\bar{x}\sim Ga(n,\frac{1}{\lambda})$である。
\end{proof}

\if 0
\subsubsection{対数正規分布の再生性}
$X\sim \Lambda(\mu_1,\sigma_1^2), Y\sim \Lambda(\mu_2,\sigma^2_2)$とするとき、$XY\sim\Lambda(\mu_1+\mu_2,\sigma_1^2+\sigma_2^2)$である。
これを使えば、$X_1,X_2,\cdots,X_n \sim \Lambda(\mu,\sigma^2)$について、その積$X_1,X_2\cdots X_n \sim \Lambda(n\mu,n\sigma^2)$である。

ここで、$X_1X_2\cdots X_n$は十分統計量にならないので、検定が作れないのか?
一方で、$\log X_1+\log X_2 \cdots \log X_n $は十分統計量$N(\mu,\sigma^2)$に従う。
$\log X_1^2+\log X_2^2 \cdots \log X_n $は十分統計量$\chi^2$分布に従う
% https://stats.stackexchange.com/questions/202890/jointly-sufficient-statistic-question
% https://mcm-www.jwu.ac.jp/~konno/pdf/statr28.pdf
% https://www.stats.ox.ac.uk/~reinert/stattheory/solutions109.pdf
% https://pages.stern.nyu.edu/~wgreene/MathStat/old-exam-1.pdf
% https://math.stackexchange.com/questions/3597198/sketching-power-function-for-a-log-normal-density
% https://stats.stackexchange.com/questions/402522/likelihood-ratio-test-of-log-normal-distribution
% https://abicky.net/2014/03/03/202054/
\fi

\subsection{正規分布とt分布の関係}
$X_1,X_2,\cdots,X_n \sim N(\mu,\sigma^2)$とする。統計量$T$を、
\begin{equation*}
    T = \frac{\bar{X}-\mu}{\sqrt{\frac{S^2}{\sqrt{n}}}}.
\end{equation*}
ここで、$\bar{X}=\frac{X_1+X_2+\cdots+X_n}{n}$、$S^2=\frac{1}{n-1}\sum_{i=1}^{n}(X_i-\bar{X})^2$である。
この統計量$T$は、$t(n-1)$分布に従うことが知られている。統計量$T$の中に母数$\sigma$が入っていないので、$\sigma$わからないときでも、$T$を計算すれば、それが$t(n-1)$に従うことがわかる。

2つの正規分布$X_1,X_2,\cdots,X_{n_1} \sim N(\mu_1,\sigma_1^2), Y_1,Y_2,\cdots,Y_{n_2}\sim N(\mu_2,\sigma_1^2)$とする。このとき、
\begin{equation*}
    T = \frac{(\bar{X}-\bar{Y})-(\mu_1-\mu_2)}{\sqrt{\frac{U^2}{n_1}+\frac{U^2}{n_2}}}
\end{equation*}
は、$n_1+n_2-2$の$t$分布に従う。ここで、$U$は、
\begin{equation*}
    U^2 = \frac{(n-1)U_1^2+(n_2-1)U_2^2}{n_1-1+n_2-1}
\end{equation*}
であり、$U_1,U_2$は、不偏分散
\begin{eqnarray*}
    U_1^2 = \frac{1}{n_1-1}\sum_{i=1}^{n_1}(X_i-\bar{X})\\
    U_2^2 = \frac{1}{n_2-1}\sum_{i=1}^{n_2}(Y_i-\bar{Y})\\
\end{eqnarray*}
である。

\section{問題意識}
確率変数$x_1$または、$x_1,x_2,\cdots,x_n$を得たとき、それらが独立同分布に従うという前提のもと、ある母数をもつ分布関数に従う、または従わないと推測することは可能であるだろうか。
最尤推定から、確率変数を得たなら、最尤推定を行って、母数を推測可能な場合がある。
具体的には、正規分布から得られた確率変数については、その平均と分散は、$(\mu,\sigma^2)=(\bar{x},\sum_{i=1}^{n} (x_i-\bar{x})^2/n)$である。

この問題に対して、
正規分布から確率変数を得たとき、ある母数平均$\mu$をもつ正規分布からサンプリングされていないということはできるだろうか。これを議論する。

%\subsection{言葉の準備}


\section{ひとつの確率変数から推測する}
ひとつの確率変数$x$が$N(\mu,\sigma^2)$に従わないことを推定したい。
$N(\mu,\sigma^2)$から得られる確率変数は、$95\%$の確率で、$\mu-\sigma z_{0.025}\sim \mu-\sigma z_{0.025}$の間で見つかる。
$\sigma^2=1$とすると、$95\%$の確率で、確率変数は、$\mu-1.96\sim\mu+1.96$の間で見つかる。
$\mu=0$なら、$x=0.1$は、この区間の中にあるので、$x$は、$N(0,1)$では良く見つかる値になる。
%この基準では、複数の$\mu$で$x=0.1$はよく見つかる範囲に入る。例えば、$\mu=0.1$でも良く確率変数が見つかる区間は、$-1.85\sim 2.05$なので、確率変数は、$N(0.1,1)$に従うとしても問題ない。
$x=0.1$がその区間に入らない母数は、$\mu=z_{0.025}+0.1$のときで、区間は$0.10003\sim4.01$で、この区間に$x$は入ってません。
これを言い換えれば、母数$x-z_{0.025}\leq\mu\leq x+z_{0.025}$の間でよくある値になり、この間から外れた母数をもつ$N(\mu,1)$に従っていいないと推測できる。

我々の科学では、このようなたったひとつの値から、分布関数の母数を推測することはしません。
例えば、$x=1.97$が得られたとすると、$N(0,1)$のよく出る値の範囲は、$-1.96\sim1.96$であることから、母数$0$ではないと判断できます。
一方で、$N(0,1)$で$x=1.97$は、サンプルサイズが$20$であれば、そのうち$1$回は、$1.97$をとる値です。もしももう一度サンプリングできたとして、その値が$0$になることもあり得ます。
以上のことから、$1$回のサンプリングだけで判断しません。

\section{複数の確率変数から推測する}
複数の確率変数$x_1,x_2,\cdots,x_n$が$N(\mu,\sigma^2)$に従わないことを推定したい。

サンプルサイズが大きいときは、最尤推定を行い、$\mu_1=\bar{x},\sigma_1^2=\sum_{i=1}^{n} (x_i-\bar{x})^2/n$を計算し、その値が、$\mu,\sigma^2$と著しく異なっていれば、$N(\mu,\sigma^2)$に従わないと言えそうである。
例えば、図\ref{fig:maximum_likelihood_0}は、正規分布$N(170,6.8^2)$からサンプルサイズ$100$の標本を得たとき、図\ref{fig:maximum_likelihood_1}は、サンプルサイズ$30$、そしてそのデータの分布と、最尤推定量から予測されるも確率密度関数、$N(168,6.8^2),N(171,6.8^2)$を表している。図\ref{fig:maximum_likelihood_0}をみると、最尤推定した確率密度関数がデータの出現頻度をよく表していること、そして、周辺の二つの正規分布$N(168,6.8^2),N(171,6.8^2)$とデータが乖離していることがわかる。
一方、図\ref{fig:maximum_likelihood_1}では、データが$N(171,6.8^2)$に従っているのではないかという疑惑が残る。

\begin{figure}
    \begin{center}
        \includegraphics[width=15cm]{./image/02_/maximum_likelihood_0.pdf}
        \caption{$N(170,6.8^2)$からサンプルサイズ$100$の標本を得たときの分布。その最尤推定量により求められる分布関数。$N(168,6.8^2),N(171,6.8^2)$の分布関数を示す}
        \label{fig:maximum_likelihood_0}
    \end{center}
\end{figure}
\begin{figure}
    \begin{center}
        \includegraphics[width=15cm]{./image/02_/maximum_likelihood_1.pdf}
        \caption{$N(170,6.8^2)$からサンプルサイズ$30$の標本を得たときの分布。その最尤推定量により求められる分布関数。$N(168,6.8^2),N(171,6.8^2)$の分布関数を示す}
        \label{fig:maximum_likelihood_1}
    \end{center}
\end{figure}

このように、サンプルサイズが大きいと、最尤推定により推測した確率密度関数を見れば、そのほかの母数に従わないことがわかる場合がある。
また、より近くにある母数の確率密度関数と区別することは難しいこともわかる。

\begin{figure}
    \begin{center}
        \includegraphics[width=15cm]{./image/02_/maximum_likelihood_3.pdf}
        \caption{$N(170,6.8^2)$からサンプルサイズ$3$の標本を得たときの分布。その最尤推定量により求められる分布関数。$N(168,6.8^2),N(171,6.8^2)$の分布関数を示す}
        \label{fig:maximum_likelihood_0}
    \end{center}
\end{figure}

\section{全然違うはなんとなくわかる}
図\ref{fig:maximum_likelihood_false_3},\ref{fig:maximum_likelihood_false_30}は、$N(170,6.8^2)$からサンプリングしたデータの度数分布と、その確率密度関数とは著しく異なる確率密度関数を表示したものです。

\begin{figure}
    \begin{center}
        \includegraphics[width=15cm]{./image/02_/maximum_likelihood_false_3.pdf}
        \caption{$N(170,6.8^2)$からサンプルサイズ$3$の標本を得たときの分布。その最尤推定量により求められる分布関数。$N(168,6.8^2),N(171,6.8^2)$の分布関数を示す}
        \label{fig:maximum_likelihood_false_3}
    \end{center}
\end{figure}

\begin{figure}
    \begin{center}
        \includegraphics[width=15cm]{./image/02_/maximum_likelihood_false_30.pdf}
        \caption{$N(170,6.8^2)$からサンプルサイズ$3$の標本を得たときの分布。その最尤推定量により求められる分布関数。$N(168,6.8^2),N(171,6.8^2)$の分布関数を示す}
        \label{fig:maximum_likelihood_false_30}
    \end{center}
\end{figure}

\clearpage
\section{再生性}
\subsubsection{$(\star)$ $N(\mu,\sigma^2)$に従う確率変数であることを判定できるか}
$N(0,1)$に従う確率$x_1,x_2,\cdots,x_n$から計算した統計量、$z=\frac{\bar{X}-0}{\sqrt{\frac{1}{n}}}$は、$N(0,1)$に従い、$z$が$95\%$の確率で見つかる範囲は$[-1.96,1.96]$である。
同様に、$y_1,y_2,\cdots,y_n \sim N(1.96,1)$であるならば、$z=\frac{\bar{Y}-1.96}{\sqrt{\frac{1}{n}}}$は、$N(0,1)$に従う。

確率変数から、特定の母数を持つ正規分布に従わないことを示すことはできるだろうか。
具体的な問題設定として、
$y_1,y_2,\cdots,y_n$を正規分布に従う確率変数とする。そのとき、$y_1,y_2,\cdots,y_n$が$N(\mu,\sigma^2)$に従わないことを判断する良い方法はどのようなものだろうか。

ここで、$y_1,y_2,\cdots,y_m \sim N(1.96,1)$にもかかわらず、$N(0,1)$に従うと推測した場合、$z=\frac{\bar{Y}-0}{\frac{1}{\sqrt{n}}} \sim N(0,1)$であると考えられる。
$z$の分子の$\mu$が$0$になっていることに注意が必要である。
実際に、$y_1,y_2,\cdots y_{100}$を$N(1.96,1)$からサンプリングした標本を$100$個作ってみると、およそ$19$を中心に分布することがわかる。
このことは、$y_1,y_2,\cdots,y_m\sim N(0,1)$であるならば、$z$は、$[-1.96,1.96]$の間で$95\%$の確率で入るので、この推測が間違いであることが推測される。
以上の考察から、$y_1,y_2\cdots,y_n\sim N(0,1)$ではないと判断する。

\begin{figure}
    \begin{center}
        \includegraphics[width=15cm]{./image/02_/normal_distribution_test.pdf}
        \caption{(a)$N(1.96)$に従う確率変数を100個サンプリングし、その標本を1000個集めたときの$z=\sqrt{100}(\bar{X}-0)$のヒストグラム (b)$N(0,1)$に従う確率変数を100個サンプリングし、その標本を1000個集めたときの$z=\sqrt{100}(\bar{X}-0)$値のヒストグラム}
    \end{center}
\end{figure}


もう一つ例を挙げる。
$y_1,y_2,\cdots,y_n \sim N(170,5.8)$とする。このとき、この標本が$N(168,5.8)$によりサンプリングされたものではなくことを示すことはできるだろうか。
$z=\sqrt{n}\frac{\bar{y}-168}{\sigma}$を計算すればよい。
図には、$N(170,5.8)$に従う確率変数を100個サンプリングし、その標本を1000個集め、ヒストグラムを描いた。
これをみると、$0.5$を中心に分布が広がることがわかる。$z=\frac{\bar{X}-168}{\sqrt{\frac{5.8}{n}}}\sim N(0,1)$であるはずである。
複数回、標本を得た場合でも、$z$が$[-1.96,1.96]$の範囲に収まっている。このことは、$N(168,5.8)$ではないと判断できないことを示唆している。


\begin{figure}
    \begin{center}
        \includegraphics[width=15cm]{./image/02_/normal_distribution_test2.pdf}
        \caption{$N(170,5.8)$に従う確率変数を100個サンプリングし、その標本を1000個集めたときの$z=\sqrt{100}(\bar{X}-168)$のヒストグラム}
    \end{center}
\end{figure}


ある正規分布に従う確率変数$x_1,x_2,\cdots,x_n$が母数の異なる正規分布で得られる確率も計算できる。具体的には、$x_1,x_2,\cdots,x_n\sim N(\mu,\sigma^2)$とし、これが$N(\mu_1,\sigma_1^2)$で得られるとすると、そのときの統計量は、$z=\frac{\bar{x}-\mu_1}{\frac{\sigma_1}{n}}$である。この$z$は、$N(0,1)$に従うと考えられるので、$\phi(|z|>Z)$となる確率を計算すれば良い。

\begin{theo}
    確率変数$x_1,x_2,\cdots,x_n \sim N(\mu,\sigma^2)$ならば、$z=\frac{\bar{X}-\mu}{\sqrt{\frac{\sigma}{n}}} \sim N(0,1)$である。
    一方で、確率変数$x_1,x_2,\cdots,x_n \sim N(\mu,\sigma^2)$とする。$N(\mu_1,\sigma_1^2)$は正規分布とする。ただし、$\mu\neq \mu_1, \sigma =\sigma_1$このとき、$z=\frac{\bar{X}-\mu_1}{\sqrt{\frac{\sigma_1}{n}}} \sim N(0,1)$ではない。
\end{theo}
$\mu$と$ \mu_1$が極めて近い値のとき、$z=\frac{\bar{X}-\mu_1}{\sqrt{\frac{\sigma_1}{n}}} $も$N(0,1)$におけるよくある値になる言い換えれば、$\phi(|z|>Z)$は十分大きい。
一方で、$\mu$と$ \mu_1$が離れた値を取ると、$\phi(|z|>Z)$は小さな値になる。


\subsection{$(\star)$ $Exp(\lambda)$に従う確率変数であることを判定できるか}
$x_1,x_2,\cdots,x_n \sim Exp(\lambda)$であるとき、$n\bar{x}\sim Ga(n,\frac{1}{\lambda})$である。
母数不明の指数分布に従う確率変数が、$x_1,x_2,\cdots,x_n \sim Exp(\lambda)$と仮定したとき、$n\bar{x}\sim Ga(n,\frac{1}{\lambda})$でないならば、$x_1,x_2,\cdots,x_n \sim Exp(\lambda)$ではないと判断できるだろうか。シミュレーションによって確認してみよう。

この論法は、母数が不明の指数分布に従う確率変数を得たとき、その指数分布の母数が特定の値ではないことを示すためにこの論法を利用する。ここでは、母数が$\lambda=1,2,5,10,100$からサンプルサイズ4の標本を$1000$生成し、それら標本の統計量$n\bar{X}$のヒストグラムと、ガンマ関数$Ga(100,1)$の確率密度関数を比較する。

\begin{figure}
    \centering
    \includegraphics[width=15cm]{./image/02_/Exp_Gamma_simulation.pdf}
    \caption{(a)$Ga(10,1)$の確率密度関数。(b-e)指数分布からサンプルサイズ$4$の標本を$1000$回生成し、その統計量$n\bar{x}$のヒストグラム}
    \label{fig:exp_gamma_simulation}
\end{figure}

図\ref{fig:exp_gamma_simulation}(a)は、指数分布$Exp(\lambda=1)$の確率密度関数を示している。
図\ref{fig:exp_gamma_simulation}b-eは、シミュレーションの結果を示している。
図\ref{fig:exp_gamma_simulation}(b)には、指数分布$Exp(1)$に従う確率変数の統計量$n\bar{x}$が確かに、$Ga(100,1)$に従うことが確かめれる。
図\ref{fig:exp_gamma_simulation}(c-e)では、指数分布の$\lambda$が$1/2,1/5,1/10$のときの統計量のヒストグラムである。これらと、図\ref{fig:exp_gamma_simulation}(a)を比較すると、分布が異なっているので、確かに、$Ga(100,1)$には従わないことがわかる。



\subsection{問題点}
aa


\chapter{身長を予測する統計モデル}
\section{正規分布を組み入れた統計モデル}
日本人の$17$歳男性の身長を予測する統計モデルを構築する。この統計モデルは次の1-3から構成される。
\begin{quote}
    \begin{enumerate}[(1)]
    \item 独立同分布
    \item その分布は、正規分布
    \item 正規分布の母数(平均と分散)はそれぞれ$\mu,\sigma^2=5.7$。
    \end{enumerate}
\end{quote}

$\mu$を変数としたこの統計モデルを$M(\mu)$とする。
%数理統計学の知識を使うには、少なくとも$3$つので一つの統計モデルであると私は考えている。
およその平均値は日本にいれば母集団の分布をなんとなく知っているので、$\mu=171.1\mathrm{cm}$であると推測できる。
母集団のばらつき具合を意識することが少ないので、分散の値を決定することは難しい。
今回は、カンで$5.7$としました\footnote{統計データを覗き見した。分散を経験で推定できる人は少ないはずです。}。


\begin{SMbox}{なぜ正規分布を仮定できるのか}
  %\blindtext[5]
  数理統計学の本には、正規分布を前提にして書かれていることが多々あることから、科学において統計を利用するには、その前提が満たされる必要があるという考えがある。私も以前はそのように考えており、同様の考えにハマってしまう人は少なくない。

  \begin{rightbubbles}{bubblegreen}{Katsushi Kagaya}{./image/Twitter_logo_EPS/2021_Twitter_logo_blue.eps}
  学生のころ先生とデータについて議論していて(生物学分野です)「そもそもなぜ正規分布が仮定できるのか…」とおっしゃって二人でしばらく固まったことを思い出します。実現可能性の考え方から学ぶのが良いのかなと思います
      \begin{flushright} 
          \small	\url{https://twitter.com/katzkagaya/status/1209656621523058691}
      \end{flushright}    
  \end{rightbubbles}

  学問の世界において、分布関数に関する仮定が可能な理由についての認識は様々である。数学においては、仮定をして結論を導くことはよくある。数学から離れた科学の領域では、仮定することに対して妥当性や客観的であること要求していることもある。本書では、恣意的に考えたモデルを使って推測をしてみるという考えに基づいて、統計モデルを構築し、現象について推測を行う。
\end{SMbox}


\section{統計モデルによる推測}
$\mu=171.1$としたときの統計モデル$M(171.1)$を使って、身長に関する推測を行う。

\subsection{ $\circ\circ \mathrm{cm}$以下、$\diamond\diamond \mathrm{cm}$以上の人の割合}
まず、母集団に$180cm$以下、$180cm$以上の人の割合を推測する。正規分布関数を使い、$P(x>180)$を計算する。

\begin{lstlisting}
norm.cdf(180,171.1,5.7)
1-norm.cdf(180,171.1,5.7)
\end{lstlisting}
結果、 $P(x<180)=0.940$より、$P(x>180)=0.059$ということが分かります。
このことから、母集団から100人無作為抽出を行うと内$5-6$人程度は$180cm$以上であることが推測できる。

もう一つ、$160cm$以下の人割合を推測する。

\begin{lstlisting}
norm.cdf(160,171.1,5.7)
1-norm.cdf(160,171.1,5.7)
\end{lstlisting}
結果、$P(x<160)=0.059$、$P(x>160)=0.940$と推測できる。

$P(x<160)$と$P(x>180)$が極めて近い値でるのは、利用した正規分布は、母平均$\mu=171.1$を中心に、対称に分布する関数なので、$171.1$からおよそ$10cm$離れた$160cm$以下の人と$180cm$以上の人ではおよそ同じくらいの割合でいると推定される。
    
\subsection{擬似的に無作為を行うサンプリング}
$10$人分のデータをサンプリングしてみると、以下の数値が得られる。
$10$人を母集団から無作為抽出すると、およそこのようなデータが得られることがあると推測できる。

\begin{lstlisting}
168.575192 164.5988088 162.7027275 163.9689649 169.8187076 174.8851702 172.767133 165.0665034 175.7370453 163.0385381
\end{lstlisting}



\section{統計モデルの比較 1}
統計モデル$M(171.1)$による推測と実データを比較し、モデルがデータを推測できていることを確認する。
17歳男性の身長を無作為抽出して標本を得るには時間とお金がかかるので、公開されているデータ\footnote{ \url{https://www.e-stat.go.jp/dbview?sid=0003107092} }\footnote{\url{https://www.e-stat.go.jp/dbview?sid=0003037791}}を使う。
このデータは文部科学大臣があらかじめ指定した1410校の高校に在籍する生徒を対象にした標本である。

\begin{figure}
\begin{center}
    \includegraphics[width=15cm]{./image/03_/cm_data.pdf}
    \caption{17歳の男性から無作為抽出したデータ。上は、データと統計モデル$M(170)$の度数。下は、データと統計モデル$M(170)$の累積相対度数}
    \label{fig:real_height_men}
\end{center}
\end{figure}


\begin{SMbox}{$170cm$を少し超えた人が多いのは、不正(無作為抽出の手順に異常)があったから?}
\begin{quotation}
    「生物学上、グラフは曲線になっていなければならないが、169cmの部分はへこんでいる。これは先生や生徒による四捨五入で生まれるサバ読みの結果。身長が170cmなのか169cmなのかで気持ち的に違ってきますからね」と話すと、食料自給率や犯罪発生件数とは異なる微笑ましいサバ読みのトリックに、出演者一同、笑みを浮かべていた。\footnote{国民を欺く“統計のウソ” 知らないと怖い“統計トリック”を専門家が解説
    \url{https://times.abema.tv/articles/-/5640846} 2022/04/30確認}
\end{quotation}

このように、データが統計モデルに一致しないことから、データに不正な操作が加わっているという推測がされることがある。議論となっている身長のデータを観察してみる。
図\ref{fig:real_height_men}上を見ると、確かに、$170$を超えたあたりの度数は、$169$の度数よりも多い。
また、$170cm$以下のデータは統計モデルの度数よりも低く、$170cm$以上のデータは統計モデルの度数よりも大きい。一方で、図\ref{fig:real_height_men}下の累積相対度数を見ると、度数と同様の変異は少ないように見える。このようなデータと統計モデルの相違の原因は、不正な計測により生じたと断言できるのだろうか。

データとモデルの相違が生じる原因が、不正な計測だけではないことを確認する。
具体的には、データを統計モデルからサンプリングし、そのデータが統計モデルと一致するかを観察してみる(図\ref{fig:simulation_height_men})。図を見るとわかるように、サンプリングを行った場合、$168cm$付近で、度数が曲線よりも上にくる部分がある。また、$170cm$より小さいところでは、統計モデルよりもデータの度数が上にあり、$170cm$より大きなところでは、統計モデルより、データの度数が下にある。このように、統計モデルによりサンプリングし、統計モデルとサンプリングデータを比較した場合でも、ズレが生じる。これは、不正なモデルの予測とデータの間のズレが計測以外から生じることを示唆している。
%単純なズレだけを見たとしても、それが不正なのかは結論をつけることは難しい。


不正を見つけるには、次の経験が必要である。恣意的な操作を一切介入させない、かつ、無作為にデータを取得する条件のもと、得られたデータ、と同じ計測方法・同じ生徒において、教員が計測したデータこの二つのデータが一致しないならば不正な操作が加わったことが疑える。

データ解析をするには、常に、データを収集する手順が守られていないことを疑うことをするべきである。
例えば、髪の毛や靴などを履いる人がそうではない人と同じように計測をされると、平均値が大きくなる。身長の低い生徒に対してその傾向が高ければデータには歪みが生じやすくなる。計測を行なった先生方の疲れなども考慮すれば、データ収集の手順の誤りにより、データが偏ることもある。

データの収集には多大な労力がかかっている。誰かがどこかで腰を痛めながら高校生の身長を測る仕事をしていることは心に留めておくべきで、不正があったと主張するのは、彼らの仕事を低く評価しすぎではないだろうか。おそらく先生たちは、正確に計測できるように正確に手順を満たすように計測しているはずである。
不正を疑うならば、それなりに確証できる証拠を提示すべきである。具体的には、自分が手順を守って計測したデータと、先生が測ったときのデータにおいて、それらの間の差を示すべきである。



もう一つこの論者と私とで異なる点は、生物学データのグラフが曲線になるべきという点である。私は、推論のために統計モデルを利用しているので、統計モデルとデータが一致しない場合でも、推測に利用できると考え、統計モデルを利用する。一方で、この論者は、統計モデルとデータが一致すべきと考えている。言い換えれば、データが統計モデルに従うことを前提にする立場と、データを推論するために統計モデルを仮定すると言う立場がある。

\if 0 
$180cm$以上の割合についてはデータと一致していますが、$160cm$以下は、データと不一致です。この統計モデルで推測できていると考えても良いのでしょうか。
無作為抽出したときに得られるデータをできます。
ここで、$\mu=169.1$の統計モデル$M(\mu=169.1)$と、$\mu=180$の統計モデル$M(180)$を
標本から無作為抽出を行い、集計すると平均$169.1cm$程度であることがわかったとします。このとき統計モデル$M(169.1)$の推測は母集団の特徴をよく捉えているだろうか?
\fi 
\end{SMbox}

\begin{figure}
    \begin{center}
        \includegraphics[width=15cm]{./image/03_/cm_data_simulation.pdf}
        \caption{上:正規分布を含む統計モデル$M(170)$によりサンプリングされたDataの頻度と、統計モデルの頻度。下:上と同じデータ・統計モデルの累積相対頻度}
        \label{fig:simulation_height_men}
    \end{center}
\end{figure}

\begin{SMbox}{軽いパンばかり買わされる}
ある国では、ある時期、パンを作るための道具、手順、材料が政府からパン屋に配布され、パン屋がパンを作ることになっていた。パンを焼くための型は、完成時に$1000g$になるように設計されており、手順を厳密に守り作ったパンは確かにおよそ$1000g$になっていた。どの季節に作っても手順を守りさえすれば、$1000g$になったのだ。
この材料、道具をパン屋が利用し、手順にそってパンを作れば、やはりパンはおよそ$1000g$になるはずである。

その国では、小麦の値段が高騰しており、支給された小麦をそのまま売った方が儲かるという状況になっていた。そんなとき、パンが$1000g$よりも軽いと感じた数学者が、数ヶ月にわたりパンの重量を計測していった。その結果、パンの重量は平均で$950g$となっており、本来の$1000g$よりも、軽いことがわかった。

このとき、パン屋が不正をしていると主張できる。
手順を踏めば平均で$1000g$になるパンが平均およそ$950$になったのは、パン屋が手順通りにパンを作っていないことを疑える。手順を守って作れば$1000g$になるという経験(データ)があるから疑うことができる。
%もしも、季節によってパンの重さが変化するものだったとするなら、$4$月には$1000g$だったものが$6$月には、$900g$になる可能性を排除できない。
\end{SMbox}


\subsection{サンプルサイズが大きい場合}

データと統計モデルを比較する。$180cm$以上の割合は、0.0642であり、モデル$M(171.1)$の推測値$P(x>180)=0.059$と数値が近い。
また$160cm$以下の割合は、$0.023$程度であり、統計モデルの推測値$P(x<160)=0.025$であり、やはり数値が近い。
%このように数学的フィクションである統計モデルを使うことで、現実に関する推測が可能になった。

ここまでは、$M(171.1)$を用いて、母集団を推測した。統計モデル$M(170)$の代わりに$M(168)$により推測を行うとデータとの一致具合を確かめる。$180cm$以上の人を推測すると$M(168)$では$P(x>180)=0.03$であり、統計モデル$M(171.1)$の推測$P(x>180)=0.059$よりもさらに実際の計測値$0.0642$と乖離している。
これは、$M(168)$では、ピークが平均値の$168$に移動するので、$180cm$を超える割合がさらに低くなるので、実際の数値から離れる。

一方で、$160$以下の人では、$M(168)$では、$P(x<160)=0.08$程であり、$M(171.1)$の推測値$P(x<160)=0.025$よりも、実際の数値$0.023$から離れている.
これも、$M(168)$では、ピークが$170$よりも小さな値になるので、$160cm$より小さい人の割合が大きくなるので、予測と実際のデータの不一致度が大きくなる(表\ref{table:data_type}にまとめておいた)。
このように、統計モデルの母数に応じて、現実の予測精度が変化する。

\begin{table}[hbtp]
    \caption{統計モデルとデータの比較}
    \label{table:data_type}
    \centering
    \begin{tabular}{lcc}
    %\hline
    統計モデル  & $P(x<160)$  & $P(x>180)$   \\
    \hline \hline
    データ &  0.023 &  0.0642\\
      %\hline \hline
    M(171.1) & 0.025 & 0.059  \\
    M(168) &  0.08 & 0.03 \\
      \hline
    \end{tabular}
  \end{table}


この統計モデルの予測の良さが分かったのは、無作為抽出を繰り返して、サンプルサイズを大きくしたときのデータの分布を得ていることによって、
そのデータとモデルとを比較をすることで、$M(171.1)$が$M(168)$より良い統計モデルであることを判別できた。

では、データが十分でない場合においても、推測とデータの一致を基準にして、より良い統計モデルを選ぶことはできるのでしょうか?

\subsection{サンプルサイズが小さい場合}
%\subsection{推測値とデータの比較}
母集団のことをほとんど知らない場合において、統計モデルとデータの比較はできるが、これを元に統計モデルが良いことを検討できない。
\if 0
母集団に関して次のことを知っていることにします。
\begin{itemize}
    \item 平均がおよそ$170cm$
    \item $160cm$の人や$180cm$の人と出会う確率は同じくらい($170cm$を中心に対象に分布している)
    \item 分散は5.7
\end{itemize}
\fi
サンプルサイズ10の標本が二つ得られたとします(実際には、コンピュータを使って正規分布からサンプリングした。このデータは母集団から無作為抽出したと考える)。標本は、次の通り。

\begin{lstlisting}
sample1 = [162.56944902, 178.42128764, 171.15286336, 172.2581195 , 160.21499345, 175.35072013, 173.17952774, 173.73301156, 179.52758126, 178.35924221]
\end{lstlisting}

\begin{table}[hbtp]
  \caption{統計モデルと小さいサンプルサイズの標本}
  \label{table:smalle_sample_size}
  \centering
  \begin{tabular}{lccc}
  %\hline
  統計モデル  & $P(x<160)$  & $P(x>180)$  & $\bar{X}$ \\
  \hline \hline
  標本1 &  0 &  0 & 172.8 \\
    %\hline \hline
  M(171.1) & 0.025 & 0.059  & 171.1 \\
  M(168) &  0.08 & 0.03 & 168\\
    \hline
  \end{tabular}
\end{table}
$180cm$以上の人は、$0$人、$160cm$以下の人も$0$人、どちらの統計モデルでも推測と一致しているかを推測できない[表\ref{table:smalle_sample_size}]。
標本平均$\bar{X}=172.8$であり、$M(170)$の母数$170$が$M(168)$の母数平均$168cm$でどちらも同じ程度の差である。
サンプルサイズが小さいときには、統計モデルの予測とデータを比較できないことがあるので、予測精度の良いモデルがどれかを決定できないことがある。


\if 0
sample2 = [164.04222157, 162.19052559, 172.03420244, 168.03580415, 176.73750537, 166.41177205, 165.27050656, 168.02537023, 176.18720054, 171.78005419]
\end{lstlisting}


\begin{table}[hbtp]
    \caption{統計モデルと小さいサンプルサイズの標本}
    \label{table:smalle_sample_size}
    \centering
    \begin{tabular}{lccc}
    %\hline
    統計モデル  & $P(x<160)$  & $P(x>180)$  & $\bar{X}$ \\
    \hline \hline
    標本1 &  0 &  0 & 172.8 \\
    標本2 &  0 &  0 & 169 \\
      %\hline \hline
    M(171.1) & 0.025 & 0.059  & 171.1 \\
    M(168) &  0.08 & 0.03 & 168\\
      \hline
    \end{tabular}
  \end{table}
どちらの標本でも

\fi
% また、標本の平均値と統計モデルの平均値でも標本が
% 仮説検定の枠組みでは、絶対にだめな統計モデルを明らかにします。

%\subsubsection{標本内の偏った値に注目}

\section{統計モデルの比較 2}



\section{統計モデルの性質を使った方法}
ここまでは、統計モデルの予測がデータと一致するかことを定量的に評価した。ここでは、推測に適していないと判断する方法である、統計的仮説検定を紹介する。
この方法は、統計量の一つである統計検定量の統計モデル上での出現しやすさにより、モデルを評価する。
今回考えている統計モデル$M(\mu)$では、次の統計量$Z$が標準正規分布$N(0,1)$に従うことが、正規分布の再生性によってわかっている。
\if 0
 川久保統計学P.166
 \fi
$$
Z(\bar{X},\mu)=\frac{\sqrt{n}(\bar{X}-\mu)}{\sigma} \sim N(0,1)
$$
ここで$\bar{X}$は、統計モデル$M(\mu)$からサンプリングした標本の標本平均値(データの平均値ではない)、$\mu,\sigma$は統計モデルで設定した母数平均、母数分散。
$Z(\bar{X},\mu)$が$N(0,1)$に従うということから、$Z(\bar{X},\mu)$が$N(0,1)$における出現頻度が計算できる。

%![Z値の頻度]()
\begin{figure}
\begin{center}
    \includegraphics[width=15cm]{./image/03_/normal_Z_frequency.pdf}
    \label{fig:cm_standard_normal_distribution}
  \end{center}
\end{figure}

$Z$の出現頻度を$Z$または$\bar{X}$の値に応じて書いたものが図\ref{fig:cm_standard_normal_distribution}です。
$Z(\bar{X},\mu)$の$95\%$予測区間が次のように求められる。
$$
-z_{0.025}<Z(\bar{X},\mu)<z_{0.025}
$$
このの範囲で、サンプリングされた標本の統計量$Z(\bar{X},\mu)$が$95\%$の確率で得られる。
統計モデルを使った判断でよく出てくる確率として分野を問わず、$95\%$が使われている。
%この値には身長に関する経験を使わずに決定しています。

また、$Z(\bar{X},\mu)$を式変形することで、標本平均が$95\%$の確率で出現する区間が推定できる。式を変形する。
\begin{eqnarray*}
    & -z_{0.025} < Z(\bar{X},\mu)<z_{0.025} \\
\rightarrow & -z_{0.025} < \frac{\sqrt{n}(\bar{X}-\mu)}{\sigma}  <z_{0.025} \\
\rightarrow & \mu - z_{0.025} \frac{\sigma}{\sqrt{n}} < \bar{X} < \mu + z_{0.025} \frac{\sigma}{\sqrt{n}}
\end{eqnarray*}
この統計モデルからサンプリングした標本の標本平均$\bar{X}$が$95\%$の確率で見つかる範囲のことを$95\%$信頼区間という。
%このことから、統計モデル$M(\mu)$でサンプリングしたときに、$95\%$の確率で、この範囲に平均値$\bar{X}$がえられます。

\subsection{サンプルサイズによる影響}
式を見てわかるように、サンプルサイズ$n$が大きくなれば、$\bar{x}$が入る範囲は狭くなる。
信頼区間がサンプルサイズに依存することを数値的に確認する。
図\ref{fig:confidence_interval_n}は、信頼区間が$N$に応じて変化する様子を図示している。


\begin{figure}
\begin{center}
    \includegraphics[width=15cm]{./image/03_/confidence_interval.pdf}
    \caption{信頼区間}
    \label{fig:confidence_interval_n}
  \end{center}
\end{figure}


実際に、$M(\mu=170)$を使って、$サンプルサイズを10$とし、標本を$100$個作ってみると、その分布は、図(B)のようになった。それぞれの標本に対してその信頼区間を描いたものが図(A)である。図Aの170cmのところにある縦の線は、統計モデル$M(\mu=170)$の母数平均である。
この170cmを跨いでいる信頼区間の個数はこの図では$96$個ある。コンピュータシミュレーションをするたびに毎回跨いでいる信頼区間の個数は変化するがおよそ95個である。このことは、信頼区間の定義から明らかである。


\begin{figure}
\begin{center}
    \includegraphics[width=15cm]{../markdown/section1/confidence_interval_model_count.pdf}
  \end{center}
\end{figure}


\begin{SMbox}{信頼区間は、データをたくさん取ったときに(サンプルサイズを大きくしたのではなく、サンプルサイズが同じ標本をたくさん集めたときに)、その範囲に真値が$95\%$の確率で含まれるの区間のこと}
信頼区間は、データをたくさん取ったときに、その範囲に真値が入る$95\%$の確率で含まれるの区間のこと\footnote{\url{https://www.slideshare.net/simizu706/ss-123679555}}。このように解説されることがある。
データを元に、統計モデルの母数を決定したときに、信頼区間が得られる。さらに計測を行い標本を作ると、標本の標本平均がこの信頼区間の間に含まれる確率が$95\%$であることを主張していると考えられる。

一般に、母集団が統計モデルにより、よく推測できるならば、無作為抽出の標本平均が$95\%$くらいの確率で信頼区間に含まれる。そうではないならば、$95\%$信頼区間にモデルの母数が含まれる確率は低く$95\%$とは異なる値をとる。

以上のことから、信頼区間は、データをたくさん取ったときに、その範囲に真値が$95\%$の確率で含まれるの区間のことという解釈はやめておいた方が良いと考えられる。
\end{SMbox}


\begin{framed}
まとめ、
\begin{itemize}
    \item 統計モデル$M(\mu)$によってサンプリングし、標本を得たとき、その平均値のよくある値の範囲(信頼区間)が計算できた
\end{itemize}
\end{framed}


\section{推測に利用できないと判定する}
\if 0
$N=1$では、$\bar{x}$が$154\sim185$あたりであれば、$M(170)$は棄却できない。$N=4$であれば、$\bar{x}$が$162\sim177$であれば、$M(170)$は棄却できない。$N=10$であれば、$\bar{x}$が$165\sim174$であれば、$M(170)$は棄却できない。
このようにサンプルサイズが増加することで、信頼区間が狭まり、棄却できるモデルの母数の範囲が狭まることがわかる。
\fi

\subsection{モデルを基準にしたデータと統計モデルの比較}
ここからは、母集団から無作為抽出したデータについて考えます。
無作為抽出によって得られた標本のサンプル$x_1,x_2,\cdots,x_n$について、その平均値を$\bar{x}$とします。
$M(171)$において、$\bar{x}=172$の場合($\phi(z)$を標準正規分布とする)、$\phi(z>\frac{\sqrt{n}(\bar{x}-\mu)}{\sigma}) = 0.289$であり、$\bar{x}=169$の場合、$\phi(z>\frac{\sqrt{n}(\bar{x}-\mu)}{\sigma}) = 0.133$です。このことから、統計モデル$M(\mu)$において、これらの観測値はそこまで稀ではありません。$M(168)$でも同様に計算できます。

\begin{lstlisting}
    xbar = 172
    mu=171
    sigma = 5.7
    N=10
    c = np.sqrt(N)*(xbar-mu)/sigma
    1-norm.cdf(c,0,1)
\end{lstlisting}
    


\subsection{統計モデルを基準にしたデータと統計モデルの比較}
統計量$Z(\bar{X},\mu)$がよく入る区間の式を変形し、データを得たときに、そのデータを基準にした$\mu$の範囲に変形してみます。
\begin{eqnarray*}
 & -z_{0.025} < Z(\bar{X},\mu)<z_{0.025} \\
\rightarrow & -z_{0.025} < \frac{\sqrt{n}(\bar{X}-\mu)}{\sigma}  <z_{0.025} \\
\rightarrow & \bar{x}- z_{0.025}\frac{\sigma}{\sqrt{n}} < \mu < \bar{x} + z_{0.025}\frac{\sigma}{\sqrt{n}}
\end{eqnarray*}
標本$1$標本$2$について、これを計算してみる。平均値は、それぞれ172.4, 169.0です。
標本$1$では、$168.9 < \mu < 176.0$、標本$2$では、$165.5 < \mu <172.6$です。
この範囲にある$\mu$をもつ統計モデルであれば、データをよくある範囲に入れることができます。
例えば、$M(168)$でも標本$1,2$の平均値はよくある範囲に収まります。

まとめ、
\begin{framed}
    \begin{itemize}
        \item 統計モデル$M(\mu)$のサンプルの平均が$95\%$の確率で入る範囲$\mu - z_{0.025} \frac{\sigma}{\sqrt{n}} < \bar{X} < \mu + z_{0.025} \frac{\sigma}{\sqrt{n}}$。現実の母集団が統計モデルによってよく推測できるなら、この範囲に平均値が入る確率は$95\%$に近くなることもある。逆に、統計モデルが現実をよく捉えることができなければ、母集団から無作為抽出した標本の平均値はこの範囲に入ることは少なくなる。
        \item データがよくある範囲に入る統計モデル$M(\mu)$の$\mu$の範囲$\bar{x}- z_{0.025}\frac{\sigma}{\sqrt{n}} < \mu < \bar{x} + z_{0.025}\frac{\sigma}{\sqrt{n}}$
        \item  統計モデル$M(\mu)$ではサンプルサイズを大きくすると、平均値が入る範囲が狭くなる。
    \end{itemize}
\end{framed}

\subsection{$Z(\bar{x},\mu)$以上の値が得られる確率}
$Z(\bar{x},\mu)\sim N(0,1)$により、$Z$以上の値が得られる確率も計算できます。つまり、
\begin{equation*}
    p = \varPhi(Z(\bar{x},\mu)>x)
\end{equation*}
です。$\bar{x}=172.4,\mu=168,\sigma^2=6.8,n=10$であれば、$Z(\bar{x},\mu)=2.04$であり、
$p=0.04$です。

\begin{lstlisting}
xbar = 172.4
mu = 168
sigma2 = 6.8**2
n=10
Z = np.sqrt(n)*(xbar-mu)/np.sqrt(sigma2)
print(Z)
p=1-norm.cdf(Z,0,1)
print(p*2)
\end{lstlisting}

%\chapter{科学的仮説検定}
ここまで、推測とデータが一致することを利用し、統計モデルの良さを評価し、良いモデルを選択しようとした。
科学的仮説検定では、データに対して、絶対にダメな統計モデルを調べる方法である。
この方法は、数理統計学の統計的仮説検定の枠組みを利用する。統計モデルからサンプリングされた標本に対して、その統計量が、統計モデルに対応する分布関数に従うことがわかっている。このことから、標本がある母数を持つ統計モデル由来であるかを判定する。
言い換えるなら、ある母数を持つ統計モデルから標本がサンプリングされたのかい?そうじゃないのか?どっちなんだい?に答える方法の一つである。
これを科学においては、ある母数をもつ統計モデルによって推測してもいいのかい?そうじゃないのかい?どっちなんだい?統計モデルが「ピー」と答える。



\section{自己標本の批判}
統計モデルからサンプリングした標本の統計量が従う確率密度関数が理論的に求められる。
正規モデルであれば、
\begin{equation*}
    Z = \frac{\sqrt{n}(\bar{x}-\mu)}{\sigma} \sim N(0,1)
\end{equation*}
である。このことを利用すれば、$Z$の値から、統計量が出現しにくさがモデル上で計算できる。
例えば、$Z=0$であれば、これ以上の値が出る確率は$0.5$程度なので、よくある統計量であることがわかる。
$Z=1.96$であれば、これ以上の値が得られる確率は$0.025$程度なので、なかなかのレアさであることがわかる。
これらの統計量以上に偏った値の出現しにくさを$p$値という。

$p$値が小さいなら(統計量が出現しにくいなら)、統計量の元の標本もそのモデルから得られにくい標本であるという判断をする。
つまり、モデルから標本の得られにくさの指標の一つが$p$値であるとも言え、$p$値が小さいほど、その標本はそのモデルから得られにくい。
どの標本もモデルから生成されたものであるが、ある閾値$\alpha$を決めて、それよりも小さな$p$値をもつ標本について、モデルから得られたものではないと判断する。ここで、$\alpha$を有意水準という。
モデルが生成したはずの標本であるが、閾値を決めてモデルから生成されたものではないとするのである。
モデルが自身から得られた標本を批判するのであるから、自己の標本を批判するのである。

言い換えを明示的に書いておく。
\begin{center}
    標本$\rightarrow$ 統計量 $\rightarrow$ $p$値
\end{center}

ここで、母集団から無作為抽出した標本(モデルから生成された標本ではない)を正規モデルにより、予測できるかを考える。
上記の議論と同様に、標本から、統計モデルにあった統計量を計算し、それよりも偏った値が出現する確率($p$値)を計算する。
$p$値が小さければ、モデルにより予測できないと考え、値が1に近いほど、もしかしたらモデルで予測できるのかもしれないと考える\footnote{$p$値だけで判断してはいけない}。
$p$値が$\alpha$よりも小さいとき、流石にこのモデルでは予測できないでしょうと判定する。このとき、モデルを却下すると宣言する。
$p$値が$\alpha$よりも大きい場合でも、そのままこのモデルで予測できるとは宣言しない。他の指標やデータとモデルをグラフにより比較し、予測できそうかを考察する必要がある。

%この標本は、モデルからサンプリングしたものではない。
%標本の統計量が、モデルの上で得られやすいものかを調べる。
%$M_a$を棄却する判断をする閾値は、言い換えると、統計モデル$M_a$の棄却される母数(棄却域$R$)の出現確率を$\alpha$とした。


以上のことは、托卵行動に例えることができる。
モズは、カッコウに対して卵を託す托卵を行い、カッコウは、モズの卵とは気が付かず、そのまま育てる。
ここで言い換えたいのは、カッコウは統計モデルであり、卵は標本そして、モズは科学者である。
統計モデルは、モデルからのサンプリングされた標本を巣穴に置いている。
卵の情報を要約した統計量が、モデル由来であることをモデルはその推定量の出現頻度を推測できる。
出現頻度が$p$値である。
モデルの巣に自然から無作為抽出した標本を科学者が置く。
その標本の統計量の出現頻度をモデルは推測できる。
得られた推測から、標本がモデルの卵であることを判定するのは科学者である。

\begin{figure}
    \begin{center}
        \includegraphics[width=15cm]{./image/01_/conceptual_diagram/conceptual_diagram.003.png}
        \caption{科学的仮説検定の概念図}
        \label{fig:conceptual_diagram_test}
    \end{center}
\end{figure}
    

ここで、いくつかのことを定義しておく。
\begin{defi}
    統計モデルと標本を比較して、モデルが母集団のことを予測できないと判断するとき、統計モデルを却下すると宣言する。
    特に、データから推測される却下されないモデルの母数の範囲を信頼区間といい、それ以外の区間を却下区間という。
    統計モデルにおいて、標本の統計量以上に偏った(大きいまたは小さな)値が得られる確率を$p$値と呼ぶ。

    %ある標本から求められた統計量以上に大きな値が得られる確率を$p$値と呼ぶ。
    %絶対にダメと判断されないときは、統計モデルを採択(棄却の対義語)すると宣言しない。
    %統計モデルが棄却されるのは、統計モデルの仮定によって変化する。本書の範囲内であれば、統計モデルの母数、分布関数、独立同一の分布関数からサンプリングされたことによる。
    %最尤統計モデルにおいて、棄却されない統計モデルの母数の範囲を信頼区間といい、棄却されるモデルの母数の範囲を棄却域という。
    棄却される$p$値の閾値を有意水準$\alpha$と言い、一般に$\alpha=0.05$が使われる。
    言い換えれば、$\alpha$値は、統計モデルからサンプリングされた値について、これが元の統計モデルからサンプリングなのかどうかを判定する閾値\footnote{限界値}のことである。
    
    この定義から、統計モデルから得た標本だとしても$\alpha \%$の割合で、統計モデルが棄却される\footnote{実際のデータが$\alpha\%$の割合で棄却されるということではない。モデルから生成された標本であるのに、この統計モデルから生成されていないと判定を下される}。

    統計モデルの分布関数が変化すれば、その統計モデルにおける信頼区間・棄却域の式も変わる\footnote{中心極限定理を利用し、統計量の出現範囲を近似することが多い。サンプルサイズが多ければ中心極限定理が使えるのではなく、常に解析解と乖離があることを意識するべき}。
\end{defi}





\section{再生性}
\subsubsection{$(\star)$ $N(\mu,\sigma^2)$に従う確率変数であることを判定できるか}
$N(0,1)$に従う確率$x_1,x_2,\cdots,x_n$から計算した統計量、$z=\frac{\bar{X}-0}{\sqrt{\frac{1}{n}}}$は、$N(0,1)$に従い、$z$が$95\%$の確率で見つかる範囲は$[-1.96,1.96]$である。
同様に、$y_1,y_2,\cdots,y_n \sim N(1.96,1)$であるならば、$z=\frac{\bar{Y}-1.96}{\sqrt{\frac{1}{n}}}$は、$N(0,1)$に従う。

確率変数から、特定の母数を持つ正規分布に従わないことを示すことはできるだろうか。
具体的な問題設定として、
$y_1,y_2,\cdots,y_n$を正規分布に従う確率変数とする。そのとき、$y_1,y_2,\cdots,y_n$が$N(\mu,\sigma^2)$に従わないことを判断する良い方法はどのようなものだろうか。

ここで、$y_1,y_2,\cdots,y_m \sim N(1.96,1)$にもかかわらず、$N(0,1)$に従うと推測した場合、$z=\frac{\bar{Y}-0}{\frac{1}{\sqrt{n}}} \sim N(0,1)$であると考えられる。
$z$の分子の$\mu$が$0$になっていることに注意が必要である。
実際に、$y_1,y_2,\cdots y_{100}$を$N(1.96,1)$からサンプリングした標本を$100$個作ってみると、およそ$19$を中心に分布することがわかる。
このことは、$y_1,y_2,\cdots,y_m\sim N(0,1)$であるならば、$z$は、$[-1.96,1.96]$の間で$95\%$の確率で入るので、この推測が間違いであることが推測される。
以上の考察から、$y_1,y_2\cdots,y_n\sim N(0,1)$ではないと判断する。

\begin{figure}
    \begin{center}
        \includegraphics[width=15cm]{./image/02_/normal_distribution_test.pdf}
        \caption{(a)$N(1.96)$に従う確率変数を100個サンプリングし、その標本を1000個集めたときの$z=\sqrt{100}(\bar{X}-0)$のヒストグラム (b)$N(0,1)$に従う確率変数を100個サンプリングし、その標本を1000個集めたときの$z=\sqrt{100}(\bar{X}-0)$値のヒストグラム}
    \end{center}
\end{figure}


もう一つ例を挙げる。
$y_1,y_2,\cdots,y_n \sim N(170,5.8)$とする。このとき、この標本が$N(168,5.8)$によりサンプリングされたものではなくことを示すことはできるだろうか。
$z=\sqrt{n}\frac{\bar{y}-168}{\sigma}$を計算すればよい。
図には、$N(170,5.8)$に従う確率変数を100個サンプリングし、その標本を1000個集め、ヒストグラムを描いた。
これをみると、$0.5$を中心に分布が広がることがわかる。$z=\frac{\bar{X}-168}{\sqrt{\frac{5.8}{n}}}\sim N(0,1)$であるはずである。
複数回、標本を得た場合でも、$z$が$[-1.96,1.96]$の範囲に収まっている。このことは、$N(168,5.8)$ではないと判断できないことを示唆している。


\begin{figure}
    \begin{center}
        \includegraphics[width=15cm]{./image/02_/normal_distribution_test2.pdf}
        \caption{$N(170,5.8)$に従う確率変数を100個サンプリングし、その標本を1000個集めたときの$z=\sqrt{100}(\bar{X}-168)$のヒストグラム}
    \end{center}
\end{figure}


ある正規分布に従う確率変数$x_1,x_2,\cdots,x_n$が母数の異なる正規分布で得られる確率も計算できる。具体的には、$x_1,x_2,\cdots,x_n\sim N(\mu,\sigma^2)$とし、これが$N(\mu_1,\sigma_1^2)$で得られるとすると、そのときの統計量は、$z=\frac{\bar{x}-\mu_1}{\frac{\sigma_1}{n}}$である。この$z$は、$N(0,1)$に従うと考えられるので、$\phi(|z|>Z)$となる確率を計算すれば良い。

\begin{theo}
    確率変数$x_1,x_2,\cdots,x_n \sim N(\mu,\sigma^2)$ならば、$z=\frac{\bar{X}-\mu}{\sqrt{\frac{\sigma}{n}}} \sim N(0,1)$である。
    一方で、確率変数$x_1,x_2,\cdots,x_n \sim N(\mu,\sigma^2)$とする。$N(\mu_1,\sigma_1^2)$は正規分布とする。ただし、$\mu\neq \mu_1, \sigma =\sigma_1$このとき、$z=\frac{\bar{X}-\mu_1}{\sqrt{\frac{\sigma_1}{n}}} \sim N(0,1)$ではない。
\end{theo}
$\mu$と$ \mu_1$が極めて近い値のとき、$z=\frac{\bar{X}-\mu_1}{\sqrt{\frac{\sigma_1}{n}}} $も$N(0,1)$におけるよくある値になる言い換えれば、$\phi(|z|>Z)$は十分大きい。
一方で、$\mu$と$ \mu_1$が離れた値を取ると、$\phi(|z|>Z)$は小さな値になる。

\if 0
\subsection{$(\star)$ $Exp(\lambda)$に従う確率変数であることを判定できるか}
$x_1,x_2,\cdots,x_n \sim Exp(\lambda)$であるとき、$n\bar{x}\sim Ga(n,\frac{1}{\lambda})$である。
母数不明の指数分布に従う確率変数が、$x_1,x_2,\cdots,x_n \sim Exp(\lambda)$と仮定したとき、$n\bar{x}\sim Ga(n,\frac{1}{\lambda})$でないならば、$x_1,x_2,\cdots,x_n \sim Exp(\lambda)$ではないと判断できるだろうか。シミュレーションによって確認してみよう。

この論法は、母数が不明の指数分布に従う確率変数を得たとき、その指数分布の母数が特定の値ではないことを示すためにこの論法を利用する。ここでは、母数が$\lambda=1,2,5,10,100$からサンプルサイズ4の標本を$1000$生成し、それら標本の統計量$n\bar{X}$のヒストグラムと、ガンマ関数$Ga(100,1)$の確率密度関数を比較する。

\begin{figure}
    \centering
    \includegraphics[width=15cm]{./image/02_/Exp_Gamma_simulation.pdf}
    \caption{(a)$Ga(10,1)$の確率密度関数。(b-e)指数分布からサンプルサイズ$4$の標本を$1000$回生成し、その統計量$n\bar{x}$のヒストグラム}
    \label{fig:exp_gamma_simulation}
\end{figure}

図\ref{fig:exp_gamma_simulation}(a)は、指数分布$Exp(\lambda=1)$の確率密度関数を示している。
図\ref{fig:exp_gamma_simulation}b-eは、シミュレーションの結果を示している。
図\ref{fig:exp_gamma_simulation}(b)には、指数分布$Exp(1)$に従う確率変数の統計量$n\bar{x}$が確かに、$Ga(100,1)$に従うことが確かめれる。
図\ref{fig:exp_gamma_simulation}(c-e)では、指数分布の$\lambda$が$1/2,1/5,1/10$のときの統計量のヒストグラムである。これらと、図\ref{fig:exp_gamma_simulation}(a)を比較すると、分布が異なっているので、確かに、$Ga(100,1)$には従わないことがわかる。
\fi



\if 0 
これらの事象は、統計モデルの上で観測される、数学的な事実です。
数学を扱っている以上はこの事実は決して崩れることはありえません。
一方で、我々が扱う現象ではどうなるでしょうか。現象が数学な分布関数から生成されていることは決してありえません。
誰かがサイコロを振って、人々の身長を決めているのなら話は別ですが、
人の身長が、ランダムに正規分布によって決定されることはありませんね。

$M(168)$モデルの平均値は$168cm$、データでは$171cm$程度なので、$3cm$小さい。また、
$180cm$以上の人の割合を使ってモデルとデータの乖離を調べることができました。
$180cm$の人がたまたまいなかった場合は、$M(169.1),M(168)$のどちらも推測できているとは言い難いことになります。このことから、特定の値を使って乖離を判定することは難しいと考えられます。


$\phi(z>Z(\mu))を$p値として、絶対に選択してはいけない統計モデル$M(\mu)$の母数$\mu$を調べます。具体的には、指標$p$が$0.05$より小さい統計モデルを選択しないようにします。その母数の範囲は$162.14 >\mu, \mu > 174.31$です。この母数の統計モデルは$p=0.05$の基準で使わないことを統計モデルを棄却すると言います。逆に、$p>0.05$となるモデルは、積極的に正しいとは考えません。明らかに間違いではないけども正しくもないという判断をします。


$p$値を使う方法がとられます。p値とは統計モデルとデータの乖離度合いを示す指標です。p値は$0~1$の値をとり、$0$に近いと統計モデルとデータが乖離していると判断します。
\fi 




\begin{SMbox}{偶然の差が生じたかを確かめたい}
    「偶然の差が生じたかを確かめたい」や「こんなことが起こる確率は$5\%$くらい」という言葉を統計学の教科書で見たことがある。これらは、本書での説明とは異なっており、本書と互換性はない。
    %「統計モデルの上で統計量が現れる確率が十分小さいことを確かめたい」や「統計モデル上でそのような統計量が得られる確率が$5\%$」を省略して書いたものです。
    
    科学では、実験で得られたデータは、同様の実験を行った場合、同様のものが得られるということが前提になっている。このことを現象に再現性があると言う。
    再現性のないデータを現状の統計学で扱うことや、現実の現象が得られる確率を議論することは困難である。
\end{SMbox}

    

\section{統計量をもとにしたモデル間類似度}
母数の異なる二つの統計モデル$M_a,M_b$について考察する。母数から標本を得て、それぞれの統計モデルを統計量を元にモデル間の類似度を計算する。

\subsection{検出力の定義}
$M_a$の棄却できない統計量の範囲(信頼区間$A$)に$M_b$の統計量が出現する確率を$\beta$とする。$\beta$を検出力という\footnote{検出力を検定力または統計力と呼ぶこともある。\\ \url{https://id.fnshr.info/2014/12/17/stats-done-wrong-03/}}。
%$\alpha$は統計モデルとデータを比較したとき、そのモデルを棄却する指標である。
$\beta$は、二つの異なるモデルを比較するための指標で、一方のモデルで棄却できない母数がもう一方のモデルで出現する確率である。
$M_a$に対する$M_a$の検出力は、$1-\alpha$であり、$M_a$を棄却する閾値を低く設定すると、$\beta$は大きな値になる。
二つの統計モデルの母数がよく一致するならば、$\beta$は$1-\alpha$に近い値を取り、一致していないならば、$\beta$は0に近い値を取る。
具体的に、$\alpha,\beta$を式で書くと、
\begin{eqnarray*}
    P_a(\mu \in R_a) = \alpha\\
    P_b(\mu \in A_a) = \beta
\end{eqnarray*}
ここで、$R_a,A_a$はそれぞれ統計モデル$M_a$の棄却域、信頼区間、$P_a,P_b$は、それぞれ統計モデル$M_a,M_b$における統計量に関する確率密度関数。

\subsection{正規分布モデルの検出力}
具体的に、$P_a(\mu \in R_a),P_b(\mu\in A_a)$を計算してみる。
正規モデルを構築する
\begin{quote}
    \begin{enumerate}[(1)]
\item i.i.d
\item $N(\mu,\sigma^2)$
\item 母数$\mu$。$\sigma$は既知とする(一般性を持たせるために、具体的な値は書かない。$\sigma=1$と読み替えて進めても良い)
\end{enumerate}
\end{quote}
このモデルを$M(\mu)$とし、$M_a=M(\mu_a),M_b=M(\mu_b)$とする。
$M_a$または、$M_b$からサンプリングされた確率変数$x_1,x_2,\cdots,x_n$の平均値は、それぞれ$\bar{x}_a\sim N(\mu_a,\sigma/n)$または$\bar{x}_b\sim N(\mu_b,\sigma/n)$である。
$M_a$の信頼区間$A_a$は、$|\bar{x}_a|<\mu_a+\sigma / \sqrt{n}z_{2.5\%}$である。
このとき、$P_a$を$N(\mu_a,\sigma)$の確率密度関数とすると、
\begin{equation*}
    P_a(\mu \in A_a) = \alpha
\end{equation*}
であるのは定義から明らか。
また、$P_b$を$N(\mu_b,\sigma)$の確率密度関数とすると、
\begin{equation*}
    P_b(\mu \in A_a ) = \beta
\end{equation*}
である。
$\mu_a$と、$\mu_b$が一致していれば、$P_b(\mu \in A_a ) = 1-\alpha$である。
$\mu_b$が$\mu_a$から離れていくと、$P_b(\mu \in A_a)=0$に近づいていく。


\begin{figure}
\begin{center}
    \includegraphics[width=15cm]{./image/04_/power_of_a_test_2.pdf}
    \caption{統計モデル$M_a,M_b$から計算された統計量$\bar{x}$の確率分布$P_a,P_b$。(a)灰色の範囲は$M_a$の信頼区間。(b)灰色の領域は、$1-\beta$の領域を示している。$\beta$の領域が小さいので、描画できなかった (c)$\mu_b$が$\mu_a$に近いときの$\beta$と$1-\beta$の領域。(d)灰色の範囲の面積が$\alpha$を示している。}
    \label{fig:power_of_test_alpha_beta}
\end{center}
\end{figure}


検出力と$\alpha$の領域を図示した(図\ref{fig:power_of_test_alpha_beta})。$M_a$の$95\%$信頼区間は、$|\mu|<\mu_a+z_{0.025}\frac{\sigma}{\sqrt{N}}$である。信頼区間は、図\ref{fig:power_of_test_alpha_beta}(a)において灰色で塗った$x$軸の範囲である。$\alpha$は図\ref{fig:power_of_test_alpha_beta}(c)の灰色で塗りつぶした領域の面積である。
検出力$1-\beta$は、$M_b$における$M_a$の信頼区間の外側の領域の面積なので、図\ref{fig:power_of_test_alpha_beta}(b)の濃い灰色の範囲である。

$\alpha$を0に近づけていくと、信頼区間は徐々に大きくなり、$\beta$は大きくなる。
$\alpha$を1に近づけていくと、信頼区間は徐々に狭くなり、$\beta$は小さくなる。



\begin{figure}
    \begin{center}
        \includegraphics[width=15cm]{./image/04_/power_of_a_test_3.pdf}
        \caption{統計モデル$M_a,M_b$から計算された統計量$\bar{x}$の確率分布$P_a,P_b$。(a)$\mu_a,\mu_b$のサンプルサイズ$1$の平均値がしたがう確率密度関数$N(\mu_a,\sigma^2/1),N(\mu_a,\sigma^2/1)$。(b)(a)と同じ$\mu_a,\mu_b$に対して、サンプルサイズを$30$にした場合の確率密度関数。(c)$\mu_a,\mu_b$が(a)よりも近いときの$\bar{x}$の確率密度関数。(d)(c)と同じ$\mu_a,\mu_b$に対してサンプルサイズを$30$にした場合の$\bar{x}$の確率密度関数。}
        \label{fig:power_of_test_alpha_beta_sample_size}
    \end{center}
    \end{figure}

    

$\alpha$、$M_a$の母数$\mu_a$、$M_b$の母数$\mu_b$を固定したまま、サンプルサイズを変化させ,
$\beta$の変化を表す(図\ref{fig:power_of_test_alpha_beta_sample_size})。$\bar{x}$の確率密度関数($N(\mu,\sigma^2/n)$)の分散がサンプルサイズによって変化することは明らかである。このことから、サンプルサイズが大きくなると、信頼区間は徐々に狭くなり、$1-\beta$は大きくなる。サンプルサイズが小さいときは、$1-\beta$も小さくなる。

$\mu_a$を固定し、$\mu_b$を変化させたときの検出力$1-\beta$を図\ref{fig:power_of_test_N_mu0_variable}に示した。
サンプルサイズが大きければ、$1-\beta$も大きくなることがわかる。

\begin{figure}
    \begin{center}
        \includegraphics[width=15cm]{./image/04_/power_of_test.pdf}
        \label{fig:power_of_test_N_mu0_variable}
        \caption{$\mu_a$を変数にしたときの検出力(検出力関数)。}
    \end{center}
\end{figure}

$\beta$を定義したことにより、$\beta$の数値を決定し、$M_a,M_b$の違いが$\beta$になるために必要なサンプルのサイズが推測できる。ここでは、$\mu_a,\mu_b$が固定されている状況を考える。
検出力$1-\beta$は$1$に近いほど、$M_a,M_b$が違うと主張できる。
あらかじめ決めたおいた基準の$1-\beta$を閾値を設定し、それ以上の$1-\beta$となるサンプルサイズを推測する。
サンプルサイズが小さければ、$M_a$と$M_b$の違いは曖昧であり、サンプルサイズが大きくなると、はっきりとモデルの違いがわかる。




\subsection{$\beta$の計算}
正規モデル$M_a,M_b$を使って、$\beta$を計算してみる。
$M_a$の信頼区間は、
\begin{equation*}
    -z_{0.025}\leq \frac{\sqrt{n}(\bar{x}-\mu_a)}{\sigma}\leq z_{0.025}
\end{equation*}
より、
\begin{equation*}
    A_a = \{ \mu ; \mu_a -\frac{\sigma}{\sqrt{n}}z_{0.025} \leq \mu \leq \mu_a +\frac{\sigma}{\sqrt{n}}z_{0.025} \}
\end{equation*}
である。ここで、$a=\mu_a -\frac{\sigma}{\sqrt{n}}z_{0.025},b = \mu_a +\frac{\sigma}{\sqrt{n}}z_{0.025} $とおく。棄却域は$A_a$以外の$\mu$である。$M_b$の標本平均$\bar{x}_b$は、$N(\mu,\frac{\sigma^2}{n})$に従うので、$A_a$の区間で、$N(\mu_b,\frac{\sigma^2}{n})$の面積を計算すれば良い。
ここで、$\frac{\sqrt{n}(\bar{x}_b-\mu_b)}{\sigma}\sim N(0,1)$である。
このことを利用すると、
$a,b$は、$N(\mu_b,\frac{\sigma^2}{n})$の確率変数だとすると、
\begin{eqnarray*}
    A &=& \frac{\sqrt{n}(a-\mu_b)}{\sigma} \\
    &=& \frac{\sqrt{n}(\mu_a-\frac{\sigma}{\sqrt{n} z_{\alpha/2}})}{\sigma}\\
    &=& -z_{\alpha/2}+\frac{\sqrt{n}}{\sigma}(\mu_a-\mu_b)
\end{eqnarray*}
同様に、
\begin{eqnarray*}
    B &=& \frac{\sqrt{n}(b-\mu_b)}{\sigma} \\
    &=& \frac{\sqrt{n}(\mu_a-\frac{\sigma}{\sqrt{n} z_{\alpha/2}})}{\sigma}\\
    &=& z_{\alpha/2}+\frac{\sqrt{n}}{\sigma}(\mu_a-\mu_b)
\end{eqnarray*}
である。以上より、確率密度関数$N(0,1)$において、$-z_{\alpha/2}+\frac{\sqrt{n}}{\sigma}(\mu_a-\mu_b) \leq x\leq  z_{\alpha/2}+\frac{\sqrt{n}}{\sigma}(\mu_a-\mu_b)$の間で積分すれば良い。

$d=\frac{\mu_a-\mu_b}{\sigma}$とおく。$d=0.6,n=9$とする。このときの$\beta$を計算してみる。$N(0,1)$において、$-z_{\alpha/2} -0.6\sqrt{n} \leq x \leq z_{\alpha/2} +0.6\sqrt{n}$の区間で積分する。

\begin{lstlisting}
A,B = norm.interval(0.95,0.,1)
N = 9
d = 0.6
a,b = A+d*np.sqrt(N),B+d*np.sqrt(N)
print(a,b)
norm.cdf(b,0,1)-norm.cdf(a,0,1)
\end{lstlisting}

答えは、$0.564$


\subsubsection{サンプルサイズ}
$d$と検出力を指定したときに、$M_a,M_b$の類似度を検出力以上にするためのサンプルサイズが計算できる。
$\beta=0.1,\d=0.8$とし、この$\beta$を満たすように$N$を計算した。

\begin{lstlisting}
A,B = norm.interval(0.95,0.,1)
beta = 0.1
d = 0.8
for N in range(10,200,2):
    a,b = A+d*np.sqrt(N),B+d*np.sqrt(N)
    beta_ = norm.cdf(b,0,1)-norm.cdf(a,0,1)
    if beta_ < beta:
        break
print(N)
\end{lstlisting}
計算を実行すると、$18$であることがわかる。



\subsection{最尤モデルでの$\beta$の計算}
\subsubsection{データを元にしたモデルとモデルの類似度}
統計モデルAを$M(\mu=170)$とし、統計モデルBを$M(\bar{X})$とする。ここで、$\bar{X}$は、無作為抽出によって得られた標本の平均であり、標本の大きさを$100$とする。
モデルA,Bの間の検出力が計算可能である。
$d=\frac{170-\bar{X}}{6.8}$、$n=100$であるので、$\bar{X}=168$を得たとすると、
\begin{lstlisting}
A,B = norm.interval(0.95,0,1)
N = 100
d = (170-168)/(6.8)
a,b = A+d*np.sqrt(N),B+d*np.sqrt(N)
print(a,b)
norm.cdf(b,0,1)-norm.cdf(a,0,1)
\end{lstlisting}
その検出力は、$0.163$


\section{自己否定の過誤}
統計モデルの中で、統計モデルを統計量により検査するときに、モデル自身を絶対にダメなモデルと判断してしまうことを自己否定の過誤\footnote{データとモデルを比べたときに、誤ってモデルが間違いと判定することを第一の過誤と一般の教科書は紹介している。誤ってモデルが間違いと判断するとはどのようなことなのかの定義がないので、この定義の意図がわからない。}と言う。
この過誤は2つの要因に分解でき、\footnote{$\alpha_2$は$\alpha_1$に関係するので実際には、分解できない。気持ちとしては、$\alpha_2$は、$\alpha_1$を変数に持つ関数である$\alpha=\alpha_2(\alpha_1)$。}、不適切な統計量を使用することで、棄却域と統計量の違いにより生じる$\alpha_1$、そして、検定を繰り返して生じる$\alpha_2$である($0<\alpha_2 \leq 1$)。
$\alpha_2=\alpha$となっていれば、有意水準$\alpha$の検定ができる。
$\alpha_1$は、統計モデルと、その統計量の関数になっており、言い換えれば、統計量が統計モデルの中で設計通りの振る舞いをしているかを測る指標である。
正規モデルを使い、統計量$T$を使った場合、$\alpha_1 \approx	 0 $であるが、指数モデルを使い、統計量$T$を使った場合、$\alpha_1$が指定した$\alpha$よりも多くなる。これを見ていく。
$\alpha_2$は、$\alpha\times 2$以上になる場合、軽視されることはないが、
$\alpha_1$が同程度の隔たりになる場合においては無視され、$\alpha_1$は$\alpha_2$よりも軽視されがちであることも説明する。
%統計モデルに対して不適切な統計量を使ってモデルの検証を試みると、第一の過誤が変化することがわかっている。

\subsection{どんな統計モデルでも$T$統計量で調べよう($\alpha_1$)}
統計モデルの分布の仮定が正規分布以外の場合においても、$T$統計量を使ってモデル自身を検証できるのかを調べる。次の統計モデル$M_E(\lambda)$を構築する。
\begin{enumerate}
    \item $X_1,X_2,\cdots,X_n $はi.i.d
    \item 指数分布
    \item $\lambda$
\end{enumerate}
母数$\lambda=1$とした統計モデルを$M_E(1)$とする。
$M(1)$からランダムサンプリングした確率変数$x_1,x_2,\cdots,x_n$から次の統計量を計算する。
\begin{equation*}
    T = \frac{\bar{X}-1}{\sqrt{\frac{\sigma^2}{n}}}
\end{equation*}
ここで、$T \sim t(n-1)$とする。
$T$値が$t(n-1)$の棄却域に入っている頻度を数値計算により計算する。
具体的に、平均$1$の指数分布または、平均$1$、標準偏差$1$の正規分布からサンプルを得て標本を作る。その標本を$100000$回取得する。
このとき、$T$値を計算し、$T$値いじょの値が得られる確率$p$を計算する。その$p$が$p<0.05$となる割合を計算する。以上をサンプルサイズを変化させてシミュレーションを行なった。
平均$1$、標準偏差$1$の正規分布の場合、$T$値は$t(n)$分布に従ので、$p<0.05$となる頻度も、$5\%$程度になることが期待される。
一方で、平均$1$の指数分布の場合、$T$は$t(n-1)$分布に従うとはいえない。このことから、$p<0.05$となる頻度は計算してみなければわからない。


\begin{figure}
    \begin{center}
        \includegraphics[width=15cm]{./image/04_/t_test_expon_norm.pdf}
        \caption{正規分布または指数ぶんぷから得た標本の$T$値から計算した$p$値で、$p<0.05$以下になる割合}
    \end{center}
\end{figure}

シミュレーションの結果、正規分布から標本を得た場合、$p<0.05$になる割合は、サンプルサイズに依存せず、$5\%$程度であり、期待通りである。
一方で、指数分布から標本を得た場合、$p<0.05$になる割合はサンプルサイズに応じて変化しており、また、どのサンプルサイズでも$p<0.05$となる割合は$5\%$より多い。

このことから、指数モデルの$\alpha_1$は、$\alpha_1>0.05$であることがわかり、統計量を正しく選ばなかったことで、自己否定の過誤が期待した$0.05$よりも大きくなっていることがわかる。

\if 0
\subsubsection{いつでも正規モデルでいこう}
データが非対称に分布しているのに、統計モデルに正規分布を指定した場合、推定が正しく行えないことを確認しておこう\footnote{元ネタ。
    小標本 t 検定の誤解:中心極限定理と一般化線形モデル 井口豊(生物科学研究所,長野県岡谷市)\url{https://biolab.sakura.ne.jp/small-sample-t-test-glm.html}}。
次のような統計モデルを構築する。
\begin{enumerate}
    \item $X_1,X_2,\cdots,X_n $はi.i.d
    \item 正規分布
    \item 正規分布の母数$\mu$,$\sigma^2$の値は不明
\end{enumerate}
正規分布の母数$\mu=1$とした統計モデルを$M(1)$と記述する。
この$X_1,X_2,\cdots,X_n$について次の統計量が$t(n)$分布に従うことがわかっている。
\begin{equation*}
    T = \frac{\bar{X}-1}{\sqrt{\frac{\sigma^2}{n}}} \sim t(n)
\end{equation*}
このとき、データが、既知の確率分布から得られた場合に、$p$値がサンプルサイズによってどのように変化するのかを調べる。
具体的に、平均$1$の指数分布または、平均$1$、標準偏差$1$の正規分布からサンプルを得て標本を作る。その標本を$100000$回取得する。
このとき、$T$値を計算し、$T$値いじょの値が得られる確率$p$を計算する。その$p$が$p<0.05$となる割合を計算する。以上をサンプルサイズを変化させてシミュレーションを行なった。

平均$1$、標準偏差$1$の正規分布の場合、統計モデルの仮定と一致するので、$T$値は$t(n)$分布に従う。よって、$p<0.05$となる頻度も、$5\%$程度になることが期待される。
一方で、平均$1$の指数分布の場合、統計モデルの仮定と一致しない。このことから、$T$は$t(n)$分布に従うとはいえない。このことから、$p<0.05$となる頻度はわからない。


\begin{figure}
    \begin{center}
        \includegraphics[width=15cm]{./image/04_/t_test_expon_norm.pdf}
        \caption{正規分布または指数ぶんぷから得た標本の$T$値から計算した$p$値で、$p<0.05$以下になる割合}
    \end{center}
\end{figure}

シミュレーションの結果、正規分布から標本を得た場合、$p<0.05$になる割合は、サンプルサイズに依存せず、$5\%$程度であり、理論と一致する。
一方で、指数分布から標本を得た場合、$p<0.05$になる割合はサンプルサイズに応じて変化しており、また、どのサンプルサイズでも$p<0.05$となる割合は$5\%$より多い。
このように、データが正規分布とかけ離れているにもかかわらず、正規モデルを構築し、そこから統計量を計算しても、的外れになることがあることを示唆している\footnote{$n$を大きくしたとき、中心極限定理より、$p<0.05$となる割合も$5\%$に近づくと解釈することがある。本当だろうか。具体的には、次の定理が成り立つのだろうか。
\begin{quote}
\begin{theo}
    $X_1,X_2,\cdots,X_n \sim Exp(\lambda)$とするとき、$T=\frac{\bar{X}-1/\lambda}{\sqrt{\frac{S^2}{n}}}$ここで、$S^2=\frac{1}{n-1}\sum_{i=1}^n(X_i-\bar{X})^2$である。$T\sim t(n-1)$または、$t$がなんらかの統計分布に従う。または、$E[T]<\infty,Var[T]<\infty$
\end{theo}
\end{quote}
このことが成り立つなら、中心極限定理も成立し、$n$が十分大きいときに、分布関数を近似できそうである。
}
\fi
\begin{SMbox}{サンプルサイズがxx以上あるから$t$検定}
        サンプルサイズがある値以上あるので、中心極限定理により、$t$検定が利用できるというものもある\footnote{http://id.ndl.go.jp/bib/024660739}。このロジックが読み込めなかったので、その謎を明らかにすべく我々はアマゾンの奥地へ向かった。

        %サンプルサイズが1以上であれば、$t$検定を行うことは原理的には可能である。
        データが指数分布的であるときに、$t$検定を使うときに生じる問題は上でみた通りであり、$p<0.05$となる標本の割合が多くなっているので、間違った推測をする可能性が高くなる。
        他の分布関数でもおそらく同じような現象が現れる。
        このことから、我々は「$t$検定が利用可能である」は正確ではなく、「$t$検定を使うことができるが、間違った推測である確率が高くなる」ということだと推察した。

        %サンプルサイズが大きくなれば、$\alpha_1$は小さくなる。
        業界によっては、サンプルサイズが$xx$以上であれば、過誤を無視して良いというふうに言われることもある。実際には、設計したモデルと
\end{SMbox}

\section{検定を繰り返し使おう($\alpha_2$)}
次の統計モデルによって複数の標本について推測することを考える。
\begin{enumerate}
    \item $X_1,X_2,\cdots,X_n $はi.i.d
    \item 正規分布
    \item $\mu$,$\sigma^2=10$
\end{enumerate}
ここまでは、一つの標本に対して、統計モデル$M(\mu)$により推測できるかを考えていた。
ここでは、複数の標本について、$M(\mu)$により推測できるかを仮説検定を指標にし考える。
標本が$3$個あるとする。このとき、それぞれの標本の統計量$T$が信頼区間に入っている確率は、$(1-\alpha)$である。全ての標本の統計量$T$が信頼区間に入っている確率は、その積$(1-\alpha)\times(1-\alpha)\times(1-\alpha)=(1-\alpha)^3$であり、この確率で統計モデルは棄却されない。
一方で、棄却される確率は、$1-(1-\alpha)^3$である。
\begin{table}[hbtp]
    \caption{標本数に応じた$\alpha_2$}
    \label{table:multiple_test_reject_prob}
    \centering
    \begin{tabular}{lcr}
      \hline
      標本数  & $\alpha=0.05$  &  $\alpha=0.01$ \\
      \hline \hline
       1 & $0.05$  & $0.01$ \\
       2 & $0.0975$ & $0.0199$\\
       3 & $0.142$ & $0.0297$\\
       4 & $0.185$ & $0.0394$\\
    \end{tabular}
  \end{table}
表\ref{table:multiple_test_reject_prob}は、標本数に応じた$\alpha_2$である。標本数が大きくなるについれて、$\alpha_2$が大きくなることがわかる。

$\alpha_1$がレベル$\alpha$の検定になっていない場合、$\alpha_2$はさらに有意水準$\alpha$から隔たりの多い数値になる。




\section{類似度の過誤}
統計モデルの間の類似度を検出力といった。
統計モデルに対して、不適切な統計量を与えたとき、検出力を歪める。
これを類似度の過誤といい、その確率を$\beta'$で表す。
    
%
\section{モデルと標本の乖離による過誤}
データとモデルを統計量により比較したとき、その乖離について測ることのできないことをまとめる。

\subsection{モデルの確率密度関数と標本の分布の乖離による過誤}
経験がないことで、適当なモデルを構築し、非常に少数のサンプルサイズしか得られないことで、そのモデルの妥当性について検証できないまま、モデルとデータを仮説検定により比較し、判断することにより生じる間違いである。
例えば、データの分布が非対称に分布しているのに、正規分布を含んだ統計モデルを構築し、$T$統計量により検定をおこなったとする\footnote{データが非対称に分布していることから、正規分布を含んだ統計モデルでは推論できないことがわかるので、統計検定を使う意義はなくなると思う}。前の節でみたように、分布関数に対して適切な統計量を選ばなければ$\alpha_1$が設定した有意水準$\alpha$とならないので、期待していた推論が行えないことが多くなる。


\if 0
\subsection{標本の分布がわからない事による過誤}
TODO: よくわからない。
以上により仮説検定は、既知の母集団と変異を与えた母集団との違いを図る方法の一つの方法であると言える。
母集団分布が既知でないならば統計モデルをかんで構築する必要があり、そこから変異を捉えることを試みると、
モデルの仮定と標本の特徴が一致しないことにより、過誤が生じる。

\subsubsection{独立同分布ではない事による過誤}
モデルでは、確率変数は独立であるが、自然現象の多くはなんらかの相関を持っていることが多い。
人間の身長であっても、同じ社会の中にいれば、各個人がなんからの相関を持つとも考えられる。
そんなことは、おいておいて、それぞれ独立だと思って解析してみると言うのが統計学を使った方法であるので、
自然とモデルの間にはやはりギャップが存在し、モデルの想定よりもあらい推定になることがあり得る。
\fi
\if 0
\paragraph{$p<0.05$にした理由}
https://biolab.sakura.ne.jp/statistics-5-percent.html
\fi



\if 0
\subsection{サンプルサイズが小さければ$t$検定}
西内啓 著「統計学が最強の学問である(実践編)」(ダイヤモンド社)
\url{https://biolab.sakura.ne.jp/small-sample-t-test-glm.html}
\fi 

%
\chapter{指数分布を使った統計モデル}
あるシステムの故障発生間隔日数を調べてみると、次のようなデータを得たとする(実際には、平均10の指数分布関数を使いサンプリングを行った。もちろんそんなことは忘れて、現象は数学関数により生成されていないと考える)
\begin{lstlisting}
19.9290003   0.60892905  0.55864947 29.77846887  0.28955969  1.58223429 21.10080586 10.78952122  0.59624638 15.74379646
\end{lstlisting}


平均値は10.09,分散は107.7であった。以下の仮定からなる統計モデルを構築した。
\begin{quote}
    \begin{enumerate}[(1)]
\item i.i.d
\item 正規分布
\item 平均10.09、分散107.7
\end{enumerate}
\end{quote}
では統計量を元に、$p$値を求めてみる。具体的には、

\begin{lstlisting}
Y =[19.9290003 ,  0.60892905,  0.55864947, 29.77846887,  0.28955969,  1.58223429, 21.10080586, 10.78952122,  0.59624638, 15.74379646]
stats.ttest_1samp(Y, 10.09)
\end{lstlisting}

結果、$p=0.978$である。$p=0.05$を基準にすれば、この統計モデルは棄却できない。統計モデルの仮定が妥当かを一つずつ調べる。統計モデルの仮定(2)は、分布が正規分布であることを仮定している。実際の標本をQ-Qプロットしてみよう。正規分布だと考えても問題がなさそうである。統計モデルの仮定(3)について検討する。このサンプルの平均値は$10.09$なので、これについても仮定は妥当だと考えられる。
では、帰無仮説を採択し、この統計モデルを元に推測をするべきだろうか。このモデルで推測を行なってみる。$15$日後の故障発生率は$P(X>20)=0.0036$程度である。同様に、$5$日後の故障発生率は、$P(X>5)=0.0036$である。正規分布は左右対称な関数なので、平均故障発生日から5日でも15日でも、同じ割合で故障する。 
機械などの故障は日数がたてば故障しやすくなるので、肌感覚としてあり得ないと思われる。
実際に、この統計モデルによる故障発生間隔の推測は現実と乖離していくことがわかっている。
このように、恣意的に決めた$p$より大きな統計モデルを使ったとしても、推測がうまくいくわけではないことから、棄却されなかった統計モデルを採用することには慎重になるべきである。

二つ目は、データの偏りによって、統計モデルを棄却できないことあるということである。
$10000$回実験を繰り返したとして(もちろん、指数分ぷからサンプリングを行った。もちろんそんなことは忘れて、現象は数学関数により生成されていないと考えてください)、サンプルサイズは毎回10だとすると、棄却できる割合は0程度であった。一方で、サンプルサイズを$100$程度にすると、棄却できる割合は、$0.956$程度であった。これは、平均値のあたりにデータが集まりやすく、データの端は無視できる程度の数量しか発生しないことで、正規分布を使った統計モデルを棄却しにくくなる。TODO
このように、データが十分ない場合は$p$の棄却に正規分布を仮定した統計モデルでは統計モデルを棄却する頻度は高くなることが予想される。

$p$値による判断がうまくいかないことがわかった。また、信頼区間の中にある母数でも推論ができないこともわかった。
\if 0
このような偏ったデータを手に入れた場合はどのように統計モデルを作ればいいのだろうか。
\fi

https://biolab.sakura.ne.jp/small-sample-t-test-glm.html



\subsubsection{モデルの更新}
\if 0
データ数の問題と、推測の精度の問題を解決することはできるだろうか?
尤度ひ検定
\fi
月日は流れ、データが蓄積された。その結果、次のような分布が得られた。ここまでの議論で、正規分布を仮定したモデルを棄却できないことはわかった。では、そのモデルを使って現在のデータを予測できるだろうか?パラメータを推定した正規分布とデータの分布を見てみよう。
データの最頻値が0の近くなので、平均10の正規分布では、ズレが生じることがわかる。
これではデータを捉えることはできない。
また、$30$日後までに故障が発生する確率を計算してみると、$0.971$程度であり、30よりも大きなデータの数は$0.05(1-0.05=0.95)$それなりに良く一致している。
一方で,5日までの故障発生率は、$0.30$をと予測するが、実際のデータは、0.18程度であり、こちらは乖離していると感じるだろう。
\begin{figure}
\begin{center}
    \includegraphics[width=15cm]{./image/06_/normal_exponential.pdf}
    %\caption{データの分布を正規分布で捉えれない}
\end{center}
\end{figure}


では、どのようなモデルを構築すればいいのだろうか。指数分布関数を使ってみよう
\begin{quote}
    \begin{enumerate}[(1)]
\item i.i.d
\item 指数分布
\item 母数$\lambda$
\end{enumerate}
\end{quote}
このモデルを$M(\lambda)$とかく。指数分布では、確率変数の平均は、母数$\lambda$の逆数であることがわかっている($E[x]=\frac{1}{\lambda}$)。$\lambda$として、現在手に入れたデータの平均値の逆数を代入し、データの分布と指数関数の曲線を書いてみると、よく一致しているように見える。

30日後に故障が起こる可能性は、$0.94$程度であると予想が出る。現状のデータと確認をしてみると、30よりも大きなデータの数は$0.05(1-0.05=0.95)$より、良く一致していることもわかる。
\begin{figure}
\begin{center}
    \includegraphics[width=15cm]{./image/06_/lambda_0.1.pdf}
    %\caption{しばらくした後の故障頻度の分布図}
\end{center}
\end{figure}





以上を通じて、単純に正規分布を仮定したモデルを導入すれば良し!とは言えないことがわかっただろう。言い換えれば、サンプルサイズが増えたときに見えてくる分布関数から母集団の構造を想定し、統計モデルを構築するべきだという方針が理解できる。
また、データの構造を理解した上で、推測を行うことで、データと推測が程よく一致することがわかった。


% https://biolab.sakura.ne.jp/small-sample-t-test-glm.html
\if 0
AICでモデルを選ぶと、正規分布の仮定がある。指数関数に対しては選択できないのでは? 
\fi
% https://eprints.lib.hokudai.ac.jp/dspace/bitstream/2115/49477/6/kubostat2008e.pdf
% http://www.ieice-hbkb.org/files/01/01gun_12hen_01.pdf




\paragraph{AAA}
あるとき、「改良することで故障日が伸びたみたいなんだ。調査してくれ」と依頼された。サンプルサイズは10程度であり、以下のようになった(今回も指数分布を使ってデータを生成した。もちろん実際の現象は数学の関数で生成されているわけではない)。

\begin{lstlisting}
6.46239039  7.5235678  31.84227772  6.73334029  2.9221049   1.84776618  8.7189158   2.97501827 57.78271493 20.51976339
\end{lstlisting}

統計モデルを構築しよう。
\begin{quote}
    \begin{enumerate}[(1)]
\item i.i.d
\item 正規分布
\item $\mu=10$
\end{enumerate}
\end{quote}

とする。以前のデータ解析では、指数関数を利用し、その平均値は10程度であった。今回の統計モデルでは平均$10$の正規分布を利用することで、この統計モデルが棄却できるかを試してみよう。



\begin{lstlisting}
stats.ttest_1samp(Y, 10)
\end{lstlisting}




その結果、$p=0.421$であることがわかった。このことから、有意水準$p<0.05$を満たしていないので、帰無仮説は棄却で聞いないので、改良できていないという判断を行うべきだろうか?
もちろん良くない。モデルの仮定をみると、統計モデルの仮定(2)が正規分布になっている。これは、我々が扱っている標本にはうまく適応できないことが経験的に理解してきた。
では、次のモデルはどうでしょう。

\begin{quote}
    \begin{enumerate}[(1)]
\item i.i.d
\item 指数分布
\item 指数分布の母数$\lambda=0.1$
\end{enumerate}
\end{quote}
このモデルは、故障日をよく予測してくれることがわかっている。
今回、統計モデルが正規分布ではないので、これまでの仮説検定により、統計モデルとデータの乖離を評価できない。そこで、数理統計学の知識を使う。

\subsection{指数分布関数の仮説検定}
確率変数$X_1,X_2,\cdots,X_n \sim i.i.d \ Exp(\lambda)$は、$n\bar{X}\sim Ga(1,\frac{1}{\lambda}) $ただし、$\bar{X}=X_1+X_2 \cdots +X_n$ここで、$Ga(1,\lambda)$は、尺度母数$\lambda$のガンマ分布である。統計量$\bar{X}$を利用した検定ができることが示唆される。
%https://ds.machijun.net/clear-exercise-of-statistics/%E7%AC%AC7%E7%AB%A0-%E6%8C%87%E6%95%B0%E6%AF%8D%E9%9B%86%E5%9B%A3ex%CE%BC%EF%BC%89%E3%81%AE%E6%AF%8D%E5%B9%B3%E5%9D%87%E3%81%AE%E4%BF%A1%E9%A0%BC%E5%8C%BA%E9%96%93%E3%81%A8%E6%A4%9C%E5%AE%9Ap128/



\subsection{尤度比検定}
TODO: 尤度比検定の定義

我々のデータをもとに推論をすると、
$$
\eta= \frac{\lambda_0\exp\left({-\lambda_0\sum x_i}\right)}{\bar{\lambda}^n \exp{(-n)}}
$$
ここで、$\bar{\lambda}$は最尤推定量であり、$\frac{n}{\sum x_i}$、$\lambda_0=0.1$,$n$はサンプルサイズである。
以上を元に、$\eta$を計算する。
$$
-2\log \eta \sim \chi^2_1
$$

より、$p=0.1902$と計算できる。$p$値を計算したら、やることはいつも同じで、統計モデルの仮定をもう一度調べる。統計モデルの仮定(1)はおそらく問題ない。統計モデルの仮定(2)は、少しの改良を加えただけなので、母集団の特性はほとんど変わっていなと前提を置いているので、悪くない近似ができることを期待している。統計モデルの仮定(3)は、$\lambda=0.1$である。$p>0.05$より現状では母数$\lambda$が変化しているとは言い切れない。




\begin{lstlisting}
9.70693386 14.74490149 33.03244855 21.8343649  40.73749837
\end{lstlisting}


さらにサンプルサイズが増えた。この場合、$p=0.01325$となり、$p=0.05$の有意水準を満たす。$N$数をどこまで増やすべきだったのだろうか。

TODO いつかかくけど、どこまで増やすんだろうか。
理想的には構造がわかるまで計測できたら嬉しい



\subsection{分散分析}
Section.2で行った分析では、様々な母平均に対して、統計モデルの推定がデータと一致することを確かめた。今回は、ばらつきを変化させ、統計モデルと推定の一致について考えてみよう。
統計モデルを構築しよう。
\begin{quote}
    \begin{enumerate}[(1)]
\item i.i.d
\item 正規分布$N(170,\sigma^2)$
\item 正規分布の母数$\sigma$
\end{enumerate}
\end{quote}
このモデルを、$M(\sigma)$とする。身長を予測するモデルとして、分散を替えてみよう。
Section.2で利用していた$M(5.7)$に対して、$M(10.0)$が現象を推定可能かを検討してみよう。
分布関数を書いてみると、グラフの裾野が広がったことが見て取れる。その分、ピークである$170$のあたりの頻度が減少している。つまり、$170$が出てくる頻度が下がり、より様々な身長の人がサンプリングできることが期待される。一方で、$M(2.0)$では、$170$の辺りが増え、他の場所では、頻度が現象することがわかる。$M(5.7)$よりも、身長のバラエティが少ないデータにたいし適合できる。


\begin{lstlisting}
163.54258776 179.15834405 172.6934295  166.29185695 177.65182141 165.87491547 172.08610141 158.30711988 163.74574501 176.47887419
\end{lstlisting}




拒否するべき統計モデルはどのような母数をもつだろうか。
10人から無作為抽出したデータに対して、統計モデル$M(3.1),M(12.0)$は十分データを説明できるだろうか?
統計モデルの上で、確率変数$X_1,X_2,\cdots,X_n$から計量される以下の統計量を定義する。
$$
Y_0=(n-1)\left(\frac{S_x}{\sigma}\right)^2
$$
ここで、$S_x^2=\frac{1}{n-1}\sum_{i=0}^{n}(x_i-\bar{x})^2,\bar{x}=\frac{1}{n}\sum_{i=0}^{n}x_i$である。
統計量$Y_0$は、$Y_0\sim\chi^2_{n-1}$であることがわかっている(定理\ref{fig:normal_sigma_chi2})。
このとき、$M(3.1),M(12.0)$については、$p<0.05$となり棄却される。
$p$値を基準にして、絶対にだめな統計モデル$M(3.1)$から、サンプリングを行ってみる。
\begin{lstlisting}

165.02227239, 163.16327065, 170.40109545, 170.81675656, 167.80872784, 166.91030856, 167.24096441, 170.44877048, 165.99400494, 167.59131488
\end{lstlisting}

統計モデル$M(5.7)$よりも値が広がった印象があると直ちにはわからない。
$M(3.1)$を使って、$180cm$を超える人の割合を計算すると、$P(X>180)<10^-5$となり、観測と比較してかなり少ない。
同様に、統計モデル$M(12.0)$を使って計算を行うと、$P(X>180)=0.15$となり、観測と比べて多いこともわかる。
棄却されなかった統計モデル$3.1<\sigma<12.0$の中でも、積極的に予測に利用するには、データの構造をより理解する必要がある。


\begin{figure}
\begin{center}
    \includegraphics[width=15cm]{./image/06_/normal_sigma.pdf}
    %\caption{検出力}
\end{center}
\end{figure}

\if 0
https://biolab.sakura.ne.jp/welch-test.html
https://biolab.sakura.ne.jp/welch-anova-statwing.html
https://biolab.sakura.ne.jp/welch-test.html
\fi




\section{誤差論}
%\footnote{以下の論文を読んでの意見である\cite{Error_bar_nature,krzywinski2013importance}}
誤差論は、計測に対する信頼性を定量的に扱う方法論である。
%真である理論値へ実験が近づけたことを検証することを目的にしており、理論に重点をおき、計測を疑うという点で生物学などのデータ解析とは異なる。

\subsection{標準偏差か標準誤差か}
標準偏差や標準誤差によりデータのばらつきを捉えようとすることは、正規分布を含んだモデルを使って、母集団を推測しようとする行為であると考えることができます。以下の議論は、モデルが現象をよく捉えている場合にはうまく成り立つが、そうでないなら、モデルを修正したほうが、うまく現象をうまく捉えることができる\footnote{様々な意見がある\cite{SUZUKI_SESD,池田郁男2013統計検定を理解せずに使っている人のために,池田郁男2019改訂増補版}}。
ここでは、次のモデル$M(\mu,\sigma^2)$を使う。

\begin{enumerate}
    \item i.i.d
    \item 正規分布関数
    \item $\mu,\sigma^2$
\end{enumerate}
ここで、母集団から無作為抽出した標本の標本平均と標本分散をそれぞれ、$\bar{x}=\frac{1}{n}\sum{x_i},s^2=\frac{1}{n}\sum(x_i-\bar{x})^2$である。これらを組み入れた統計モデルを$M(\bar{X},s^2)$と書く。

\subsubsection{標準偏差}
正規分布を含んだ統計モデルを仮定し、そのモデルの上で、予想されるサンプルがおよそ$68\%$の確率で出現する範囲は、母数分散$\sigma^2$より以下の範囲になります。
\begin{equation*}
    [\mu-\sigma,\mu+\sigma].
\end{equation*}
モデルの母数分散は不明な場合、母集団から無作為抽出を行なって集計した標本の偏差$s$を計算します。
\begin{equation*}
    s = \sqrt{\frac{1}{n}\sum(x_i-\bar{x})^2}.
\end{equation*}
このことから、モデルが予測するサンプルが$68\%$の確率で出現する範囲は
\begin{equation*}
    [\mu-s,\mu+s].
\end{equation*}
これを図示したものが、図です。
言い換えれば、これは、$68\%$予測区間である。


\begin{figure}
    \begin{center}
        \includegraphics[width=15cm]{../markdown/section1/Norm_standard_deviation.pdf}
        \caption{サンプルサイズに応じた標準誤差の広がり}
        %\label{fig:standard_normal_distribution}
    \end{center}
\end{figure}



\subsubsection{標準誤差}
標準誤差$SE$は、標準偏差$s$をサンプルサイズ$N$の平方根で割ったものである。
\begin{equation*}
    SE = \frac{s}{\sqrt{N}}
\end{equation*}
$\bar{x}\sim N(\mu,\sigma^2/\sqrt(N))$であるので、モデルの上で$\bar{x}$が以下の範囲に出現する確率は、およそ$68.2\%$である。
\begin{equation*}
    [\bar{\mu}-SE,\bar{\mu}+SE]
\end{equation*}
よって、$\mu$の値は一般にわからないので、標本平均$\bar{X}$を用いて、
\begin{equation*}
    [\bar{X}-SE,\bar{X}+SE]
\end{equation*}
である。
統計モデル$M(\bar{x},s^2)$からサンプリングした$\bar{X}$がこの範囲に得られる確率が$68.2\%$である\footnote{$\sigma$が変曲点であるから使ったと思われるが、なぜ、$68.2\%$または、$\bar{X}\pm SE$の範囲を使ったのかはわからなかった。誤差論の立場では、以下の記述が見つかった\cite{誤差の取り扱い_神戸大学}。
\begin{quote}
    したがって、おおよそ$70\%$の確率で誤差の絶対値は$\sigma$より小さいことがわかるので、これを測定値の信頼度の目安として用いる。
\end{quote}
}
\footnote{標準偏差に ± を付けるな!: 医療論文に多い?
\url{https://biolab.sakura.ne.jp/mean-sd.html}。まだ読めていない。$\pm SE$という表記はよろしくないらしい。}。
言い換えれば、$SE$は、$68\%$信頼区間である。


\begin{figure}
    \begin{center}
        \includegraphics[width=15cm]{../markdown/section1/Norm_SE.pdf}
        \caption{サンプルサイズに応じた標準偏差の広がり}
        %\label{fig:standard_normal_distribution}
    \end{center}
\end{figure}


\bibliography{ref} %hoge.bibから拡張子を外した名前
\bibliographystyle{junsrt}

\end{document}
%\documentclass[uplatex]{jsarticle}
\documentclass[a4paper,11pt,dvipdfmx]{jsarticle}
\usepackage{listings,jlisting}
\usepackage[dvipdfmx]{graphicx}
\usepackage{url}
\usepackage{amsthm}
\usepackage{framed}
\usepackage{svg}

\newtheorem{theo}{定理}[section]
\newtheorem{defi}{定義}[section]
\newtheorem{lemm}{補題}[section]
\newtheorem{hypoth}{仮説}[section]
%\newtheorem{Proof}{証明}[section]
\renewcommand\proofname{\bf 証明}

% https://oku.edu.mie-u.ac.jp/tex/mod/forum/discuss.php?d=2390
\usepackage{tcolorbox}
\tcbuselibrary{skins}
\newenvironment{myitemize}{%
\begin{description}}{\end{description}}
\tcolorboxenvironment{myitemize}{blanker,
before skip=6pt,after skip=12pt,top=5ex,
bottom=5ex,
borderline west={1mm}{0pt}{black!50}}
\parindent=0pt\relax % 「Some text.」と「More text.」の先頭のインデントを0にする


\newtcolorbox{mybox}[1][]{colback=red!5!white,
blanker,
before skip=6pt,after skip=12pt,top=1ex,
bottom=1ex,
borderline west={1mm}{0pt}{black!50},
%colback=white!10!white,
#1}

\usepackage{enumerate}
\lstset{
language = python,   
breaklines = true,
numbers = left,
frame = tbrl,
tabsize = 4,
captionpos = t
}


%% https://www.kerislab.jp/posts/2019-01-14-vscode-latex/
%% 
% 数式
\usepackage{amsmath,amsfonts}
\usepackage{bm}
% 画像
\usepackage[dvipdfmx]{graphicx}

\begin{document}

\title{科学統計教程}
\author{Idiot}
%\date{2020/6/29}
\maketitle
%% TODO QQplot,検出力,確率密度関数,分散分析,t分布

\if 0
\section{ \LaTeX をインストール}asd
Tex Liveをインストールしよdaasdasdsdaうasdasdasdasd
あsだあsdsasdfだsasdasdas亜sdfさdだsだswaonn和音んんnわわわわwさasdasおっっぱっっぱ

\begin{align*}
    \dot{x}      & = \cos \theta u_1          \\
    \dot{y}      & = \sin \theta u_1          \\
    \dot{\theta} & = \frac{1}{l}\tan \phi u_1 \\
    \dot{\phi}   & = u_2
\end{align*}

\begin{lstlisting}
    #include <iostream>
    using namespace std;
    
    int main(){
        int a = 0;
        cout << a << endl; # aを出力
        return 0;
    }
\end{lstlisting}
\fi 

\clearpage









\bibliography{ref} %hoge.bibから拡張子を外した名前
\bibliographystyle{junsrt}

\end{document}